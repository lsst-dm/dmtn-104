\section{Introduction}


\subsection{Obective}

The objective is to provide a clear picture on the DM management products, including description, characterization and dependencies.


\subsection{Definitions}

Following definitions are relevant for this document.


\subsubsection{Product} \label{sec:product}

A product is a component of the DM product tree, that contribute to satisfy the DM requirements.

Each product is characterize by:

\begin{itemize}
  \item \textbf{Name}: it is the main identification of the product. In some cases a \textit{short name} can be provided for documentation purposes.
  \item \textbf{Unique Key}: it is a unique string mainly for programmatic usage. The product manager/owner can provide one, or it will be assigned by the Architecture team.
  \item \textbf{Owner}: who is responsible for the quality and acceptance of a particular product (\citeds{LDM-294} \S6.6).
  \item \textbf{Manager}: {T/CAM} who has managerial and financial responsibility for the engineering teams within DM (\citeds{LDM-294} \S6.14.1).
  \item \textbf{WBS}: the WBS identification.
  \item \textbf{Team}: the team in which the product is developed.
  \item \textbf{Description}: it provides a short overview of the product purpose.
  \item \textbf{Git Package}: the Git package that implementes the product.
  \begin{itemize}
    \item \textbf{software products}: it shall be the top levelpackage as per \citeds{DMTN-106} definition. 
    \item \textbf{services}: it should be the Git package where the Dockerfile defining the service is defined. The same repository could be used to keep under configuratino control other configuration files related to kubernettes definition (except secrets).
  \end{itemize}
  \item \textbf{Dependencies}: the list of products required to implement this product.
  \item \textbf{Upstream Products}: the list is products that use this product.
  \item \textbf{Links}: internet links that provide inportant information in order to clearly define the product.
  \item \textbf{Documentation}: list of Rubin documents that characterize the product.
  \item \textbf{Requirements}: list of requirements that are related in the model to the product.
\end{itemize}


\subsubsection{Software Product} \label{sec:swproduct}

Please refer to \citeds{DMTN-106} section \$2 for the Software Product definition.


\subsubsection{Product Tree} \label{sec:ptree}

The product tree is the graphical representation of the DM products.


\subsubsection{Dependencies} \label{sec:dependencies}

The dependencies listed in this document are functional dependencies.
The aim is to provide, for each product, the list of other products needed for its implementation or execution. 


\subsection{Applicable Documents}

Following documents are applicable:

\citeds{LDM-294} Data Management Organization and Management\\
% \citeds{LDM-148} Data Management System Design\\


\subsection{Document Overview}

Information in sections \ref{sec:top} and \ref{sec:sups} is extracted from MagicDraw and provides:

\begin{itemize}
\item the DM top level product tree: it includes all products required to satisfy the DMS requirements (\citeds{LSE-61}).
\item the DM support product tree: it includes all products required for construction and maintenance.
\end{itemize}

The information regarding the Git packages implementing the above products (in sections \ref{sec:top} and \ref{sec:sups}) is provided in section \ref{sec:low}. 
This information is extracted from GitHub.

The section \ref{sec:nondm} lists all non DM products that are required in order to fulfill the DMS requirements.
This information is extracted from MagicDraw.
No GitHub package details are provided for these products. 

The last section \ref{sec:jiracomponents} will document all components in the DM Jira project.
 

\subsection{Consistency and Completeness}\label{sec:cons-comp}

The information collected in the product tree shall be consistent and complete.

In order to be consistent, following rules need to be fulfilled:

\begin{itemize}
\item Facilities: shall host Enclaves
\item Enclaves: shall host services
\item Services: shall be implemented using DM Software Products or COTS
\item DM Software Products: shall have only one GitHub package
\item Low level dependencies to be made explicit in section 4
\item Dependencies to 3rd party libraries should (not yet, but will be) derived from GitHub
\item COTS and 3rd party libraries: shall have a link to a documentation page available in the global internet
\end{itemize}

In order to be complete, following rules need to be fulfilled:

\begin{itemize}
\item All products shall have an owner
\item All packages (subsection, paragraphs) shall have a manager
In case of multiple managers in the underlying products, no manager is specified and Arch Team will review it.
\item Each manager and each owner needs ensure that:
\begin{itemize}
\item She/he is the right person of taking care of that package/product
\item The information provided for each package/product is sufficient to characterize it.
\end{itemize}
\item Architecture Team shall ensure the consistency of the document following the rules listed above.
\end{itemize}


\newpage
\section{Top Level Product Tree}\label{sec:top}

The products listed in this section are maintained in MagicDraw by the DM System Engineering group. 

These products are meant to be operational, i.e. will process the survey source data and produce L1 or L2 data products, or will be distributed to permit the external collaborators and the science community to provide their contribution to the project.

These products are organized and maintained in MagicDraw by the DM System Engineering group.

\newpage
\subsection{Data Management Operational Products}\label{sec:dmtop}


% auto generated from multiple sources - DO NOT EDIT!
% using template at <template>.
% Collecting data for component: ""
% using docsteady version 
%
% This file is meant to be included in LaTeX document in order to provide:
%   -  MagicDraw Top Level Product Tree (section 2)

\begin{longtable}{p{3.7cm}p{3.7cm}p{3.7cm}p{3.7cm}}\hline
\textbf{Manager} & \textbf{Owner} & \textbf{WBS} & \textbf{Team} \\ \hline
Wil O'Mullane &
\begin{tabular}{@{}l@{}}
Leanne Guy \\
\end{tabular} & \begin{tabular}{@{}l@{}}
1.02C \\
\end{tabular} & \begin{tabular}{@{}l@{}}
 \\
\end{tabular} \\ \hline
\multicolumn{4}{c}{
{\footnotesize ( Short Name: Data Management - Acronym: DM ) }
}\\ \hline
\end{longtable}

  This is the top level Data Management product. It includes underneath
all products required in order to satisfy the DM requirements at all
levels (LSE-61, interface and low level flow down
requirements).\\[2\baselineskip]These products are organized in the
following groups:

\begin{itemize}
\tightlist
\item
  service products
\item
  software products (developed by DM)
\item
  resource products (hardware, COTS and external provided softwares)
\item
  infrastructure products (physical or logical)
\item
  data products (reference data for development and operations)
\end{itemize}

~



\subsection{Services Products}\label{dmsrv}
\begin{longtable}{p{3.7cm}p{3.7cm}p{3.7cm}p{3.7cm}}\hline
\textbf{Manager} & \textbf{Owner} & \textbf{WBS} & \textbf{Team} \\ \hline
 &
\begin{tabular}{@{}l@{}}
Leanne Guy \\
\end{tabular} & \begin{tabular}{@{}l@{}}
 \\
\end{tabular} & \begin{tabular}{@{}l@{}}
 \\
\end{tabular} \\ \hline
\multicolumn{4}{c}{
{\footnotesize ( Short Name: Services - Acronym: DMSRV ) }
}\\ \hline
\end{longtable}

  This product includes all services required for:

\begin{itemize}
\tightlist
\item
  data distribution (DBB)
\item
  data processing (prompt and offline)
\item
  data access (LSP)
\item
  supporting activities (IT, monitoring, DBs, etc)
\end{itemize}

All services should be characterize by the implementing products and by
a configuration file (docker) maintained in GitHub.

The configuration file should follow the same development and releases
processes defined for the DM software products.

~

All services shall also specify which software product, DM or third
party, is implementing it. This information is displaced in the
\textbf{Product Dependencies} table, \emph{Depends On} column.



\subsubsection{Backbone Services}\label{dbbsrv}
\begin{longtable}{p{3.7cm}p{3.7cm}p{3.7cm}p{3.7cm}}\hline
\textbf{Manager} & \textbf{Owner} & \textbf{WBS} & \textbf{Team} \\ \hline
 &
\begin{tabular}{@{}l@{}}
Michelle Butler \\
\end{tabular} & \begin{tabular}{@{}l@{}}
 \\
\end{tabular} & \begin{tabular}{@{}l@{}}
 \\
\end{tabular} \\ \hline
\multicolumn{4}{c}{
{\footnotesize ( Short Name: DBB Services - Acronym: DBBSRV ) }
}\\ \hline
\end{longtable}

  The Backbone services supply data to other services.\\
They are implemented in the Archive enclaves.\\[2\baselineskip]Detailed
concepts of operations for each service can be found in \citeds{LDM-230}



\begin{longtable}{p{3.7cm}p{3.7cm}p{3.7cm}p{3.7cm}}\toprule
\multicolumn{2}{l}{\large \textbf{ DBB Lifetime Management } }\label{dbblifesrv}
& \multicolumn{2}{l}{(product in: DBB Services)}
\\ \hline
\textbf{\footnotesize Manager} & \textbf{\footnotesize Owner} &
\textbf{\footnotesize WBS} & \textbf{\footnotesize Team} \\ \hline
Margaret Johnson &
\begin{tabular}{@{}l@{}}
Michelle Butler \\
\end{tabular} & \begin{tabular}{@{}l@{}}
1.02C.07.07 \\
\end{tabular} & \begin{tabular}{@{}l@{}}
LDF \\
\end{tabular} \\ \hline
\multicolumn{4}{c}{
{\footnotesize ( Short Name: DBB Lifetime - Acronym: DBBLIFESRV ) }
}\\ \hline
\end{longtable}

  This service is responsible for managing the lifetimes of data products
within the DBB based on a set of policies. Data products may move from
high-speed storage to near-line or offline storage or may be deleted
completely. Some products are kept permanently. Some are kept for
defined time periods as specified in requirements. Intermediate data
products may be kept until all downstream products have been generated.



\begin{longtable}{p{3.7cm}p{3.7cm}p{3.7cm}p{3.7cm}}\hline
\textbf{\footnotesize Depends on:}  & & \textbf{\footnotesize Used in:} & \\ \hline
\multicolumn{2}{c}{
\begin{tabular}{c}
\hyperref[dbblife]{DBB Lifetime Management SW} \\ \hline
\end{tabular} }
&
\multicolumn{2}{c}{
\begin{tabular}{c}
\hyperref[encarcb]{Archive Base Enclave} \\ \hline
\hyperref[encarcn]{Archive NCSA Enclave} \\ \hline
\end{tabular} }
\\ \bottomrule
\end{longtable}

\begin{longtable}{p{3.7cm}p{3.7cm}p{3.7cm}p{3.7cm}}\toprule
\multicolumn{2}{l}{\large \textbf{ DBB Ingest/ Metadata Management } }\label{dbbmdsrv}
& \multicolumn{2}{l}{(product in: DBB Services)}
\\ \hline
\textbf{\footnotesize Manager} & \textbf{\footnotesize Owner} &
\textbf{\footnotesize WBS} & \textbf{\footnotesize Team} \\ \hline
Margaret Johnson &
\begin{tabular}{@{}l@{}}
Michelle Butler \\
\end{tabular} & \begin{tabular}{@{}l@{}}
1.02C.07.07 \\
\end{tabular} & \begin{tabular}{@{}l@{}}
LDF \\
\end{tabular} \\ \hline
\multicolumn{4}{c}{
{\footnotesize ( Short Name: DBB Metadata - Acronym: DBBMDSRV ) }
}\\ \hline
\end{longtable}

  This service within the Data Backbone is responsible for maintaining and
providing access to the metadata describing the location,
characteristics, and provenance of the data products it manages. Part of
this service involves creating the appropriate metadata during ingest
when data from external sources is incorporated into the DBB. The Batch
Production services will generally create the necessary DBB metadata as
part of their operation, so only a minimal ingest process is needed for
internally-generated data products. Metadata is kept in a database that
is a superset of the registry required by the Data Butler, allowing the
Butler to directly access data within the DBB.



\begin{longtable}{p{3.7cm}p{3.7cm}p{3.7cm}p{3.7cm}}\hline
\textbf{\footnotesize Depends on:}  & & \textbf{\footnotesize Used in:} & \\ \hline
\multicolumn{2}{c}{
\begin{tabular}{c}
\hyperref[dbbmd]{DBB Ingest/ Metadata Management SW} \\ \hline
\end{tabular} }
&
\multicolumn{2}{c}{
\begin{tabular}{c}
\hyperref[encarcb]{Archive Base Enclave} \\ \hline
\hyperref[encarcn]{Archive NCSA Enclave} \\ \hline
\end{tabular} }
\\ \bottomrule
\end{longtable}

\begin{longtable}{p{3.7cm}p{3.7cm}p{3.7cm}p{3.7cm}}\toprule
\multicolumn{2}{l}{\large \textbf{ DBB Storage } }\label{dbbstrsrv}
& \multicolumn{2}{l}{(product in: DBB Services)}
\\ \hline
\textbf{\footnotesize Manager} & \textbf{\footnotesize Owner} &
\textbf{\footnotesize WBS} & \textbf{\footnotesize Team} \\ \hline
Margaret Johnson &
\begin{tabular}{@{}l@{}}
Michelle Butler \\
\end{tabular} & \begin{tabular}{@{}l@{}}
1.02C.07.07 \\
\end{tabular} & \begin{tabular}{@{}l@{}}
LDF \\
\end{tabular} \\ \hline
\multicolumn{4}{c}{
{\footnotesize ( Short Name: DBB Storage - Acronym: DBBSTRSRV ) }
}\\ \hline
\end{longtable}

  This service is responsible for storage of data products in the DBB. The
storage service provides an interface usable by the Data Butler as a
datastore.



\begin{longtable}{p{3.7cm}p{3.7cm}p{3.7cm}p{3.7cm}}\hline
\textbf{\footnotesize Depends on:}  & & \textbf{\footnotesize Used in:} & \\ \hline
\multicolumn{2}{c}{
\begin{tabular}{c}
\hyperref[dbbtr]{DBB Transport/ Replication/ Backup SW} \\ \hline
\end{tabular} }
&
\multicolumn{2}{c}{
\begin{tabular}{c}
\hyperref[encarcb]{Archive Base Enclave} \\ \hline
\hyperref[encarcn]{Archive NCSA Enclave} \\ \hline
\end{tabular} }
\\ \bottomrule
\end{longtable}

\begin{longtable}{p{3.7cm}p{3.7cm}p{3.7cm}p{3.7cm}}\toprule
\multicolumn{2}{l}{\large \textbf{ DBB Transport/ Replication/ Backup } }\label{dbbtrsrv}
& \multicolumn{2}{l}{(product in: DBB Services)}
\\ \hline
\textbf{\footnotesize Manager} & \textbf{\footnotesize Owner} &
\textbf{\footnotesize WBS} & \textbf{\footnotesize Team} \\ \hline
Margaret Johnson &
\begin{tabular}{@{}l@{}}
Michelle Butler \\
\end{tabular} & \begin{tabular}{@{}l@{}}
1.02C.07.07 \\
\end{tabular} & \begin{tabular}{@{}l@{}}
LDF \\
\end{tabular} \\ \hline
\multicolumn{4}{c}{
{\footnotesize ( Short Name: DBB Transport - Acronym: DBBTRSRV ) }
}\\ \hline
\end{longtable}

  This service is responsible for moving data products from one Facility
to another and to backup and disaster recovery storage. It handles
recovery if a data product is found to be missing or corrupt.



\begin{longtable}{p{3.7cm}p{3.7cm}p{3.7cm}p{3.7cm}}\hline
\textbf{\footnotesize Depends on:}  & & \textbf{\footnotesize Used in:} & \\ \hline
\multicolumn{2}{c}{
\begin{tabular}{c}
\hyperref[dbbtr]{DBB Transport/ Replication/ Backup SW} \\ \hline
\end{tabular} }
&
\multicolumn{2}{c}{
\begin{tabular}{c}
\hyperref[encarcb]{Archive Base Enclave} \\ \hline
\hyperref[encarcn]{Archive NCSA Enclave} \\ \hline
\end{tabular} }
\\ \bottomrule
\end{longtable}

\subsubsection{IT Services}\label{itsrv}
\begin{longtable}{p{3.7cm}p{3.7cm}p{3.7cm}p{3.7cm}}\hline
\textbf{Manager} & \textbf{Owner} & \textbf{WBS} & \textbf{Team} \\ \hline
 &
\begin{tabular}{@{}l@{}}
 \\
\end{tabular} & \begin{tabular}{@{}l@{}}
 \\
\end{tabular} & \begin{tabular}{@{}l@{}}
 \\
\end{tabular} \\ \hline
\multicolumn{4}{c}{
{\footnotesize ( Short Name: IT Service - Acronym: ITSRV ) }
}\\ \hline
\end{longtable}

  The services listed under this product are meant to be used for
supporting activities, required bu the other products in order to work
properly.\\



\begin{longtable}{p{3.7cm}p{3.7cm}p{3.7cm}p{3.7cm}}\toprule
\multicolumn{2}{l}{\large \textbf{ EFD Cache } }\label{efdb}
& \multicolumn{2}{l}{(product in: IT Service)}
\\ \hline
\textbf{\footnotesize Manager} & \textbf{\footnotesize Owner} &
\textbf{\footnotesize WBS} & \textbf{\footnotesize Team} \\ \hline
 &
\begin{tabular}{@{}l@{}}
 \\
\end{tabular} & \begin{tabular}{@{}l@{}}
 \\
\end{tabular} & \begin{tabular}{@{}l@{}}
 \\
\end{tabular} \\ \hline
\multicolumn{4}{c}{
{\footnotesize ( Short Name: EFD Cache - Acronym: EFDB ) }
}\\ \hline
\end{longtable}

  



\begin{longtable}{p{3.7cm}p{3.7cm}p{3.7cm}p{3.7cm}}\hline
\textbf{\footnotesize Depends on:}  & & \textbf{\footnotesize Used in:} & \\ \hline
\multicolumn{2}{c}{
\begin{tabular}{c}
\hyperref[efdt]{EFD Transformation} \\ \hline
\end{tabular} }
&
\multicolumn{2}{c}{
\begin{tabular}{c}
\hyperref[encprb]{Prompt Base Enclave} \\ \hline
\end{tabular} }
\\ \bottomrule
\end{longtable}

\begin{longtable}{p{3.7cm}p{3.7cm}p{3.7cm}p{3.7cm}}\toprule
\multicolumn{2}{l}{\large \textbf{ LSP Database } }\label{lspdb}
& \multicolumn{2}{l}{(product in: IT Service)}
\\ \hline
\textbf{\footnotesize Manager} & \textbf{\footnotesize Owner} &
\textbf{\footnotesize WBS} & \textbf{\footnotesize Team} \\ \hline
 &
\begin{tabular}{@{}l@{}}
 \\
\end{tabular} & \begin{tabular}{@{}l@{}}
 \\
\end{tabular} & \begin{tabular}{@{}l@{}}
 \\
\end{tabular} \\ \hline
\multicolumn{4}{c}{
{\footnotesize ( Short Name: LSP Database - Acronym: LSPDB ) }
}\\ \hline
\end{longtable}

  Database service used by the LSP components instantiated in a DAC for
storing user data and reading available catalogs.



\begin{longtable}{p{3.7cm}p{3.7cm}p{3.7cm}p{3.7cm}}\hline
\textbf{\footnotesize Depends on:}  & & \textbf{\footnotesize Used in:} & \\ \hline
\multicolumn{2}{c}{
\begin{tabular}{c}
\hyperref[qserv]{Distributed Database} \\ \hline
\end{tabular} }
&
\multicolumn{2}{c}{
\begin{tabular}{c}
\hyperref[enccomm]{Commissioning Cluster Enclave} \\ \hline
\hyperref[encdacc]{DAC Chile Enclave} \\ \hline
\hyperref[encdacu]{DAC US Enclave} \\ \hline
\end{tabular} }
\\ \bottomrule
\end{longtable}

\begin{longtable}{p{3.7cm}p{3.7cm}p{3.7cm}p{3.7cm}}\toprule
\multicolumn{2}{l}{\large \textbf{ Network Management } }\label{netmgmt}
& \multicolumn{2}{l}{(product in: IT Service)}
\\ \hline
\textbf{\footnotesize Manager} & \textbf{\footnotesize Owner} &
\textbf{\footnotesize WBS} & \textbf{\footnotesize Team} \\ \hline
Jeff Kantor &
\begin{tabular}{@{}l@{}}
Jeff Kantor \\
\end{tabular} & \begin{tabular}{@{}l@{}}
1.02C.08.03 \\
\end{tabular} & \begin{tabular}{@{}l@{}}
Net/Base \\
\end{tabular} \\ \hline
\multicolumn{4}{c}{
{\footnotesize ( Short Name: Net Mgmt - Acronym: NETMGMT ) }
}\\ \hline
\end{longtable}

  This product is composed by tools and services that monitor networks,
control failover, manage bandwidth allocation, etc.\\



\begin{longtable}{p{3.7cm}p{3.7cm}p{3.7cm}p{3.7cm}}\hline
\textbf{\footnotesize Depends on:}  & & \textbf{\footnotesize Used in:} & \\ \hline
&
\\ \bottomrule
\end{longtable}

\subsubsection{LSP Services}\label{lspsrv}
\begin{longtable}{p{3.7cm}p{3.7cm}p{3.7cm}p{3.7cm}}\hline
\textbf{Manager} & \textbf{Owner} & \textbf{WBS} & \textbf{Team} \\ \hline
 &
\begin{tabular}{@{}l@{}}
Gregory Dubois-Felsmann \\
\end{tabular} & \begin{tabular}{@{}l@{}}
 \\
\end{tabular} & \begin{tabular}{@{}l@{}}
 \\
\end{tabular} \\ \hline
\multicolumn{4}{c}{
{\footnotesize ( Short Name: LSP Services - Acronym: LSPSRV ) }
}\\ \hline
\end{longtable}

  This group of services provide an exploratory analysis environment for
the end user. These services are usually referred as ``Aspects'',
identified as follows:

\begin{itemize}
\tightlist
\item
  LSP Portal Aspect
\item
  LSP JupiterLab Aspect
\item
  LSP WebAPIs Aspect
\end{itemize}

These services permit external users access to project resources, such
as data and processing SW.\\[2\baselineskip]More details available in
\citeds{LDM-542}and \citeds{LSE-319}



\begin{longtable}{p{3.7cm}p{3.7cm}p{3.7cm}p{3.7cm}}\toprule
\multicolumn{2}{l}{\large \textbf{ LSP JupyterLab } }\label{lspjlsrv}
& \multicolumn{2}{l}{(product in: LSP Services)}
\\ \hline
\textbf{\footnotesize Manager} & \textbf{\footnotesize Owner} &
\textbf{\footnotesize WBS} & \textbf{\footnotesize Team} \\ \hline
Frossie Economou &
\begin{tabular}{@{}l@{}}
Simon Krughoff \\
\end{tabular} & \begin{tabular}{@{}l@{}}
1.02C.10.02.02 \\
\end{tabular} & \begin{tabular}{@{}l@{}}
SQuaRE \\
\end{tabular} \\ \hline
\multicolumn{4}{c}{
{\footnotesize ( Short Name: LSP JupyterLab - Acronym: LSPJLSRV ) }
}\\ \hline
\end{longtable}

  This service provides access to a Python-oriented computational
environment, hosted at the LSST Data Access Centers. Through a Web-based
notebook interface, users are able to run Python code in close proximity
to the LSST data archive, accessing and analyzing the data and
generating derived data products.



\begin{longtable}{p{3.7cm}p{3.7cm}p{3.7cm}p{3.7cm}}\hline
\textbf{\footnotesize Depends on:}  & & \textbf{\footnotesize Used in:} & \\ \hline
\multicolumn{2}{c}{
\begin{tabular}{c}
\hyperref[lspjl]{LSP JupyterLab SW} \\ \hline
\hyperref[spdist]{Science Pipelines Distribution} \\ \hline
\end{tabular} }
&
\multicolumn{2}{c}{
\begin{tabular}{c}
\hyperref[encdacu]{DAC US Enclave} \\ \hline
\hyperref[encdacc]{DAC Chile Enclave} \\ \hline
\hyperref[enccomm]{Commissioning Cluster Enclave} \\ \hline
\end{tabular} }
\\ \bottomrule
\end{longtable}

\begin{longtable}{p{3.7cm}p{3.7cm}p{3.7cm}p{3.7cm}}\toprule
\multicolumn{2}{l}{\large \textbf{ LSP Portal } }\label{lspprtlsrv}
& \multicolumn{2}{l}{(product in: LSP Services)}
\\ \hline
\textbf{\footnotesize Manager} & \textbf{\footnotesize Owner} &
\textbf{\footnotesize WBS} & \textbf{\footnotesize Team} \\ \hline
Xiuqin Wu &
\begin{tabular}{@{}l@{}}
Gregory Dubois-Felsmann \\
\end{tabular} & \begin{tabular}{@{}l@{}}
1.02C.05.09 \\
1.02C.05.08 \\
1.02C.05.07 \\
\end{tabular} & \begin{tabular}{@{}l@{}}
SUIT \\
\end{tabular} \\ \hline
\multicolumn{4}{c}{
{\footnotesize ( Short Name: LSP Portal - Acronym: LSPPRTLSRV ) }
}\\ \hline
\end{longtable}

  This service provides Web-based query and visualization tools for all
the LSST data products.



\begin{longtable}{p{3.7cm}p{3.7cm}p{3.7cm}p{3.7cm}}\hline
\textbf{\footnotesize Depends on:}  & & \textbf{\footnotesize Used in:} & \\ \hline
\multicolumn{2}{c}{
\begin{tabular}{c}
\hyperref[suit]{SUIT} \\ \hline
\hyperref[suitoh]{SUIT Online Help} \\ \hline
\end{tabular} }
&
\multicolumn{2}{c}{
\begin{tabular}{c}
\hyperref[encdacu]{DAC US Enclave} \\ \hline
\hyperref[encdacc]{DAC Chile Enclave} \\ \hline
\hyperref[enccomm]{Commissioning Cluster Enclave} \\ \hline
\end{tabular} }
\\ \bottomrule
\end{longtable}

\paragraph{LSP Web API}\label{lspwapi}\mbox{}\\
\begin{longtable}{p{3.7cm}p{3.7cm}p{3.7cm}p{3.7cm}}\hline
\textbf{Manager} & \textbf{Owner} & \textbf{WBS} & \textbf{Team} \\ \hline
 &
\begin{tabular}{@{}l@{}}
 \\
\end{tabular} & \begin{tabular}{@{}l@{}}
 \\
\end{tabular} & \begin{tabular}{@{}l@{}}
 \\
\end{tabular} \\ \hline
\multicolumn{4}{c}{
{\footnotesize ( Short Name: LSP Web API - Acronym: LSPWAPI ) }
}\\ \hline
\end{longtable}

  The API Aspect provides remote access to the LSST data, user data, and
user computing resources, through a set of Web APIs (many based on VO
standards). The Web APIs will deliver data in community-standard
formats, including, e.g., VOTable, CSV, and FITS. The same Web APIs are
used internally in the Portal Aspect, and are also available in the
Notebook Aspect.



\begin{longtable}{p{3.7cm}p{3.7cm}p{3.7cm}p{3.7cm}}\toprule
\multicolumn{2}{l}{\large \textbf{ SIA } }\label{siasrv}
& \multicolumn{2}{l}{(product in: LSP Web API)}
\\ \hline
\textbf{\footnotesize Manager} & \textbf{\footnotesize Owner} &
\textbf{\footnotesize WBS} & \textbf{\footnotesize Team} \\ \hline
 &
\begin{tabular}{@{}l@{}}
 \\
\end{tabular} & \begin{tabular}{@{}l@{}}
 \\
\end{tabular} & \begin{tabular}{@{}l@{}}
 \\
\end{tabular} \\ \hline
\multicolumn{4}{c}{
{\footnotesize ( Short Name: SIA - Acronym: SIASRV ) }
}\\ \hline
\end{longtable}

  



\begin{longtable}{p{3.7cm}p{3.7cm}p{3.7cm}p{3.7cm}}\hline
\textbf{\footnotesize Depends on:}  & & \textbf{\footnotesize Used in:} & \\ \hline
&
\multicolumn{2}{c}{
\begin{tabular}{c}
\hyperref[enccomm]{Commissioning Cluster Enclave} \\ \hline
\hyperref[encdacc]{DAC Chile Enclave} \\ \hline
\hyperref[encdacu]{DAC US Enclave} \\ \hline
\end{tabular} }
\\ \bottomrule
\end{longtable}

\begin{longtable}{p{3.7cm}p{3.7cm}p{3.7cm}p{3.7cm}}\toprule
\multicolumn{2}{l}{\large \textbf{ SODA } }\label{soda}
& \multicolumn{2}{l}{(product in: LSP Web API)}
\\ \hline
\textbf{\footnotesize Manager} & \textbf{\footnotesize Owner} &
\textbf{\footnotesize WBS} & \textbf{\footnotesize Team} \\ \hline
 &
\begin{tabular}{@{}l@{}}
 \\
\end{tabular} & \begin{tabular}{@{}l@{}}
 \\
\end{tabular} & \begin{tabular}{@{}l@{}}
 \\
\end{tabular} \\ \hline
\multicolumn{4}{c}{
{\footnotesize ( Short Name: SODA - Acronym: SODA ) }
}\\ \hline
\end{longtable}

  



\begin{longtable}{p{3.7cm}p{3.7cm}p{3.7cm}p{3.7cm}}\hline
\textbf{\footnotesize Depends on:}  & & \textbf{\footnotesize Used in:} & \\ \hline
\multicolumn{2}{c}{
\begin{tabular}{c}
\hyperref[daximg]{Image/ Cutout Server} \\ \hline
\end{tabular} }
&
\multicolumn{2}{c}{
\begin{tabular}{c}
\hyperref[enccomm]{Commissioning Cluster Enclave} \\ \hline
\hyperref[encdacc]{DAC Chile Enclave} \\ \hline
\hyperref[encdacu]{DAC US Enclave} \\ \hline
\end{tabular} }
\\ \bottomrule
\end{longtable}

\begin{longtable}{p{3.7cm}p{3.7cm}p{3.7cm}p{3.7cm}}\toprule
\multicolumn{2}{l}{\large \textbf{ TAP } }\label{tapsev}
& \multicolumn{2}{l}{(product in: LSP Web API)}
\\ \hline
\textbf{\footnotesize Manager} & \textbf{\footnotesize Owner} &
\textbf{\footnotesize WBS} & \textbf{\footnotesize Team} \\ \hline
 &
\begin{tabular}{@{}l@{}}
 \\
\end{tabular} & \begin{tabular}{@{}l@{}}
 \\
\end{tabular} & \begin{tabular}{@{}l@{}}
 \\
\end{tabular} \\ \hline
\multicolumn{4}{c}{
{\footnotesize ( Short Name: TAP - Acronym: TAPSEV ) }
}\\ \hline
\end{longtable}

  



\begin{longtable}{p{3.7cm}p{3.7cm}p{3.7cm}p{3.7cm}}\hline
\textbf{\footnotesize Depends on:}  & & \textbf{\footnotesize Used in:} & \\ \hline
\multicolumn{2}{c}{
\begin{tabular}{c}
\hyperref[tapsw]{TAP SW} \\ \hline
\end{tabular} }
&
\multicolumn{2}{c}{
\begin{tabular}{c}
\hyperref[enccomm]{Commissioning Cluster Enclave} \\ \hline
\hyperref[encdacc]{DAC Chile Enclave} \\ \hline
\hyperref[encdacu]{DAC US Enclave} \\ \hline
\end{tabular} }
\\ \bottomrule
\end{longtable}

\begin{longtable}{p{3.7cm}p{3.7cm}p{3.7cm}p{3.7cm}}\toprule
\multicolumn{2}{l}{\large \textbf{ WebDAV } }\label{wdav}
& \multicolumn{2}{l}{(product in: LSP Web API)}
\\ \hline
\textbf{\footnotesize Manager} & \textbf{\footnotesize Owner} &
\textbf{\footnotesize WBS} & \textbf{\footnotesize Team} \\ \hline
 &
\begin{tabular}{@{}l@{}}
 \\
\end{tabular} & \begin{tabular}{@{}l@{}}
 \\
\end{tabular} & \begin{tabular}{@{}l@{}}
 \\
\end{tabular} \\ \hline
\multicolumn{4}{c}{
{\footnotesize ( Short Name: WebDAV - Acronym: WDAV ) }
}\\ \hline
\end{longtable}

  This service permits the use of filesystem over http.



\begin{longtable}{p{3.7cm}p{3.7cm}p{3.7cm}p{3.7cm}}\hline
\multicolumn{2}{r}{\textbf{GutHub Packages:}} &
\multicolumn{2}{l}{\href{https://github.com/lsst/davt}{davt} }
\\ \hline \\ \hline
\textbf{\footnotesize Depends on:}  & & \textbf{\footnotesize Used in:} & \\ \hline
&
\multicolumn{2}{c}{
\begin{tabular}{c}
\hyperref[enccomm]{Commissioning Cluster Enclave} \\ \hline
\hyperref[encdacc]{DAC Chile Enclave} \\ \hline
\hyperref[encdacu]{DAC US Enclave} \\ \hline
\end{tabular} }
\\ \bottomrule
\end{longtable}

\begin{longtable}{p{3.7cm}p{3.7cm}p{3.7cm}p{3.7cm}}\toprule
\multicolumn{2}{l}{\large \textbf{ LSP Web API } }\label{lspwebsrv}
& \multicolumn{2}{l}{(product in: LSP Services)}
\\ \hline
\textbf{\footnotesize Manager} & \textbf{\footnotesize Owner} &
\textbf{\footnotesize WBS} & \textbf{\footnotesize Team} \\ \hline
Fritz Mueller &
\begin{tabular}{@{}l@{}}
Colin Slater \\
\end{tabular} & \begin{tabular}{@{}l@{}}
1.02C.06.02 \\
\end{tabular} & \begin{tabular}{@{}l@{}}
DAX \\
\end{tabular} \\ \hline
\multicolumn{4}{c}{
{\footnotesize ( Short Name: LSP Web API - Acronym: LSPWEBSRV ) }
}\\ \hline
\end{longtable}

  {[}OBSOLETE{]} replaced by theLSP WebAPI package and corresponding
serves in it.



\begin{longtable}{p{3.7cm}p{3.7cm}p{3.7cm}p{3.7cm}}\hline
\textbf{\footnotesize Depends on:}  & & \textbf{\footnotesize Used in:} & \\ \hline
&
\multicolumn{2}{c}{
\begin{tabular}{c}
\hyperref[encdacu]{DAC US Enclave} \\ \hline
\hyperref[encdacc]{DAC Chile Enclave} \\ \hline
\hyperref[enccomm]{Commissioning Cluster Enclave} \\ \hline
\end{tabular} }
\\ \bottomrule
\end{longtable}

\subsubsection{Offline Services}\label{offlsrv}
\begin{longtable}{p{3.7cm}p{3.7cm}p{3.7cm}p{3.7cm}}\hline
\textbf{Manager} & \textbf{Owner} & \textbf{WBS} & \textbf{Team} \\ \hline
 &
\begin{tabular}{@{}l@{}}
Multiple \\
\end{tabular} & \begin{tabular}{@{}l@{}}
 \\
\end{tabular} & \begin{tabular}{@{}l@{}}
 \\
\end{tabular} \\ \hline
\multicolumn{4}{c}{
{\footnotesize ( Short Name: Offline Services - Acronym: OFFLSRV ) }
}\\ \hline
\end{longtable}

  DM services that do not need to run in a prompt manner.



\begin{longtable}{p{3.7cm}p{3.7cm}p{3.7cm}p{3.7cm}}\toprule
\multicolumn{2}{l}{\large \textbf{ Bulk Distribution } }\label{bulkdsrv}
& \multicolumn{2}{l}{(product in: Offline Services)}
\\ \hline
\textbf{\footnotesize Manager} & \textbf{\footnotesize Owner} &
\textbf{\footnotesize WBS} & \textbf{\footnotesize Team} \\ \hline
Margaret Johnson &
\begin{tabular}{@{}l@{}}
Michelle Butler \\
\end{tabular} & \begin{tabular}{@{}l@{}}
1.02C.07.06.02 \\
\end{tabular} & \begin{tabular}{@{}l@{}}
LDF \\
\end{tabular} \\ \hline
\multicolumn{4}{c}{
{\footnotesize ( Short Name: Bulk Distrib - Acronym: BULKDSRV ) }
}\\ \hline
\end{longtable}

  Bulk Data Distribution Service



\begin{longtable}{p{3.7cm}p{3.7cm}p{3.7cm}p{3.7cm}}\hline
\textbf{\footnotesize Depends on:}  & & \textbf{\footnotesize Used in:} & \\ \hline
\multicolumn{2}{c}{
\begin{tabular}{c}
\hyperref[rucio]{Rucio} \\ \hline
\end{tabular} }
&
\multicolumn{2}{c}{
\begin{tabular}{c}
\hyperref[encoffl]{Offline Production Enclave} \\ \hline
\end{tabular} }
\\ \bottomrule
\end{longtable}

\begin{longtable}{p{3.7cm}p{3.7cm}p{3.7cm}p{3.7cm}}\toprule
\multicolumn{2}{l}{\large \textbf{ Offline Quality Control } }\label{offlqcsrv}
& \multicolumn{2}{l}{(product in: Offline Services)}
\\ \hline
\textbf{\footnotesize Manager} & \textbf{\footnotesize Owner} &
\textbf{\footnotesize WBS} & \textbf{\footnotesize Team} \\ \hline
Yusra AlSayyad &
\begin{tabular}{@{}l@{}}
Jim Bosch \\
\end{tabular} & \begin{tabular}{@{}l@{}}
1.02C.04.07 \\
\end{tabular} & \begin{tabular}{@{}l@{}}
DRP \\
\end{tabular} \\ \hline
\multicolumn{4}{c}{
{\footnotesize ( Short Name: Offline QC - Acronym: OFFLQCSRV ) }
}\\ \hline
\end{longtable}

  This service collects all KPI metrics calculated by the offline
processing activities, aggregate the information and provides reports in
order to ensure the monitoring of the quality of the processing and of
the data collected by the survey.



\begin{longtable}{p{3.7cm}p{3.7cm}p{3.7cm}p{3.7cm}}\hline
\textbf{\footnotesize Depends on:}  & & \textbf{\footnotesize Used in:} & \\ \hline
\multicolumn{2}{c}{
\begin{tabular}{c}
\hyperref[qcsw]{Quality Control SW} \\ \hline
\end{tabular} }
&
\multicolumn{2}{c}{
\begin{tabular}{c}
\hyperref[encprn]{Prompt NCSA Enclave} \\ \hline
\hyperref[encoffl]{Offline Production Enclave} \\ \hline
\end{tabular} }
\\ \bottomrule
\end{longtable}

\begin{longtable}{p{3.7cm}p{3.7cm}p{3.7cm}p{3.7cm}}\toprule
\multicolumn{2}{l}{\large \textbf{ Batch Production } }\label{prodsrv}
& \multicolumn{2}{l}{(product in: Offline Services)}
\\ \hline
\textbf{\footnotesize Manager} & \textbf{\footnotesize Owner} &
\textbf{\footnotesize WBS} & \textbf{\footnotesize Team} \\ \hline
Margaret Johnson &
\begin{tabular}{@{}l@{}}
Michelle Butler \\
\end{tabular} & \begin{tabular}{@{}l@{}}
1.02C.07.06.02 \\
\end{tabular} & \begin{tabular}{@{}l@{}}
LDF \\
\end{tabular} \\ \hline
\multicolumn{4}{c}{
{\footnotesize ( Short Name: Batch Production - Acronym: PRODSRV ) }
}\\ \hline
\end{longtable}

  Batch Production Service



\begin{longtable}{p{3.7cm}p{3.7cm}p{3.7cm}p{3.7cm}}\hline
\textbf{\footnotesize Depends on:}  & & \textbf{\footnotesize Used in:} & \\ \hline
\multicolumn{2}{c}{
\begin{tabular}{c}
\hyperref[wlwf]{Workload/ Workflow Management} \\ \hline
\hyperref[htcondor]{HTCondor} \\ \hline
\hyperref[dmcal]{Calibration SW} \\ \hline
\hyperref[drp]{Data Release Production} \\ \hline
\hyperref[mops]{MOPS and Forced Photometry} \\ \hline
\hyperref[sp]{Special Programs Productions} \\ \hline
\end{tabular} }
&
\multicolumn{2}{c}{
\begin{tabular}{c}
\hyperref[encoffl]{Offline Production Enclave} \\ \hline
\end{tabular} }
\\ \bottomrule
\end{longtable}

\subsubsection{Prompt Services}\label{prsrv}
\begin{longtable}{p{3.7cm}p{3.7cm}p{3.7cm}p{3.7cm}}\hline
\textbf{Manager} & \textbf{Owner} & \textbf{WBS} & \textbf{Team} \\ \hline
 &
\begin{tabular}{@{}l@{}}
Multiple \\
\end{tabular} & \begin{tabular}{@{}l@{}}
 \\
\end{tabular} & \begin{tabular}{@{}l@{}}
 \\
\end{tabular} \\ \hline
\multicolumn{4}{c}{
{\footnotesize ( Short Name: Prompt Services - Acronym: PRSRV ) }
}\\ \hline
\end{longtable}

  DM Services that need to run in a prompt manner.



\begin{longtable}{p{3.7cm}p{3.7cm}p{3.7cm}p{3.7cm}}\toprule
\multicolumn{2}{l}{\large \textbf{ Alert Distribution } }\label{alrtdstsrv}
& \multicolumn{2}{l}{(product in: Prompt Services)}
\\ \hline
\textbf{\footnotesize Manager} & \textbf{\footnotesize Owner} &
\textbf{\footnotesize WBS} & \textbf{\footnotesize Team} \\ \hline
John Swinbank &
\begin{tabular}{@{}l@{}}
Eric Bellm \\
\end{tabular} & \begin{tabular}{@{}l@{}}
1.02C.03.03 \\
\end{tabular} & \begin{tabular}{@{}l@{}}
AP \\
\end{tabular} \\ \hline
\multicolumn{4}{c}{
{\footnotesize ( Short Name: Alert Distrib - Acronym: ALRTDSTSRV ) }
}\\ \hline
\end{longtable}

  Alert Distribution and Filtering Service



\begin{longtable}{p{3.7cm}p{3.7cm}p{3.7cm}p{3.7cm}}\hline
\textbf{\footnotesize Depends on:}  & & \textbf{\footnotesize Used in:} & \\ \hline
\multicolumn{2}{c}{
\begin{tabular}{c}
\hyperref[alrtdstr]{Alert Distribution SW} \\ \hline
\end{tabular} }
&
\multicolumn{2}{c}{
\begin{tabular}{c}
\hyperref[encprn]{Prompt NCSA Enclave} \\ \hline
\end{tabular} }
\\ \bottomrule
\end{longtable}

\begin{longtable}{p{3.7cm}p{3.7cm}p{3.7cm}p{3.7cm}}\toprule
\multicolumn{2}{l}{\large \textbf{ Archiving } }\label{arcsrv}
& \multicolumn{2}{l}{(product in: Prompt Services)}
\\ \hline
\textbf{\footnotesize Manager} & \textbf{\footnotesize Owner} &
\textbf{\footnotesize WBS} & \textbf{\footnotesize Team} \\ \hline
Margaret Johnson &
\begin{tabular}{@{}l@{}}
Felipe Menanteau \\
\end{tabular} & \begin{tabular}{@{}l@{}}
1.02C.07.06.02 \\
\end{tabular} & \begin{tabular}{@{}l@{}}
LDF \\
\end{tabular} \\ \hline
\multicolumn{4}{c}{
{\footnotesize ( Short Name: Archiving - Acronym: ARCSRV ) }
}\\ \hline
\end{longtable}

  This component describes following services:\\[2\baselineskip]- Image
archiver for LSSTCam\\
- Image archiver for ComCam\\
- Image archiver for Auxiliary Telescope Spectrograph\\
- Header Generator service\\
- EFD Transformation service\\[2\baselineskip]The archiver services
capture raw images taken by each camera, including the wavefront sensors
and the guide sensors of the LSSTCam or ComCam when so configured,
retrieving them from their respective Camera Data System
instances.\\[2\baselineskip]The Header Generator service, written by
Data Management but operated by the Observatory, captures specific sets
of metadata associated with the images, including telemetry values and
event timings, from the OCS publish/subscribe middleware and/or from the
EFD. It formats these into a metadata package that is recorded in the
EFD Large File Annex. The Archiver and CatchUp Archiver instances
retrieve this metadata package and attach it to the captured image
pixels.\\[2\baselineskip]The image pixels and metadata are passed to the
Observatory Operations Data Service (OODS), which serves as a buffer
from which observing-critical data can be retrieved. They are also
passed to a staging area for ingestion into the permanent archive in the
Data Backbone. The catch-up versions archive into the OODS and Data
Backbone any raw images and metadata that were missed by the primary
archiving services due to network or other outage, retrieving them from
the flash storage in the Camera Data System instances and the
EFD.\\[2\baselineskip]The ~EFD Transformation service extracts all
information (including telemetry, events, configurations, and commands)
from the EFD and its large file annex, transforms it into a form more
suitable for querying by image timestamp, and loads it into the
permanently archived ``Transformed EFD'' database in the Data Backbone.



\begin{longtable}{p{3.7cm}p{3.7cm}p{3.7cm}p{3.7cm}}\hline
\textbf{\footnotesize Depends on:}  & & \textbf{\footnotesize Used in:} & \\ \hline
\multicolumn{2}{c}{
\begin{tabular}{c}
\hyperref[iip]{Image Ingest and Processing} \\ \hline
\hyperref[efdt]{EFD Transformation} \\ \hline
\hyperref[header]{Header Service SW} \\ \hline
\end{tabular} }
&
\multicolumn{2}{c}{
\begin{tabular}{c}
\hyperref[encprb]{Prompt Base Enclave} \\ \hline
\end{tabular} }
\\ \bottomrule
\end{longtable}

\begin{longtable}{p{3.7cm}p{3.7cm}p{3.7cm}p{3.7cm}}\toprule
\multicolumn{2}{l}{\large \textbf{ OCS-Driven Batch } }\label{ocsbatsrv}
& \multicolumn{2}{l}{(product in: Prompt Services)}
\\ \hline
\textbf{\footnotesize Manager} & \textbf{\footnotesize Owner} &
\textbf{\footnotesize WBS} & \textbf{\footnotesize Team} \\ \hline
Margaret Johnson &
\begin{tabular}{@{}l@{}}
Felipe Menanteau \\
\end{tabular} & \begin{tabular}{@{}l@{}}
1.02C.07.06.02 \\
\end{tabular} & \begin{tabular}{@{}l@{}}
LDF \\
\end{tabular} \\ \hline
\multicolumn{4}{c}{
{\footnotesize ( Short Name: OCS Batch - Acronym: OCSBATSRV ) }
}\\ \hline
\end{longtable}

  OCS Driven Batch Processing Service



\begin{longtable}{p{3.7cm}p{3.7cm}p{3.7cm}p{3.7cm}}\hline
\textbf{\footnotesize Depends on:}  & & \textbf{\footnotesize Used in:} & \\ \hline
\multicolumn{2}{c}{
\begin{tabular}{c}
\hyperref[ocsbat]{OCS Batch SW} \\ \hline
\end{tabular} }
&
\multicolumn{2}{c}{
\begin{tabular}{c}
\hyperref[encprb]{Prompt Base Enclave} \\ \hline
\end{tabular} }
\\ \bottomrule
\end{longtable}

\begin{longtable}{p{3.7cm}p{3.7cm}p{3.7cm}p{3.7cm}}\toprule
\multicolumn{2}{l}{\large \textbf{ Observatory Operations Data } }\label{oodssrv}
& \multicolumn{2}{l}{(product in: Prompt Services)}
\\ \hline
\textbf{\footnotesize Manager} & \textbf{\footnotesize Owner} &
\textbf{\footnotesize WBS} & \textbf{\footnotesize Team} \\ \hline
Margaret Johnson &
\begin{tabular}{@{}l@{}}
Felipe Menanteau \\
\end{tabular} & \begin{tabular}{@{}l@{}}
1.02C.07.06.02 \\
\end{tabular} & \begin{tabular}{@{}l@{}}
LDF \\
\end{tabular} \\ \hline
\multicolumn{4}{c}{
{\footnotesize ( Short Name: Obs Ops Data - Acronym: OODSSRV ) }
}\\ \hline
\end{longtable}

  Observatory Operations Data Service



\begin{longtable}{p{3.7cm}p{3.7cm}p{3.7cm}p{3.7cm}}\hline
\textbf{\footnotesize Depends on:}  & & \textbf{\footnotesize Used in:} & \\ \hline
\multicolumn{2}{c}{
\begin{tabular}{c}
\hyperref[oods]{Observatory Operations Data Service SW} \\ \hline
\end{tabular} }
&
\multicolumn{2}{c}{
\begin{tabular}{c}
\hyperref[encprb]{Prompt Base Enclave} \\ \hline
\end{tabular} }
\\ \bottomrule
\end{longtable}

\begin{longtable}{p{3.7cm}p{3.7cm}p{3.7cm}p{3.7cm}}\toprule
\multicolumn{2}{l}{\large \textbf{ Planned Observation Publication } }\label{popsrv}
& \multicolumn{2}{l}{(product in: Prompt Services)}
\\ \hline
\textbf{\footnotesize Manager} & \textbf{\footnotesize Owner} &
\textbf{\footnotesize WBS} & \textbf{\footnotesize Team} \\ \hline
Margaret Johnson &
\begin{tabular}{@{}l@{}}
Felipe Menanteau \\
\end{tabular} & \begin{tabular}{@{}l@{}}
1.02C.07.06.02 \\
\end{tabular} & \begin{tabular}{@{}l@{}}
LDF \\
\end{tabular} \\ \hline
\multicolumn{4}{c}{
{\footnotesize ( Short Name: Planned Obs Pub - Acronym: POPSRV ) }
}\\ \hline
\end{longtable}

  This service receives telemetry from the OCS describing the next visit
location and the telescope scheduler's predictions of its future
observations. It publishes these as an unauthenticated,
globally-accessible web service comprising both a web page for human
inspection and a web API for usage by automated tools.



\begin{longtable}{p{3.7cm}p{3.7cm}p{3.7cm}p{3.7cm}}\hline
\textbf{\footnotesize Depends on:}  & & \textbf{\footnotesize Used in:} & \\ \hline
\multicolumn{2}{c}{
\begin{tabular}{c}
\hyperref[obspub]{Planned Observation Publication SW} \\ \hline
\end{tabular} }
&
\multicolumn{2}{c}{
\begin{tabular}{c}
\hyperref[encprb]{Prompt Base Enclave} \\ \hline
\end{tabular} }
\\ \bottomrule
\end{longtable}

\begin{longtable}{p{3.7cm}p{3.7cm}p{3.7cm}p{3.7cm}}\toprule
\multicolumn{2}{l}{\large \textbf{ Prompt Processing Ingest } }\label{prpingsrv}
& \multicolumn{2}{l}{(product in: Prompt Services)}
\\ \hline
\textbf{\footnotesize Manager} & \textbf{\footnotesize Owner} &
\textbf{\footnotesize WBS} & \textbf{\footnotesize Team} \\ \hline
Margaret Johnson &
\begin{tabular}{@{}l@{}}
Felipe Menanteau \\
\end{tabular} & \begin{tabular}{@{}l@{}}
1.02C.07.06.02 \\
\end{tabular} & \begin{tabular}{@{}l@{}}
LDF \\
\end{tabular} \\ \hline
\multicolumn{4}{c}{
{\footnotesize ( Short Name: Prompt Proc Ing - Acronym: PRPINGSRV ) }
}\\ \hline
\end{longtable}

  This service is implemented in two instances that capture
crosstalk-corrected images from the LSSTCam and ComCam Camera Data
Systems along with selected metadata from the OCS and/or EFD and
transfer them to the Prompt Processing service in the Prompt NCSA
Enclave. There is no Prompt Processing Ingest instance for the auxiliary
telescope spectrograph.



\begin{longtable}{p{3.7cm}p{3.7cm}p{3.7cm}p{3.7cm}}\hline
\textbf{\footnotesize Depends on:}  & & \textbf{\footnotesize Used in:} & \\ \hline
\multicolumn{2}{c}{
\begin{tabular}{c}
\hyperref[iip]{Image Ingest and Processing} \\ \hline
\end{tabular} }
&
\multicolumn{2}{c}{
\begin{tabular}{c}
\hyperref[encprb]{Prompt Base Enclave} \\ \hline
\hyperref[encprn]{Prompt NCSA Enclave} \\ \hline
\end{tabular} }
\\ \bottomrule
\end{longtable}

\begin{longtable}{p{3.7cm}p{3.7cm}p{3.7cm}p{3.7cm}}\toprule
\multicolumn{2}{l}{\large \textbf{ Prompt Processing } }\label{prprsrv}
& \multicolumn{2}{l}{(product in: Prompt Services)}
\\ \hline
\textbf{\footnotesize Manager} & \textbf{\footnotesize Owner} &
\textbf{\footnotesize WBS} & \textbf{\footnotesize Team} \\ \hline
 &
\begin{tabular}{@{}l@{}}
 \\
\end{tabular} & \begin{tabular}{@{}l@{}}
 \\
\end{tabular} & \begin{tabular}{@{}l@{}}
 \\
\end{tabular} \\ \hline
\multicolumn{4}{c}{
{\footnotesize ( Short Name: Prmpt Processing - Acronym: PRPRSRV ) }
}\\ \hline
\end{longtable}

  



\begin{longtable}{p{3.7cm}p{3.7cm}p{3.7cm}p{3.7cm}}\hline
\textbf{\footnotesize Depends on:}  & & \textbf{\footnotesize Used in:} & \\ \hline
\multicolumn{2}{c}{
\begin{tabular}{c}
\hyperref[apprmpt]{Alert Production} \\ \hline
\hyperref[dmcal]{Calibration SW} \\ \hline
\hyperref[mops]{MOPS and Forced Photometry} \\ \hline
\end{tabular} }
&
\multicolumn{2}{c}{
\begin{tabular}{c}
\hyperref[encprn]{Prompt NCSA Enclave} \\ \hline
\end{tabular} }
\\ \bottomrule
\end{longtable}

\begin{longtable}{p{3.7cm}p{3.7cm}p{3.7cm}p{3.7cm}}\toprule
\multicolumn{2}{l}{\large \textbf{ Prompt Quality Control } }\label{prqcsrv}
& \multicolumn{2}{l}{(product in: Prompt Services)}
\\ \hline
\textbf{\footnotesize Manager} & \textbf{\footnotesize Owner} &
\textbf{\footnotesize WBS} & \textbf{\footnotesize Team} \\ \hline
John Swinbank &
\begin{tabular}{@{}l@{}}
Eric Bellm \\
\end{tabular} & \begin{tabular}{@{}l@{}}
1.02C.03.08 \\
\end{tabular} & \begin{tabular}{@{}l@{}}
AP \\
\end{tabular} \\ \hline
\multicolumn{4}{c}{
{\footnotesize ( Short Name: Prompt QC - Acronym: PRQCSRV ) }
}\\ \hline
\end{longtable}

  This service collects all KPI metrics calculated by the prompt
processing service, aggregate the information and provides reports in
order to ensure the monitoring of the quality of the processing and of
the data collected by the survey.



\begin{longtable}{p{3.7cm}p{3.7cm}p{3.7cm}p{3.7cm}}\hline
\textbf{\footnotesize Depends on:}  & & \textbf{\footnotesize Used in:} & \\ \hline
\multicolumn{2}{c}{
\begin{tabular}{c}
\hyperref[qcsw]{Quality Control SW} \\ \hline
\end{tabular} }
&
\multicolumn{2}{c}{
\begin{tabular}{c}
\hyperref[encprn]{Prompt NCSA Enclave} \\ \hline
\end{tabular} }
\\ \bottomrule
\end{longtable}

\begin{longtable}{p{3.7cm}p{3.7cm}p{3.7cm}p{3.7cm}}\toprule
\multicolumn{2}{l}{\large \textbf{ Telemetry Gateway } }\label{tmgsrv}
& \multicolumn{2}{l}{(product in: Prompt Services)}
\\ \hline
\textbf{\footnotesize Manager} & \textbf{\footnotesize Owner} &
\textbf{\footnotesize WBS} & \textbf{\footnotesize Team} \\ \hline
Margaret Johnson &
\begin{tabular}{@{}l@{}}
Felipe Menanteau \\
\end{tabular} & \begin{tabular}{@{}l@{}}
1.02C.07.06.02 \\
\end{tabular} & \begin{tabular}{@{}l@{}}
LDF \\
\end{tabular} \\ \hline
\multicolumn{4}{c}{
{\footnotesize ( Short Name: Telem Gateway - Acronym: TMGSRV ) }
}\\ \hline
\end{longtable}

  This service is implemented by the DMCS, a more general system that
manages communication from and to the Observatory. The DMCS is developed
as part of the Image Ingest and Processing software product.



\begin{longtable}{p{3.7cm}p{3.7cm}p{3.7cm}p{3.7cm}}\hline
\textbf{\footnotesize Depends on:}  & & \textbf{\footnotesize Used in:} & \\ \hline
\multicolumn{2}{c}{
\begin{tabular}{c}
\hyperref[iip]{Image Ingest and Processing} \\ \hline
\end{tabular} }
&
\multicolumn{2}{c}{
\begin{tabular}{c}
\hyperref[encprb]{Prompt Base Enclave} \\ \hline
\end{tabular} }
\\ \bottomrule
\end{longtable}

\subsection{Software Products}\label{dmsw}
\begin{longtable}{p{3.7cm}p{3.7cm}p{3.7cm}p{3.7cm}}\hline
\textbf{Manager} & \textbf{Owner} & \textbf{WBS} & \textbf{Team} \\ \hline
 &
\begin{tabular}{@{}l@{}}
Multiple \\
\end{tabular} & \begin{tabular}{@{}l@{}}
 \\
\end{tabular} & \begin{tabular}{@{}l@{}}
 \\
\end{tabular} \\ \hline
\multicolumn{4}{c}{
{\footnotesize ( Short Name: Software Products - Acronym: DMSW ) }
}\\ \hline
\end{longtable}

  DM operational SW products. This include all SW products implemented by
the DM team in order to implement the operational services.\\

Dependencies should be derived from the corresponding Git packages
definition.



\subsubsection{Batch Production Products}\label{bpp}
\begin{longtable}{p{3.7cm}p{3.7cm}p{3.7cm}p{3.7cm}}\hline
\textbf{Manager} & \textbf{Owner} & \textbf{WBS} & \textbf{Team} \\ \hline
 &
\begin{tabular}{@{}l@{}}
Michelle Butler \\
\end{tabular} & \begin{tabular}{@{}l@{}}
 \\
\end{tabular} & \begin{tabular}{@{}l@{}}
 \\
\end{tabular} \\ \hline
\multicolumn{4}{c}{
{\footnotesize ( Short Name: Batch Prod SW - Acronym: BPP ) }
}\\ \hline
\end{longtable}

  Software products to orchestrate the long term data processing.



\begin{longtable}{p{3.7cm}p{3.7cm}p{3.7cm}p{3.7cm}}\toprule
\multicolumn{2}{l}{\large \textbf{ Campaign Management } }\label{cmpgn}
& \multicolumn{2}{l}{(product in: Batch Prod SW)}
\\ \hline
\textbf{\footnotesize Manager} & \textbf{\footnotesize Owner} &
\textbf{\footnotesize WBS} & \textbf{\footnotesize Team} \\ \hline
Margaret Johnson &
\begin{tabular}{@{}l@{}}
Michelle Butler \\
\end{tabular} & \begin{tabular}{@{}l@{}}
1.02C.07.08 \\
\end{tabular} & \begin{tabular}{@{}l@{}}
LDF \\
\end{tabular} \\ \hline
\multicolumn{4}{c}{
{\footnotesize ( Short Name: Campaign Mgmt - Acronym: CMPGN ) }
}\\ \hline
\end{longtable}

  This software product orchestrate the run of the science pipeline in a
docker. SW still to be written.



\begin{longtable}{p{3.7cm}p{3.7cm}p{3.7cm}p{3.7cm}}\hline
\textbf{\footnotesize Depends on:}  & & \textbf{\footnotesize Used in:} & \\ \hline
&
\\ \bottomrule
\end{longtable}

\begin{longtable}{p{3.7cm}p{3.7cm}p{3.7cm}p{3.7cm}}\toprule
\multicolumn{2}{l}{\large \textbf{ Workload/ Workflow Management } }\label{wlwf}
& \multicolumn{2}{l}{(product in: Batch Prod SW)}
\\ \hline
\textbf{\footnotesize Manager} & \textbf{\footnotesize Owner} &
\textbf{\footnotesize WBS} & \textbf{\footnotesize Team} \\ \hline
Margaret Johnson &
\begin{tabular}{@{}l@{}}
Michelle Butler \\
\end{tabular} & \begin{tabular}{@{}l@{}}
1.02C.07.08 \\
\end{tabular} & \begin{tabular}{@{}l@{}}
LDF \\
\end{tabular} \\ \hline
\multicolumn{4}{c}{
{\footnotesize ( Short Name: Workload/flow - Acronym: WLWF ) }
}\\ \hline
\end{longtable}

  This software product orchestrates the run of the science pipeline in a
docker. SW still to be written.



\begin{longtable}{p{3.7cm}p{3.7cm}p{3.7cm}p{3.7cm}}\hline
\textbf{\footnotesize Depends on:}  & & \textbf{\footnotesize Used in:} & \\ \hline
&
\multicolumn{2}{c}{
\begin{tabular}{c}
\hyperref[prodsrv]{Batch Production} \\ \hline
\end{tabular} }
\\ \bottomrule
\end{longtable}

\subsubsection{Backbone SW Products}\label{dbb}
\begin{longtable}{p{3.7cm}p{3.7cm}p{3.7cm}p{3.7cm}}\hline
\textbf{Manager} & \textbf{Owner} & \textbf{WBS} & \textbf{Team} \\ \hline
 &
\begin{tabular}{@{}l@{}}
Michelle Butler \\
\end{tabular} & \begin{tabular}{@{}l@{}}
 \\
\end{tabular} & \begin{tabular}{@{}l@{}}
 \\
\end{tabular} \\ \hline
\multicolumn{4}{c}{
{\footnotesize ( Short Name: DBB SW - Acronym: DBB ) }
}\\ \hline
\end{longtable}

  Software products that implement the Data Backbone services



\begin{longtable}{p{3.7cm}p{3.7cm}p{3.7cm}p{3.7cm}}\toprule
\multicolumn{2}{l}{\large \textbf{ DBB Lifetime Management SW } }\label{dbblife}
& \multicolumn{2}{l}{(product in: DBB SW)}
\\ \hline
\textbf{\footnotesize Manager} & \textbf{\footnotesize Owner} &
\textbf{\footnotesize WBS} & \textbf{\footnotesize Team} \\ \hline
Margaret Johnson &
\begin{tabular}{@{}l@{}}
Michelle Butler \\
\end{tabular} & \begin{tabular}{@{}l@{}}
1.02C.07.08 \\
\end{tabular} & \begin{tabular}{@{}l@{}}
LDF \\
\end{tabular} \\ \hline
\multicolumn{4}{c}{
{\footnotesize ( Short Name: DBB Lifetime SW - Acronym: DBBLIFE ) }
}\\ \hline
\end{longtable}

  This software product implements the management of the data in the
sciences storage.



\begin{longtable}{p{3.7cm}p{3.7cm}p{3.7cm}p{3.7cm}}\hline
\textbf{\footnotesize Depends on:}  & & \textbf{\footnotesize Used in:} & \\ \hline
&
\multicolumn{2}{c}{
\begin{tabular}{c}
\hyperref[dbblifesrv]{DBB Lifetime Management} \\ \hline
\end{tabular} }
\\ \bottomrule
\end{longtable}

\begin{longtable}{p{3.7cm}p{3.7cm}p{3.7cm}p{3.7cm}}\toprule
\multicolumn{2}{l}{\large \textbf{ DBB Ingest/ Metadata Management SW } }\label{dbbmd}
& \multicolumn{2}{l}{(product in: DBB SW)}
\\ \hline
\textbf{\footnotesize Manager} & \textbf{\footnotesize Owner} &
\textbf{\footnotesize WBS} & \textbf{\footnotesize Team} \\ \hline
Margaret Johnson &
\begin{tabular}{@{}l@{}}
Michelle Butler \\
\end{tabular} & \begin{tabular}{@{}l@{}}
1.02C.07.08 \\
\end{tabular} & \begin{tabular}{@{}l@{}}
LDF \\
\end{tabular} \\ \hline
\multicolumn{4}{c}{
{\footnotesize ( Short Name: DBB Meta SW - Acronym: DBBMD ) }
}\\ \hline
\end{longtable}

  Listener in the Endpoint Data Backbone Enclave to provide ingestion
services from the Facilities enclaves to the backbone.



\begin{longtable}{p{3.7cm}p{3.7cm}p{3.7cm}p{3.7cm}}\hline
\multicolumn{2}{r}{\textbf{GutHub Packages:}} &
\multicolumn{2}{l}{\href{https://github.com/lsst/dbb_gwclient}{dbb\_gwclient} }
\\ \cline{3-4}
& & \multicolumn{2}{l}{\href{https://github.com/lsst/dbb_gateway}{dbb\_gateway} }
\\ \hline \\ \hline
\textbf{\footnotesize Depends on:}  & & \textbf{\footnotesize Used in:} & \\ \hline
&
\multicolumn{2}{c}{
\begin{tabular}{c}
\hyperref[dbbmdsrv]{DBB Ingest/ Metadata Management} \\ \hline
\end{tabular} }
\\ \bottomrule
\end{longtable}

\begin{longtable}{p{3.7cm}p{3.7cm}p{3.7cm}p{3.7cm}}\toprule
\multicolumn{2}{l}{\large \textbf{ DBB Transport/ Replication/ Backup SW } }\label{dbbtr}
& \multicolumn{2}{l}{(product in: DBB SW)}
\\ \hline
\textbf{\footnotesize Manager} & \textbf{\footnotesize Owner} &
\textbf{\footnotesize WBS} & \textbf{\footnotesize Team} \\ \hline
Margaret Johnson &
\begin{tabular}{@{}l@{}}
Michelle Butler \\
\end{tabular} & \begin{tabular}{@{}l@{}}
1.02C.07.08 \\
\end{tabular} & \begin{tabular}{@{}l@{}}
LDF \\
\end{tabular} \\ \hline
\multicolumn{4}{c}{
{\footnotesize ( Short Name: DBB Transport SW - Acronym: DBBTR ) }
}\\ \hline
\end{longtable}

  This software products will be used in the services to replicate data
from different facilities and to send it to tape for long time
preservation.



\begin{longtable}{p{3.7cm}p{3.7cm}p{3.7cm}p{3.7cm}}\hline
\textbf{\footnotesize Depends on:}  & & \textbf{\footnotesize Used in:} & \\ \hline
&
\multicolumn{2}{c}{
\begin{tabular}{c}
\hyperref[dbbtrsrv]{DBB Transport/ Replication/ Backup} \\ \hline
\hyperref[dbbstrsrv]{DBB Storage} \\ \hline
\end{tabular} }
\\ \bottomrule
\end{longtable}

\subsubsection{LSP SW Products}\label{lsp}
\begin{longtable}{p{3.7cm}p{3.7cm}p{3.7cm}p{3.7cm}}\hline
\textbf{Manager} & \textbf{Owner} & \textbf{WBS} & \textbf{Team} \\ \hline
 &
\begin{tabular}{@{}l@{}}
Gregory Dubois-Felsmann \\
\end{tabular} & \begin{tabular}{@{}l@{}}
 \\
\end{tabular} & \begin{tabular}{@{}l@{}}
 \\
\end{tabular} \\ \hline
\multicolumn{4}{c}{
{\footnotesize ( Short Name: LSP SW - Acronym: LSP ) }
}\\ \hline
\end{longtable}

  Software products that implement the LSP services.



\begin{longtable}{p{3.7cm}p{3.7cm}p{3.7cm}p{3.7cm}}\toprule
\multicolumn{2}{l}{\large \textbf{ Image/ Cutout Server } }\label{daximg}
& \multicolumn{2}{l}{(product in: LSP SW)}
\\ \hline
\textbf{\footnotesize Manager} & \textbf{\footnotesize Owner} &
\textbf{\footnotesize WBS} & \textbf{\footnotesize Team} \\ \hline
Fritz Mueller &
\begin{tabular}{@{}l@{}}
Colin Slater \\
\end{tabular} & \begin{tabular}{@{}l@{}}
1.02C.06.02.04 \\
\end{tabular} & \begin{tabular}{@{}l@{}}
DAX \\
\end{tabular} \\ \hline
\multicolumn{4}{c}{
{\footnotesize ( Short Name: Image Server - Acronym: DAXIMG ) }
}\\ \hline
\end{longtable}

  Web Interface for LSST Image Services software product.



\begin{longtable}{p{3.7cm}p{3.7cm}p{3.7cm}p{3.7cm}}\hline
\multicolumn{2}{r}{\textbf{GutHub Packages:}} &
\multicolumn{2}{l}{\href{https://github.com/lsst/dax_imgserv}{dax\_imgserv} }
\\ \hline \\ \hline
\textbf{\footnotesize Depends on:}  & & \textbf{\footnotesize Used in:} & \\ \hline
&
\multicolumn{2}{c}{
\begin{tabular}{c}
\hyperref[soda]{SODA} \\ \hline
\end{tabular} }
\\ \bottomrule
\end{longtable}

\begin{longtable}{p{3.7cm}p{3.7cm}p{3.7cm}p{3.7cm}}\toprule
\multicolumn{2}{l}{\large \textbf{ LSP JupyterLab SW } }\label{lspjl}
& \multicolumn{2}{l}{(product in: LSP SW)}
\\ \hline
\textbf{\footnotesize Manager} & \textbf{\footnotesize Owner} &
\textbf{\footnotesize WBS} & \textbf{\footnotesize Team} \\ \hline
Frossie Economou &
\begin{tabular}{@{}l@{}}
Simon Krughoff \\
\end{tabular} & \begin{tabular}{@{}l@{}}
1.02C.10.02.02 \\
\end{tabular} & \begin{tabular}{@{}l@{}}
SQuaRE \\
\end{tabular} \\ \hline
\multicolumn{4}{c}{
{\footnotesize ( Short Name: LSP JL SW - Acronym: LSPJL ) }
}\\ \hline
\end{longtable}

  LSP Jupiter Notebook Product



\begin{longtable}{p{3.7cm}p{3.7cm}p{3.7cm}p{3.7cm}}\hline
\multicolumn{2}{r}{\textbf{GutHub Packages:}} &
\multicolumn{2}{l}{\href{https://github.com/lsst/jupyterlab*}{jupyterlab*} }
\\ \hline \\ \hline
\textbf{\footnotesize Depends on:}  & & \textbf{\footnotesize Used in:} & \\ \hline
&
\multicolumn{2}{c}{
\begin{tabular}{c}
\hyperref[lspjlsrv]{LSP JupyterLab} \\ \hline
\end{tabular} }
\\ \bottomrule
\end{longtable}

\begin{longtable}{p{3.7cm}p{3.7cm}p{3.7cm}p{3.7cm}}\toprule
\multicolumn{2}{l}{\large \textbf{ LSP Web API SW } }\label{lspweb}
& \multicolumn{2}{l}{(product in: LSP SW)}
\\ \hline
\textbf{\footnotesize Manager} & \textbf{\footnotesize Owner} &
\textbf{\footnotesize WBS} & \textbf{\footnotesize Team} \\ \hline
Fritz Mueller &
\begin{tabular}{@{}l@{}}
Colin Slater \\
\end{tabular} & \begin{tabular}{@{}l@{}}
1.02C.06.02 \\
\end{tabular} & \begin{tabular}{@{}l@{}}
DAX \\
\end{tabular} \\ \hline
\multicolumn{4}{c}{
{\footnotesize ( Short Name: LSP Web SW - Acronym: LSPWEB ) }
}\\ \hline
\end{longtable}

  {[}OBSOLETE{]}

LSP Web API software product



\begin{longtable}{p{3.7cm}p{3.7cm}p{3.7cm}p{3.7cm}}\hline
\multicolumn{2}{r}{\textbf{GutHub Packages:}} &
\multicolumn{2}{l}{\href{https://github.com/lsst/dax_webserv}{dax\_webserv} }
\\ \hline \\ \hline
\textbf{\footnotesize Depends on:}  & & \textbf{\footnotesize Used in:} & \\ \hline
&
\\ \bottomrule
\end{longtable}

\begin{longtable}{p{3.7cm}p{3.7cm}p{3.7cm}p{3.7cm}}\toprule
\multicolumn{2}{l}{\large \textbf{ SUIT } }\label{suit}
& \multicolumn{2}{l}{(product in: LSP SW)}
\\ \hline
\textbf{\footnotesize Manager} & \textbf{\footnotesize Owner} &
\textbf{\footnotesize WBS} & \textbf{\footnotesize Team} \\ \hline
Xiuqin Wu &
\begin{tabular}{@{}l@{}}
Gregory Dubois-Felsmann \\
\end{tabular} & \begin{tabular}{@{}l@{}}
1.02C.05.09 \\
1.02C.05.07 \\
1.02C.05.08 \\
\end{tabular} & \begin{tabular}{@{}l@{}}
SUIT \\
\end{tabular} \\ \hline
\multicolumn{4}{c}{
{\footnotesize ( Short Name: SUIT - Acronym: SUIT ) }
}\\ \hline
\end{longtable}

  Implements LSST Portal Aspect-specific behaviors added to the core
Firefly library. Defines the Portal web application. Primarily
JavaScript but also contains Java extensions (search processors) to the
Firefly server side.



\begin{longtable}{p{3.7cm}p{3.7cm}p{3.7cm}p{3.7cm}}\hline
\multicolumn{2}{r}{\textbf{GutHub Packages:}} &
\multicolumn{2}{l}{\href{https://github.com/lsst/suit}{suit} }
\\ \hline \\ \hline
\textbf{\footnotesize Depends on:}  & & \textbf{\footnotesize Used in:} & \\ \hline
\multicolumn{2}{c}{
\begin{tabular}{c}
\hyperref[firefly]{Firefly} \\ \hline
\end{tabular} }
&
\multicolumn{2}{c}{
\begin{tabular}{c}
\hyperref[lspprtlsrv]{LSP Portal} \\ \hline
\end{tabular} }
\\ \bottomrule
\end{longtable}

\begin{longtable}{p{3.7cm}p{3.7cm}p{3.7cm}p{3.7cm}}\toprule
\multicolumn{2}{l}{\large \textbf{ SUIT Online Help } }\label{suitoh}
& \multicolumn{2}{l}{(product in: LSP SW)}
\\ \hline
\textbf{\footnotesize Manager} & \textbf{\footnotesize Owner} &
\textbf{\footnotesize WBS} & \textbf{\footnotesize Team} \\ \hline
Xiuqin Wu &
\begin{tabular}{@{}l@{}}
Gregory Dubois-Felsmann \\
\end{tabular} & \begin{tabular}{@{}l@{}}
 \\
\end{tabular} & \begin{tabular}{@{}l@{}}
SUIT \\
\end{tabular} \\ \hline
\multicolumn{4}{c}{
{\footnotesize ( Short Name: SUIT OnlineHelp - Acronym: SUITOH ) }
}\\ \hline
\end{longtable}

  Contains help content, as HTML, and a small amount of GWT GUI code to
implement a help application that provides a navigation sidebar for the
HTML content.



\begin{longtable}{p{3.7cm}p{3.7cm}p{3.7cm}p{3.7cm}}\hline
\multicolumn{2}{r}{\textbf{GutHub Packages:}} &
\multicolumn{2}{l}{\href{https://github.com/lsst/suit-onlinehelp}{suit-onlinehelp} }
\\ \hline \\ \hline
\textbf{\footnotesize Depends on:}  & & \textbf{\footnotesize Used in:} & \\ \hline
&
\multicolumn{2}{c}{
\begin{tabular}{c}
\hyperref[lspprtlsrv]{LSP Portal} \\ \hline
\end{tabular} }
\\ \bottomrule
\end{longtable}

\begin{longtable}{p{3.7cm}p{3.7cm}p{3.7cm}p{3.7cm}}\toprule
\multicolumn{2}{l}{\large \textbf{ TAP SW } }\label{tapsw}
& \multicolumn{2}{l}{(product in: LSP SW)}
\\ \hline
\textbf{\footnotesize Manager} & \textbf{\footnotesize Owner} &
\textbf{\footnotesize WBS} & \textbf{\footnotesize Team} \\ \hline
 &
\begin{tabular}{@{}l@{}}
 \\
\end{tabular} & \begin{tabular}{@{}l@{}}
 \\
\end{tabular} & \begin{tabular}{@{}l@{}}
 \\
\end{tabular} \\ \hline
\multicolumn{4}{c}{
{\footnotesize ( Short Name: TAP - Acronym: TAPSW ) }
}\\ \hline
\end{longtable}

  



\begin{longtable}{p{3.7cm}p{3.7cm}p{3.7cm}p{3.7cm}}\hline
\textbf{\footnotesize Depends on:}  & & \textbf{\footnotesize Used in:} & \\ \hline
&
\multicolumn{2}{c}{
\begin{tabular}{c}
\hyperref[tapsev]{TAP} \\ \hline
\end{tabular} }
\\ \bottomrule
\end{longtable}

\subsubsection{Prompt SW Products}\label{pr}
\begin{longtable}{p{3.7cm}p{3.7cm}p{3.7cm}p{3.7cm}}\hline
\textbf{Manager} & \textbf{Owner} & \textbf{WBS} & \textbf{Team} \\ \hline
 &
\begin{tabular}{@{}l@{}}
Multiple \\
\end{tabular} & \begin{tabular}{@{}l@{}}
 \\
\end{tabular} & \begin{tabular}{@{}l@{}}
 \\
\end{tabular} \\ \hline
\multicolumn{4}{c}{
{\footnotesize ( Short Name: Prompt SW - Acronym: PR ) }
}\\ \hline
\end{longtable}

  DM software products required to implement the DM Prompt Services.



\begin{longtable}{p{3.7cm}p{3.7cm}p{3.7cm}p{3.7cm}}\toprule
\multicolumn{2}{l}{\large \textbf{ Alert Distribution SW } }\label{alrtdstr}
& \multicolumn{2}{l}{(product in: Prompt SW)}
\\ \hline
\textbf{\footnotesize Manager} & \textbf{\footnotesize Owner} &
\textbf{\footnotesize WBS} & \textbf{\footnotesize Team} \\ \hline
John Swinbank &
\begin{tabular}{@{}l@{}}
Eric Bellm \\
\end{tabular} & \begin{tabular}{@{}l@{}}
1.02C.03.03 \\
\end{tabular} & \begin{tabular}{@{}l@{}}
AP \\
\end{tabular} \\ \hline
\multicolumn{4}{c}{
{\footnotesize ( Short Name: Alert Distrib SW - Acronym: ALRTDSTR ) }
}\\ \hline
\end{longtable}

  SW product for alert distribution and filtering.



\begin{longtable}{p{3.7cm}p{3.7cm}p{3.7cm}p{3.7cm}}\hline
\multicolumn{2}{r}{\textbf{GutHub Packages:}} &
\multicolumn{2}{l}{\href{https://github.com/lsst/alert_stream}{alert\_stream} }
\\ \hline \\ \hline
\textbf{\footnotesize Depends on:}  & & \textbf{\footnotesize Used in:} & \\ \hline
&
\multicolumn{2}{c}{
\begin{tabular}{c}
\hyperref[alrtdstsrv]{Alert Distribution} \\ \hline
\end{tabular} }
\\ \bottomrule
\end{longtable}

\begin{longtable}{p{3.7cm}p{3.7cm}p{3.7cm}p{3.7cm}}\toprule
\multicolumn{2}{l}{\large \textbf{ EFD Transformation } }\label{efdt}
& \multicolumn{2}{l}{(product in: Prompt SW)}
\\ \hline
\textbf{\footnotesize Manager} & \textbf{\footnotesize Owner} &
\textbf{\footnotesize WBS} & \textbf{\footnotesize Team} \\ \hline
Margaret Johnson &
\begin{tabular}{@{}l@{}}
Simon Krughoff \\
\end{tabular} & \begin{tabular}{@{}l@{}}
1.02C.07.08 \\
\end{tabular} & \begin{tabular}{@{}l@{}}
LDF \\
\end{tabular} \\ \hline
\multicolumn{4}{c}{
{\footnotesize ( Short Name: EFD Transform - Acronym: EFDT ) }
}\\ \hline
\end{longtable}

  This SW is the one used for Image and EFD Archiving in the Archive
services. (EFD Extract-Tranform-Load) Software still to be implemented.



\begin{longtable}{p{3.7cm}p{3.7cm}p{3.7cm}p{3.7cm}}\hline
\textbf{\footnotesize Depends on:}  & & \textbf{\footnotesize Used in:} & \\ \hline
&
\multicolumn{2}{c}{
\begin{tabular}{c}
\hyperref[arcsrv]{Archiving} \\ \hline
\hyperref[efdb]{EFD Cache} \\ \hline
\end{tabular} }
\\ \bottomrule
\end{longtable}

\begin{longtable}{p{3.7cm}p{3.7cm}p{3.7cm}p{3.7cm}}\toprule
\multicolumn{2}{l}{\large \textbf{ Header Service SW } }\label{header}
& \multicolumn{2}{l}{(product in: Prompt SW)}
\\ \hline
\textbf{\footnotesize Manager} & \textbf{\footnotesize Owner} &
\textbf{\footnotesize WBS} & \textbf{\footnotesize Team} \\ \hline
Margaret Johnson &
\begin{tabular}{@{}l@{}}
Felipe Menanteau \\
\end{tabular} & \begin{tabular}{@{}l@{}}
1.02C.07.08 \\
\end{tabular} & \begin{tabular}{@{}l@{}}
LDF \\
\end{tabular} \\ \hline
\multicolumn{4}{c}{
{\footnotesize ( Short Name: Header Srv SW - Acronym: HEADER ) }
}\\ \hline
\end{longtable}

  This software product is developed by DM but is inteded to be used by
external users. It is developed at University of Illinois (Urbana)



\begin{longtable}{p{3.7cm}p{3.7cm}p{3.7cm}p{3.7cm}}\hline
\multicolumn{2}{r}{\textbf{GutHub Packages:}} &
\multicolumn{2}{l}{\href{https://github.com/lsst-dm/HeaderService}{lsst-dm/HeaderService} }
\\ \hline \\ \hline
\textbf{\footnotesize Depends on:}  & & \textbf{\footnotesize Used in:} & \\ \hline
&
\multicolumn{2}{c}{
\begin{tabular}{c}
\hyperref[arcsrv]{Archiving} \\ \hline
\end{tabular} }
\\ \bottomrule
\end{longtable}

\begin{longtable}{p{3.7cm}p{3.7cm}p{3.7cm}p{3.7cm}}\toprule
\multicolumn{2}{l}{\large \textbf{ Image Ingest and Processing } }\label{iip}
& \multicolumn{2}{l}{(product in: Prompt SW)}
\\ \hline
\textbf{\footnotesize Manager} & \textbf{\footnotesize Owner} &
\textbf{\footnotesize WBS} & \textbf{\footnotesize Team} \\ \hline
Margaret Johnson &
\begin{tabular}{@{}l@{}}
Felipe Menanteau \\
\end{tabular} & \begin{tabular}{@{}l@{}}
1.02C.07.08 \\
\end{tabular} & \begin{tabular}{@{}l@{}}
LDF \\
\end{tabular} \\ \hline
\multicolumn{4}{c}{
{\footnotesize ( Short Name: Image Ingest SW - Acronym: IIP ) }
}\\ \hline
\end{longtable}

  Image ingest and processing SW product.\\[2\baselineskip]This includes
the DMCS, used to implements the Telemetry Gateway service.



\begin{longtable}{p{3.7cm}p{3.7cm}p{3.7cm}p{3.7cm}}\hline
\multicolumn{2}{r}{\textbf{GutHub Packages:}} &
\multicolumn{2}{l}{\href{https://github.com/lsst/ctrl_iip}{ctrl\_iip} }
\\ \hline \\ \hline
\textbf{\footnotesize Depends on:}  & & \textbf{\footnotesize Used in:} & \\ \hline
&
\multicolumn{2}{c}{
\begin{tabular}{c}
\hyperref[prpingsrv]{Prompt Processing Ingest} \\ \hline
\hyperref[arcsrv]{Archiving} \\ \hline
\hyperref[tmgsrv]{Telemetry Gateway} \\ \hline
\end{tabular} }
\\ \bottomrule
\end{longtable}

\begin{longtable}{p{3.7cm}p{3.7cm}p{3.7cm}p{3.7cm}}\toprule
\multicolumn{2}{l}{\large \textbf{ Planned Observation Publication SW } }\label{obspub}
& \multicolumn{2}{l}{(product in: Prompt SW)}
\\ \hline
\textbf{\footnotesize Manager} & \textbf{\footnotesize Owner} &
\textbf{\footnotesize WBS} & \textbf{\footnotesize Team} \\ \hline
Margaret Johnson &
\begin{tabular}{@{}l@{}}
Felipe Menanteau \\
\end{tabular} & \begin{tabular}{@{}l@{}}
1.02C.07.08 \\
\end{tabular} & \begin{tabular}{@{}l@{}}
LDF \\
\end{tabular} \\ \hline
\multicolumn{4}{c}{
{\footnotesize ( Short Name: Plan Obs Pub SW - Acronym: OBSPUB ) }
}\\ \hline
\end{longtable}

  Pointing Prediction Publishing software products.



\begin{longtable}{p{3.7cm}p{3.7cm}p{3.7cm}p{3.7cm}}\hline
\textbf{\footnotesize Depends on:}  & & \textbf{\footnotesize Used in:} & \\ \hline
&
\multicolumn{2}{c}{
\begin{tabular}{c}
\hyperref[popsrv]{Planned Observation Publication} \\ \hline
\end{tabular} }
\\ \bottomrule
\end{longtable}

\begin{longtable}{p{3.7cm}p{3.7cm}p{3.7cm}p{3.7cm}}\toprule
\multicolumn{2}{l}{\large \textbf{ OCS Batch SW } }\label{ocsbat}
& \multicolumn{2}{l}{(product in: Prompt SW)}
\\ \hline
\textbf{\footnotesize Manager} & \textbf{\footnotesize Owner} &
\textbf{\footnotesize WBS} & \textbf{\footnotesize Team} \\ \hline
Margaret Johnson &
\begin{tabular}{@{}l@{}}
Felipe Menanteau \\
\end{tabular} & \begin{tabular}{@{}l@{}}
1.02C.07.08 \\
\end{tabular} & \begin{tabular}{@{}l@{}}
LDF \\
\end{tabular} \\ \hline
\multicolumn{4}{c}{
{\footnotesize ( Short Name: OCS Batch SW - Acronym: OCSBAT ) }
}\\ \hline
\end{longtable}

  Batch commandable SAL component.



\begin{longtable}{p{3.7cm}p{3.7cm}p{3.7cm}p{3.7cm}}\hline
\textbf{\footnotesize Depends on:}  & & \textbf{\footnotesize Used in:} & \\ \hline
&
\multicolumn{2}{c}{
\begin{tabular}{c}
\hyperref[ocsbatsrv]{OCS-Driven Batch} \\ \hline
\end{tabular} }
\\ \bottomrule
\end{longtable}

\begin{longtable}{p{3.7cm}p{3.7cm}p{3.7cm}p{3.7cm}}\toprule
\multicolumn{2}{l}{\large \textbf{ Observatory Operations Data Service SW } }\label{oods}
& \multicolumn{2}{l}{(product in: Prompt SW)}
\\ \hline
\textbf{\footnotesize Manager} & \textbf{\footnotesize Owner} &
\textbf{\footnotesize WBS} & \textbf{\footnotesize Team} \\ \hline
Margaret Johnson &
\begin{tabular}{@{}l@{}}
Michelle Butler \\
\end{tabular} & \begin{tabular}{@{}l@{}}
1.02C.07.08 \\
\end{tabular} & \begin{tabular}{@{}l@{}}
LDF \\
\end{tabular} \\ \hline
\multicolumn{4}{c}{
{\footnotesize ( Short Name: Obs Ops Data SW - Acronym: OODS ) }
}\\ \hline
\end{longtable}

  SW Product to use in the OODS service.



\begin{longtable}{p{3.7cm}p{3.7cm}p{3.7cm}p{3.7cm}}\hline
\multicolumn{2}{r}{\textbf{GutHub Packages:}} &
\multicolumn{2}{l}{\href{https://github.com/lsst-dm/ctrl_oods}{lsst-dm/ctrl\_oods} }
\\ \hline \\ \hline
\textbf{\footnotesize Depends on:}  & & \textbf{\footnotesize Used in:} & \\ \hline
&
\multicolumn{2}{c}{
\begin{tabular}{c}
\hyperref[oodssrv]{Observatory Operations Data} \\ \hline
\end{tabular} }
\\ \bottomrule
\end{longtable}

\subsubsection{Science Pipeline SW Products}\label{prodn}
\begin{longtable}{p{3.7cm}p{3.7cm}p{3.7cm}p{3.7cm}}\hline
\textbf{Manager} & \textbf{Owner} & \textbf{WBS} & \textbf{Team} \\ \hline
 &
\begin{tabular}{@{}l@{}}
Robert Lupton \\
\end{tabular} & \begin{tabular}{@{}l@{}}
 \\
\end{tabular} & \begin{tabular}{@{}l@{}}
 \\
\end{tabular} \\ \hline
\multicolumn{4}{c}{
{\footnotesize ( Short Name: Sci Pipelines SW - Acronym: PRODN ) }
}\\ \hline
\end{longtable}

  DM Software products that implements the Science Pipelines payloads.



\begin{longtable}{p{3.7cm}p{3.7cm}p{3.7cm}p{3.7cm}}\toprule
\multicolumn{2}{l}{\large \textbf{ Alert Production } }\label{apprmpt}
& \multicolumn{2}{l}{(product in: Sci Pipelines SW)}
\\ \hline
\textbf{\footnotesize Manager} & \textbf{\footnotesize Owner} &
\textbf{\footnotesize WBS} & \textbf{\footnotesize Team} \\ \hline
John Swinbank &
\begin{tabular}{@{}l@{}}
Eric Bellm \\
\end{tabular} & \begin{tabular}{@{}l@{}}
1.02C.03 \\
\end{tabular} & \begin{tabular}{@{}l@{}}
AP \\
\end{tabular} \\ \hline
\multicolumn{4}{c}{
{\footnotesize ( Short Name: Alert Prod SW - Acronym: APPRMPT ) }
}\\ \hline
\end{longtable}

  Software product for Allert Production processing.



\begin{longtable}{p{3.7cm}p{3.7cm}p{3.7cm}p{3.7cm}}\hline
\multicolumn{2}{r}{\textbf{GutHub Packages:}} &
\multicolumn{2}{l}{\href{https://github.com/lsst/spl_ap}{spl\_ap} }
\\ \hline \\ \hline
\textbf{\footnotesize Depends on:}  & & \textbf{\footnotesize Used in:} & \\ \hline
\multicolumn{2}{c}{
\begin{tabular}{c}
\hyperref[scipipe]{Science Pipelines Libraries} \\ \hline
\end{tabular} }
&
\multicolumn{2}{c}{
\begin{tabular}{c}
\hyperref[prprsrv]{Prompt Processing} \\ \hline
\hyperref[spdist]{Science Pipelines Distribution} \\ \hline
\end{tabular} }
\\ \bottomrule
\end{longtable}

\begin{longtable}{p{3.7cm}p{3.7cm}p{3.7cm}p{3.7cm}}\toprule
\multicolumn{2}{l}{\large \textbf{ Calibration SW } }\label{dmcal}
& \multicolumn{2}{l}{(product in: Sci Pipelines SW)}
\\ \hline
\textbf{\footnotesize Manager} & \textbf{\footnotesize Owner} &
\textbf{\footnotesize WBS} & \textbf{\footnotesize Team} \\ \hline
John Swinbank &
\begin{tabular}{@{}l@{}}
Robert Lupton \\
\end{tabular} & \begin{tabular}{@{}l@{}}
1.02C.04.02 \\
\end{tabular} & \begin{tabular}{@{}l@{}}
DRP \\
\end{tabular} \\ \hline
\multicolumn{4}{c}{
{\footnotesize ( Short Name: Calibration SW - Acronym: DMCAL ) }
}\\ \hline
\end{longtable}

  Software product for generating calibration data products.



\begin{longtable}{p{3.7cm}p{3.7cm}p{3.7cm}p{3.7cm}}\hline
\multicolumn{2}{r}{\textbf{GutHub Packages:}} &
\multicolumn{2}{l}{\href{https://github.com/lsst/spl_calibration}{spl\_calibration} }
\\ \hline \\ \hline
\textbf{\footnotesize Depends on:}  & & \textbf{\footnotesize Used in:} & \\ \hline
\multicolumn{2}{c}{
\begin{tabular}{c}
\hyperref[scipipe]{Science Pipelines Libraries} \\ \hline
\end{tabular} }
&
\multicolumn{2}{c}{
\begin{tabular}{c}
\hyperref[prprsrv]{Prompt Processing} \\ \hline
\hyperref[prodsrv]{Batch Production} \\ \hline
\hyperref[spdist]{Science Pipelines Distribution} \\ \hline
\end{tabular} }
\\ \bottomrule
\end{longtable}

\begin{longtable}{p{3.7cm}p{3.7cm}p{3.7cm}p{3.7cm}}\toprule
\multicolumn{2}{l}{\large \textbf{ Data Release Production } }\label{drp}
& \multicolumn{2}{l}{(product in: Sci Pipelines SW)}
\\ \hline
\textbf{\footnotesize Manager} & \textbf{\footnotesize Owner} &
\textbf{\footnotesize WBS} & \textbf{\footnotesize Team} \\ \hline
Yusra AlSayyad &
\begin{tabular}{@{}l@{}}
Jim Bosch \\
\end{tabular} & \begin{tabular}{@{}l@{}}
1.02C.04 \\
\end{tabular} & \begin{tabular}{@{}l@{}}
DRP \\
\end{tabular} \\ \hline
\multicolumn{4}{c}{
{\footnotesize ( Short Name: DR Prod SW - Acronym: DRP ) }
}\\ \hline
\end{longtable}

  Software product for data release production.



\begin{longtable}{p{3.7cm}p{3.7cm}p{3.7cm}p{3.7cm}}\hline
\multicolumn{2}{r}{\textbf{GutHub Packages:}} &
\multicolumn{2}{l}{\href{https://github.com/lsst/spl_drp}{spl\_drp} }
\\ \hline \\ \hline
\textbf{\footnotesize Depends on:}  & & \textbf{\footnotesize Used in:} & \\ \hline
\multicolumn{2}{c}{
\begin{tabular}{c}
\hyperref[scipipe]{Science Pipelines Libraries} \\ \hline
\end{tabular} }
&
\multicolumn{2}{c}{
\begin{tabular}{c}
\hyperref[prodsrv]{Batch Production} \\ \hline
\hyperref[spdist]{Science Pipelines Distribution} \\ \hline
\end{tabular} }
\\ \bottomrule
\end{longtable}

\begin{longtable}{p{3.7cm}p{3.7cm}p{3.7cm}p{3.7cm}}\toprule
\multicolumn{2}{l}{\large \textbf{ MOPS and Forced Photometry } }\label{mops}
& \multicolumn{2}{l}{(product in: Sci Pipelines SW)}
\\ \hline
\textbf{\footnotesize Manager} & \textbf{\footnotesize Owner} &
\textbf{\footnotesize WBS} & \textbf{\footnotesize Team} \\ \hline
John Swinbank &
\begin{tabular}{@{}l@{}}
Eric Bellm \\
\end{tabular} & \begin{tabular}{@{}l@{}}
1.02C.03.06 \\
\end{tabular} & \begin{tabular}{@{}l@{}}
AP \\
\end{tabular} \\ \hline
\multicolumn{4}{c}{
{\footnotesize ( Short Name: MOPS SW - Acronym: MOPS ) }
}\\ \hline
\end{longtable}

  Software product for MOPS and Forced Photometry data processing.



\begin{longtable}{p{3.7cm}p{3.7cm}p{3.7cm}p{3.7cm}}\hline
\multicolumn{2}{r}{\textbf{GutHub Packages:}} &
\multicolumn{2}{l}{\href{https://github.com/lsst/mops_daymops}{mops\_daymops} }
\\ \hline \\ \hline
\textbf{\footnotesize Depends on:}  & & \textbf{\footnotesize Used in:} & \\ \hline
\multicolumn{2}{c}{
\begin{tabular}{c}
\hyperref[scipipe]{Science Pipelines Libraries} \\ \hline
\end{tabular} }
&
\multicolumn{2}{c}{
\begin{tabular}{c}
\hyperref[prprsrv]{Prompt Processing} \\ \hline
\hyperref[prodsrv]{Batch Production} \\ \hline
\hyperref[spdist]{Science Pipelines Distribution} \\ \hline
\end{tabular} }
\\ \bottomrule
\end{longtable}

\begin{longtable}{p{3.7cm}p{3.7cm}p{3.7cm}p{3.7cm}}\toprule
\multicolumn{2}{l}{\large \textbf{ Special Programs Productions } }\label{sp}
& \multicolumn{2}{l}{(product in: Sci Pipelines SW)}
\\ \hline
\textbf{\footnotesize Manager} & \textbf{\footnotesize Owner} &
\textbf{\footnotesize WBS} & \textbf{\footnotesize Team} \\ \hline
John Swinbank &
\begin{tabular}{@{}l@{}}
Melissa Graham \\
\end{tabular} & \begin{tabular}{@{}l@{}}
1.02C.03 \\
1.02C.04 \\
\end{tabular} & \begin{tabular}{@{}l@{}}
AP \\
DRP \\
\end{tabular} \\ \hline
\multicolumn{4}{c}{
{\footnotesize ( Short Name: Spec Prog SW - Acronym: SP ) }
}\\ \hline
\end{longtable}

  Software product for special programs data processing.



\begin{longtable}{p{3.7cm}p{3.7cm}p{3.7cm}p{3.7cm}}\hline
\textbf{\footnotesize Depends on:}  & & \textbf{\footnotesize Used in:} & \\ \hline
\multicolumn{2}{c}{
\begin{tabular}{c}
\hyperref[scipipe]{Science Pipelines Libraries} \\ \hline
\end{tabular} }
&
\multicolumn{2}{c}{
\begin{tabular}{c}
\hyperref[prodsrv]{Batch Production} \\ \hline
\hyperref[spdist]{Science Pipelines Distribution} \\ \hline
\end{tabular} }
\\ \bottomrule
\end{longtable}

\begin{longtable}{p{3.7cm}p{3.7cm}p{3.7cm}p{3.7cm}}\toprule
\multicolumn{2}{l}{\large \textbf{ Science Pipelines Distribution } }\label{spdist}
& \multicolumn{2}{l}{(product in: Sci Pipelines SW)}
\\ \hline
\textbf{\footnotesize Manager} & \textbf{\footnotesize Owner} &
\textbf{\footnotesize WBS} & \textbf{\footnotesize Team} \\ \hline
John Swinbank &
\begin{tabular}{@{}l@{}}
Robert Lupton \\
\end{tabular} & \begin{tabular}{@{}l@{}}
1.02C.04.02 \\
\end{tabular} & \begin{tabular}{@{}l@{}}
DRP \\
\end{tabular} \\ \hline
\multicolumn{4}{c}{
{\footnotesize ( Short Name: Science P. Dist. - Acronym: SPDIST ) }
}\\ \hline
\end{longtable}

  Top level distribution product for the science pipelines



\begin{longtable}{p{3.7cm}p{3.7cm}p{3.7cm}p{3.7cm}}\hline
\multicolumn{2}{r}{\textbf{GutHub Packages:}} &
\multicolumn{2}{l}{\href{https://github.com/lsst/lsst_distrib}{lsst\_distrib} }
\\ \hline \\ \hline
\textbf{\footnotesize Depends on:}  & & \textbf{\footnotesize Used in:} & \\ \hline
\multicolumn{2}{c}{
\begin{tabular}{c}
\hyperref[apprmpt]{Alert Production} \\ \hline
\hyperref[dmcal]{Calibration SW} \\ \hline
\hyperref[drp]{Data Release Production} \\ \hline
\hyperref[mops]{MOPS and Forced Photometry} \\ \hline
\hyperref[sp]{Special Programs Productions} \\ \hline
\hyperref[splug]{Science Plugins} \\ \hline
\hyperref[tmplgen]{Template Generation} \\ \hline
\hyperref[butler]{Data Butler} \\ \hline
\hyperref[scipipe]{Science Pipelines Libraries} \\ \hline
\hyperref[txf]{Task Framework} \\ \hline
\hyperref[and]{Atronomy.net Data} \\ \hline
\end{tabular} }
&
\multicolumn{2}{c}{
\begin{tabular}{c}
\hyperref[lspjlsrv]{LSP JupyterLab} \\ \hline
\end{tabular} }
\\ \bottomrule
\end{longtable}

\begin{longtable}{p{3.7cm}p{3.7cm}p{3.7cm}p{3.7cm}}\toprule
\multicolumn{2}{l}{\large \textbf{ Science Plugins } }\label{splug}
& \multicolumn{2}{l}{(product in: Sci Pipelines SW)}
\\ \hline
\textbf{\footnotesize Manager} & \textbf{\footnotesize Owner} &
\textbf{\footnotesize WBS} & \textbf{\footnotesize Team} \\ \hline
John Swinbank &
\begin{tabular}{@{}l@{}}
Robert Lupton \\
\end{tabular} & \begin{tabular}{@{}l@{}}
1.02C.04.02 \\
\end{tabular} & \begin{tabular}{@{}l@{}}
DRP \\
\end{tabular} \\ \hline
\multicolumn{4}{c}{
{\footnotesize ( Short Name: Science Plugins - Acronym: SPLUG ) }
}\\ \hline
\end{longtable}

  This software product includes all Science Pipelines plugin packages. A
top level meta-package need to be defined, and all corresponding plugins
shall to be a dependency in it. Definition: a plugin is any piece of
software that meets a well-defined interface standard and can be
configured in or out of the processing by the user.



\begin{longtable}{p{3.7cm}p{3.7cm}p{3.7cm}p{3.7cm}}\hline
\multicolumn{2}{r}{\textbf{GutHub Packages:}} &
\multicolumn{2}{l}{\href{https://github.com/lsst/spl_plugins}{spl\_plugins} }
\\ \hline \\ \hline
\textbf{\footnotesize Depends on:}  & & \textbf{\footnotesize Used in:} & \\ \hline
&
\multicolumn{2}{c}{
\begin{tabular}{c}
\hyperref[spdist]{Science Pipelines Distribution} \\ \hline
\end{tabular} }
\\ \bottomrule
\end{longtable}

\begin{longtable}{p{3.7cm}p{3.7cm}p{3.7cm}p{3.7cm}}\toprule
\multicolumn{2}{l}{\large \textbf{ Template Generation } }\label{tmplgen}
& \multicolumn{2}{l}{(product in: Sci Pipelines SW)}
\\ \hline
\textbf{\footnotesize Manager} & \textbf{\footnotesize Owner} &
\textbf{\footnotesize WBS} & \textbf{\footnotesize Team} \\ \hline
Yusra AlSayyad &
\begin{tabular}{@{}l@{}}
Jim Bosch \\
\end{tabular} & \begin{tabular}{@{}l@{}}
1.02C.04.04 \\
\end{tabular} & \begin{tabular}{@{}l@{}}
DRP \\
\end{tabular} \\ \hline
\multicolumn{4}{c}{
{\footnotesize ( Short Name: Tmpl Gen SW - Acronym: TMPLGEN ) }
}\\ \hline
\end{longtable}

  



\begin{longtable}{p{3.7cm}p{3.7cm}p{3.7cm}p{3.7cm}}\hline
\textbf{\footnotesize Depends on:}  & & \textbf{\footnotesize Used in:} & \\ \hline
&
\multicolumn{2}{c}{
\begin{tabular}{c}
\hyperref[spdist]{Science Pipelines Distribution} \\ \hline
\end{tabular} }
\\ \bottomrule
\end{longtable}

\subsubsection{Quality Control Products}\label{qc}
\begin{longtable}{p{3.7cm}p{3.7cm}p{3.7cm}p{3.7cm}}\hline
\textbf{Manager} & \textbf{Owner} & \textbf{WBS} & \textbf{Team} \\ \hline
 &
\begin{tabular}{@{}l@{}}
Simon Krughoff \\
\end{tabular} & \begin{tabular}{@{}l@{}}
 \\
\end{tabular} & \begin{tabular}{@{}l@{}}
 \\
\end{tabular} \\ \hline
\multicolumn{4}{c}{
{\footnotesize ( Short Name: QC Products - Acronym: QC ) }
}\\ \hline
\end{longtable}

  SW products for quality control.



\begin{longtable}{p{3.7cm}p{3.7cm}p{3.7cm}p{3.7cm}}\toprule
\multicolumn{2}{l}{\large \textbf{ Quality Control SW } }\label{qcsw}
& \multicolumn{2}{l}{(product in: QC Products)}
\\ \hline
\textbf{\footnotesize Manager} & \textbf{\footnotesize Owner} &
\textbf{\footnotesize WBS} & \textbf{\footnotesize Team} \\ \hline
Frossie Economou &
\begin{tabular}{@{}l@{}}
Simon Krughoff \\
\end{tabular} & \begin{tabular}{@{}l@{}}
1.02C.10.02.01 \\
\end{tabular} & \begin{tabular}{@{}l@{}}
SQuaRE \\
\end{tabular} \\ \hline
\multicolumn{4}{c}{
{\footnotesize ( Short Name: Quality Ctrl SW - Acronym: QCSW ) }
}\\ \hline
\end{longtable}

  This software product is used to instantiate the quality control
services.



\begin{longtable}{p{3.7cm}p{3.7cm}p{3.7cm}p{3.7cm}}\hline
\multicolumn{2}{r}{\textbf{GutHub Packages:}} &
\multicolumn{2}{l}{\href{https://github.com/lsst/validate_base}{validate\_base} }
\\ \cline{3-4}
& & \multicolumn{2}{l}{\href{https://github.com/lsst/verify}{verify} }
\\ \cline{3-4}
& & \multicolumn{2}{l}{\href{https://github.com/lsst/squash-*}{squash-*} }
\\ \cline{3-4}
& & \multicolumn{2}{l}{\href{https://github.com/lsst/verify_metrics}{verify\_metrics} }
\\ \cline{3-4}
& & \multicolumn{2}{l}{\href{https://github.com/lsst/validate_drp}{validate\_drp} }
\\ \hline \\ \hline
\textbf{\footnotesize Depends on:}  & & \textbf{\footnotesize Used in:} & \\ \hline
&
\multicolumn{2}{c}{
\begin{tabular}{c}
\hyperref[offlqcsrv]{Offline Quality Control} \\ \hline
\hyperref[prqcsrv]{Prompt Quality Control} \\ \hline
\end{tabular} }
\\ \bottomrule
\end{longtable}

\subsubsection{Supporting SW Products}\label{suppsw}
\begin{longtable}{p{3.7cm}p{3.7cm}p{3.7cm}p{3.7cm}}\hline
\textbf{Manager} & \textbf{Owner} & \textbf{WBS} & \textbf{Team} \\ \hline
 &
\begin{tabular}{@{}l@{}}
Multiple \\
\end{tabular} & \begin{tabular}{@{}l@{}}
 \\
\end{tabular} & \begin{tabular}{@{}l@{}}
 \\
\end{tabular} \\ \hline
\multicolumn{4}{c}{
{\footnotesize ( Short Name: Supporting SW - Acronym: SUPPSW ) }
}\\ \hline
\end{longtable}

  DM software products used as libraries and shared between different
other DM software products.



\begin{longtable}{p{3.7cm}p{3.7cm}p{3.7cm}p{3.7cm}}\toprule
\multicolumn{2}{l}{\large \textbf{ ADQL Translator } }\label{adql}
& \multicolumn{2}{l}{(product in: Supporting SW)}
\\ \hline
\textbf{\footnotesize Manager} & \textbf{\footnotesize Owner} &
\textbf{\footnotesize WBS} & \textbf{\footnotesize Team} \\ \hline
Fritz Mueller &
\begin{tabular}{@{}l@{}}
Colin Slater \\
\end{tabular} & \begin{tabular}{@{}l@{}}
1.02C.06.02.05 \\
\end{tabular} & \begin{tabular}{@{}l@{}}
DAX \\
\end{tabular} \\ \hline
\multicolumn{4}{c}{
{\footnotesize ( Short Name: ADQL Translator - Acronym: ADQL ) }
}\\ \hline
\end{longtable}

  {[}OBSOLETE{]}\\
DAX Query Services in Kotlin.\\[2\baselineskip]



\begin{longtable}{p{3.7cm}p{3.7cm}p{3.7cm}p{3.7cm}}\hline
\multicolumn{2}{r}{\textbf{GutHub Packages:}} &
\multicolumn{2}{l}{\href{https://github.com/lsst/albuquery}{albuquery} }
\\ \hline \\ \hline
\textbf{\footnotesize Depends on:}  & & \textbf{\footnotesize Used in:} & \\ \hline
&
\\ \bottomrule
\end{longtable}

\begin{longtable}{p{3.7cm}p{3.7cm}p{3.7cm}p{3.7cm}}\toprule
\multicolumn{2}{l}{\large \textbf{ Data Butler } }\label{butler}
& \multicolumn{2}{l}{(product in: Supporting SW)}
\\ \hline
\textbf{\footnotesize Manager} & \textbf{\footnotesize Owner} &
\textbf{\footnotesize WBS} & \textbf{\footnotesize Team} \\ \hline
Fritz Mueller &
\begin{tabular}{@{}l@{}}
Jim Bosch \\
\end{tabular} & \begin{tabular}{@{}l@{}}
1.02C.06.02.01 \\
\end{tabular} & \begin{tabular}{@{}l@{}}
DAX \\
\end{tabular} \\ \hline
\multicolumn{4}{c}{
{\footnotesize ( Short Name: Data Butler - Acronym: BUTLER ) }
}\\ \hline
\end{longtable}

  Butler middleware software product.



\begin{longtable}{p{3.7cm}p{3.7cm}p{3.7cm}p{3.7cm}}\hline
\multicolumn{2}{r}{\textbf{GutHub Packages:}} &
\multicolumn{2}{l}{\href{https://github.com/lsst/daf_butler}{daf\_butler} }
\\ \hline \\ \hline
\textbf{\footnotesize Depends on:}  & & \textbf{\footnotesize Used in:} & \\ \hline
&
\multicolumn{2}{c}{
\begin{tabular}{c}
\hyperref[spdist]{Science Pipelines Distribution} \\ \hline
\end{tabular} }
\\ \bottomrule
\end{longtable}

\begin{longtable}{p{3.7cm}p{3.7cm}p{3.7cm}p{3.7cm}}\toprule
\multicolumn{2}{l}{\large \textbf{ Distributed Database } }\label{qserv}
& \multicolumn{2}{l}{(product in: Supporting SW)}
\\ \hline
\textbf{\footnotesize Manager} & \textbf{\footnotesize Owner} &
\textbf{\footnotesize WBS} & \textbf{\footnotesize Team} \\ \hline
Fritz Mueller &
\begin{tabular}{@{}l@{}}
Colin Slater \\
\end{tabular} & \begin{tabular}{@{}l@{}}
1.02C.06.02.03 \\
\end{tabular} & \begin{tabular}{@{}l@{}}
DAX \\
\end{tabular} \\ \hline
\multicolumn{4}{c}{
{\footnotesize ( Short Name: Distrib Database - Acronym: QSERV ) }
}\\ \hline
\end{longtable}

  Distributed database software product. This is linked to the top level
package in github.



\begin{longtable}{p{3.7cm}p{3.7cm}p{3.7cm}p{3.7cm}}\hline
\multicolumn{2}{r}{\textbf{GutHub Packages:}} &
\multicolumn{2}{l}{\href{https://github.com/lsst/qserv}{qserv} }
\\ \hline \\ \hline
\textbf{\footnotesize Depends on:}  & & \textbf{\footnotesize Used in:} & \\ \hline
&
\multicolumn{2}{c}{
\begin{tabular}{c}
\hyperref[lspdb]{LSP Database} \\ \hline
\end{tabular} }
\\ \bottomrule
\end{longtable}

\begin{longtable}{p{3.7cm}p{3.7cm}p{3.7cm}p{3.7cm}}\toprule
\multicolumn{2}{l}{\large \textbf{ Science Pipelines Libraries } }\label{scipipe}
& \multicolumn{2}{l}{(product in: Supporting SW)}
\\ \hline
\textbf{\footnotesize Manager} & \textbf{\footnotesize Owner} &
\textbf{\footnotesize WBS} & \textbf{\footnotesize Team} \\ \hline
John Swinbank &
\begin{tabular}{@{}l@{}}
Jim Bosch \\
\end{tabular} & \begin{tabular}{@{}l@{}}
1.02C.04 \\
1.02C.03 \\
\end{tabular} & \begin{tabular}{@{}l@{}}
DRP \\
AP \\
\end{tabular} \\ \hline
\multicolumn{4}{c}{
{\footnotesize ( Short Name: Sci Pipelines Libs - Acronym: SCIPIPE ) }
}\\ \hline
\end{longtable}

  Science Pipeline Top Level Software product.



\begin{longtable}{p{3.7cm}p{3.7cm}p{3.7cm}p{3.7cm}}\hline
\multicolumn{2}{r}{\textbf{GutHub Packages:}} &
\multicolumn{2}{l}{\href{https://github.com/lsst/lsst_apps}{lsst\_apps} }
\\ \hline \\ \hline
\textbf{\footnotesize Depends on:}  & & \textbf{\footnotesize Used in:} & \\ \hline
&
\multicolumn{2}{c}{
\begin{tabular}{c}
\hyperref[apprmpt]{Alert Production} \\ \hline
\hyperref[dmcal]{Calibration SW} \\ \hline
\hyperref[drp]{Data Release Production} \\ \hline
\hyperref[mops]{MOPS and Forced Photometry} \\ \hline
\hyperref[sp]{Special Programs Productions} \\ \hline
\hyperref[spdist]{Science Pipelines Distribution} \\ \hline
\end{tabular} }
\\ \bottomrule
\end{longtable}

\begin{longtable}{p{3.7cm}p{3.7cm}p{3.7cm}p{3.7cm}}\toprule
\multicolumn{2}{l}{\large \textbf{ Task Framework } }\label{txf}
& \multicolumn{2}{l}{(product in: Supporting SW)}
\\ \hline
\textbf{\footnotesize Manager} & \textbf{\footnotesize Owner} &
\textbf{\footnotesize WBS} & \textbf{\footnotesize Team} \\ \hline
Fritz Mueller &
\begin{tabular}{@{}l@{}}
Jim Bosch \\
\end{tabular} & \begin{tabular}{@{}l@{}}
1.02C.06.03 \\
\end{tabular} & \begin{tabular}{@{}l@{}}
DAX \\
\end{tabular} \\ \hline
\multicolumn{4}{c}{
{\footnotesize ( Short Name: Task Framework - Acronym: TXF ) }
}\\ \hline
\end{longtable}

  SuperTask middleware software product.



\begin{longtable}{p{3.7cm}p{3.7cm}p{3.7cm}p{3.7cm}}\hline
\multicolumn{2}{r}{\textbf{GutHub Packages:}} &
\multicolumn{2}{l}{\href{https://github.com/lsst/pipe_supertask}{pipe\_supertask} }
\\ \hline \\ \hline
\textbf{\footnotesize Depends on:}  & & \textbf{\footnotesize Used in:} & \\ \hline
&
\multicolumn{2}{c}{
\begin{tabular}{c}
\hyperref[spdist]{Science Pipelines Distribution} \\ \hline
\end{tabular} }
\\ \bottomrule
\end{longtable}

\subsection{Hardware and COTS Resources}\label{hwcots}
\begin{longtable}{p{3.7cm}p{3.7cm}p{3.7cm}p{3.7cm}}\hline
\textbf{Manager} & \textbf{Owner} & \textbf{WBS} & \textbf{Team} \\ \hline
 &
\begin{tabular}{@{}l@{}}
Multiple \\
\end{tabular} & \begin{tabular}{@{}l@{}}
 \\
\end{tabular} & \begin{tabular}{@{}l@{}}
 \\
\end{tabular} \\ \hline
\multicolumn{4}{c}{
{\footnotesize ( Short Name: Resources - Acronym: HWCOTS ) }
}\\ \hline
\end{longtable}

  External resources needed to implement the DM services:

\begin{itemize}
\tightlist
\item
  Hardware (computing hardware, network hardware, etc)
\item
  COTS
\item
  External software (not developed in DM)
\end{itemize}



\subsubsection{COTS}\label{cots}
\begin{longtable}{p{3.7cm}p{3.7cm}p{3.7cm}p{3.7cm}}\hline
\textbf{Manager} & \textbf{Owner} & \textbf{WBS} & \textbf{Team} \\ \hline
 &
\begin{tabular}{@{}l@{}}
Multiple \\
\end{tabular} & \begin{tabular}{@{}l@{}}
 \\
\end{tabular} & \begin{tabular}{@{}l@{}}
 \\
\end{tabular} \\ \hline
\multicolumn{4}{c}{
{\footnotesize ( Short Name: COTS SW - Acronym: COTS ) }
}\\ \hline
\end{longtable}

  This product includes all COTS used to implement the DM services.



\begin{longtable}{p{3.7cm}p{3.7cm}p{3.7cm}p{3.7cm}}\toprule
\multicolumn{2}{l}{\large \textbf{ CILogon } }\label{cilogon}
& \multicolumn{2}{l}{(product in: COTS SW)}
\\ \hline
\textbf{\footnotesize Manager} & \textbf{\footnotesize Owner} &
\textbf{\footnotesize WBS} & \textbf{\footnotesize Team} \\ \hline
 &
\begin{tabular}{@{}l@{}}
 \\
\end{tabular} & \begin{tabular}{@{}l@{}}
 \\
\end{tabular} & \begin{tabular}{@{}l@{}}
 \\
\end{tabular} \\ \hline
\multicolumn{4}{c}{
{\footnotesize ( Short Name: CILogon - Acronym: CILOGON ) }
}\\ \hline
\end{longtable}

  CILogon provides an integrated open source identity and access
management platform for research collaborations.



\begin{longtable}{p{3.7cm}p{3.7cm}p{3.7cm}p{3.7cm}}\hline
\textbf{References} &
\multicolumn{3}{l}{\href{http://www.cilogon.org/}{http://www.cilogon.org/} }
\\ \hline \\ \hline
\textbf{\footnotesize Depends on:}  & & \textbf{\footnotesize Used in:} & \\ \hline
&
\multicolumn{2}{c}{
\begin{tabular}{c}
\hyperref[]{Identity Management} \\ \hline
\end{tabular} }
\\ \bottomrule
\end{longtable}

\begin{longtable}{p{3.7cm}p{3.7cm}p{3.7cm}p{3.7cm}}\toprule
\multicolumn{2}{l}{\large \textbf{ Docker } }\label{docker}
& \multicolumn{2}{l}{(product in: COTS SW)}
\\ \hline
\textbf{\footnotesize Manager} & \textbf{\footnotesize Owner} &
\textbf{\footnotesize WBS} & \textbf{\footnotesize Team} \\ \hline
 &
\begin{tabular}{@{}l@{}}
 \\
\end{tabular} & \begin{tabular}{@{}l@{}}
 \\
\end{tabular} & \begin{tabular}{@{}l@{}}
 \\
\end{tabular} \\ \hline
\multicolumn{4}{c}{
{\footnotesize ( Short Name: Docker - Acronym: DOCKER ) }
}\\ \hline
\end{longtable}

  3rd party software product used to perform operating system level
virtualizations.



\begin{longtable}{p{3.7cm}p{3.7cm}p{3.7cm}p{3.7cm}}\hline
\textbf{References} &
\multicolumn{3}{l}{\href{https://www.docker.com/}{https://www.docker.com/} }
\\ \hline \\ \hline
\textbf{\footnotesize Depends on:}  & & \textbf{\footnotesize Used in:} & \\ \hline
&
\multicolumn{2}{c}{
\begin{tabular}{c}
\hyperref[cam]{Containerized Application Management} \\ \hline
\end{tabular} }
\\ \bottomrule
\end{longtable}

\begin{longtable}{p{3.7cm}p{3.7cm}p{3.7cm}p{3.7cm}}\toprule
\multicolumn{2}{l}{\large \textbf{ GPFS } }\label{gpfs}
& \multicolumn{2}{l}{(product in: COTS SW)}
\\ \hline
\textbf{\footnotesize Manager} & \textbf{\footnotesize Owner} &
\textbf{\footnotesize WBS} & \textbf{\footnotesize Team} \\ \hline
 &
\begin{tabular}{@{}l@{}}
 \\
\end{tabular} & \begin{tabular}{@{}l@{}}
 \\
\end{tabular} & \begin{tabular}{@{}l@{}}
 \\
\end{tabular} \\ \hline
\multicolumn{4}{c}{
{\footnotesize ( Short Name: GPFS - Acronym: GPFS ) }
}\\ \hline
\end{longtable}

  The General Parallel File System (GPFS) is a high-performance clustered
file system developed by IBM.



\begin{longtable}{p{3.7cm}p{3.7cm}p{3.7cm}p{3.7cm}}\hline
\textbf{References} &
\multicolumn{3}{l}{\href{https://en.wikipedia.org/wiki/IBM_General_Parallel_File_System}{https://en.wikipedia.org/wiki/IBM\_General\_Parallel\_File\_System} }
\\ \hline \\ \hline
\textbf{\footnotesize Depends on:}  & & \textbf{\footnotesize Used in:} & \\ \hline
&
\multicolumn{2}{c}{
\begin{tabular}{c}
\hyperref[]{ICT Provisioning and Management} \\ \hline
\end{tabular} }
\\ \bottomrule
\end{longtable}

\begin{longtable}{p{3.7cm}p{3.7cm}p{3.7cm}p{3.7cm}}\toprule
\multicolumn{2}{l}{\large \textbf{ Grafana } }\label{grafana}
& \multicolumn{2}{l}{(product in: COTS SW)}
\\ \hline
\textbf{\footnotesize Manager} & \textbf{\footnotesize Owner} &
\textbf{\footnotesize WBS} & \textbf{\footnotesize Team} \\ \hline
 &
\begin{tabular}{@{}l@{}}
 \\
\end{tabular} & \begin{tabular}{@{}l@{}}
 \\
\end{tabular} & \begin{tabular}{@{}l@{}}
 \\
\end{tabular} \\ \hline
\multicolumn{4}{c}{
{\footnotesize ( Short Name: Grafana - Acronym: GRAFANA ) }
}\\ \hline
\end{longtable}

  3rd party software product for analytics and monitoring



\begin{longtable}{p{3.7cm}p{3.7cm}p{3.7cm}p{3.7cm}}\hline
\textbf{References} &
\multicolumn{3}{l}{\href{https://grafana.com/}{https://grafana.com/} }
\\ \hline \\ \hline
\textbf{\footnotesize Depends on:}  & & \textbf{\footnotesize Used in:} & \\ \hline
&
\multicolumn{2}{c}{
\begin{tabular}{c}
\hyperref[]{Monitoring} \\ \hline
\end{tabular} }
\\ \bottomrule
\end{longtable}

\begin{longtable}{p{3.7cm}p{3.7cm}p{3.7cm}p{3.7cm}}\toprule
\multicolumn{2}{l}{\large \textbf{ HTCondor } }\label{htcondor}
& \multicolumn{2}{l}{(product in: COTS SW)}
\\ \hline
\textbf{\footnotesize Manager} & \textbf{\footnotesize Owner} &
\textbf{\footnotesize WBS} & \textbf{\footnotesize Team} \\ \hline
 &
\begin{tabular}{@{}l@{}}
 \\
\end{tabular} & \begin{tabular}{@{}l@{}}
 \\
\end{tabular} & \begin{tabular}{@{}l@{}}
 \\
\end{tabular} \\ \hline
\multicolumn{4}{c}{
{\footnotesize ( Short Name: HTCondor - Acronym: HTCONDOR ) }
}\\ \hline
\end{longtable}

  HTCondor is an open-source high-throughput computing software framework
for coarse-grained distributed parallelization of computationally
intensive tasks.



\begin{longtable}{p{3.7cm}p{3.7cm}p{3.7cm}p{3.7cm}}\hline
\textbf{References} &
\multicolumn{3}{l}{\href{https://research.cs.wisc.edu/htcondor/}{https://research.cs.wisc.edu/htcondor/} }
\\ \cline{2-4}
& \multicolumn{3}{l}{\href{https://github.com/htcondor}{https://github.com/htcondor} }
\\ \hline \\ \hline
\textbf{\footnotesize Depends on:}  & & \textbf{\footnotesize Used in:} & \\ \hline
&
\multicolumn{2}{c}{
\begin{tabular}{c}
\hyperref[prodsrv]{Batch Production} \\ \hline
\end{tabular} }
\\ \bottomrule
\end{longtable}

\begin{longtable}{p{3.7cm}p{3.7cm}p{3.7cm}p{3.7cm}}\toprule
\multicolumn{2}{l}{\large \textbf{ Kubernetes } }\label{k8s}
& \multicolumn{2}{l}{(product in: COTS SW)}
\\ \hline
\textbf{\footnotesize Manager} & \textbf{\footnotesize Owner} &
\textbf{\footnotesize WBS} & \textbf{\footnotesize Team} \\ \hline
 &
\begin{tabular}{@{}l@{}}
 \\
\end{tabular} & \begin{tabular}{@{}l@{}}
 \\
\end{tabular} & \begin{tabular}{@{}l@{}}
 \\
\end{tabular} \\ \hline
\multicolumn{4}{c}{
{\footnotesize ( Short Name: Kubernetes - Acronym: K8S ) }
}\\ \hline
\end{longtable}

  Kubernetes is an open-source system for automating deployment, scaling,
and management of containerized applications.



\begin{longtable}{p{3.7cm}p{3.7cm}p{3.7cm}p{3.7cm}}\hline
\textbf{References} &
\multicolumn{3}{l}{\href{https://kubernetes.io/}{https://kubernetes.io/} }
\\ \hline \\ \hline
\textbf{\footnotesize Depends on:}  & & \textbf{\footnotesize Used in:} & \\ \hline
&
\multicolumn{2}{c}{
\begin{tabular}{c}
\hyperref[cam]{Containerized Application Management} \\ \hline
\end{tabular} }
\\ \bottomrule
\end{longtable}

\begin{longtable}{p{3.7cm}p{3.7cm}p{3.7cm}p{3.7cm}}\toprule
\multicolumn{2}{l}{\large \textbf{ Oracle } }\label{oracle}
& \multicolumn{2}{l}{(product in: COTS SW)}
\\ \hline
\textbf{\footnotesize Manager} & \textbf{\footnotesize Owner} &
\textbf{\footnotesize WBS} & \textbf{\footnotesize Team} \\ \hline
 &
\begin{tabular}{@{}l@{}}
 \\
\end{tabular} & \begin{tabular}{@{}l@{}}
 \\
\end{tabular} & \begin{tabular}{@{}l@{}}
 \\
\end{tabular} \\ \hline
\multicolumn{4}{c}{
{\footnotesize ( Short Name: Oracle - Acronym: ORACLE ) }
}\\ \hline
\end{longtable}

  Oracle Relational Database 3rd party software product.



\begin{longtable}{p{3.7cm}p{3.7cm}p{3.7cm}p{3.7cm}}\hline
\textbf{References} &
\multicolumn{3}{l}{\href{https://www.oracle.com/index.html}{https://www.oracle.com/index.html} }
\\ \hline \\ \hline
\textbf{\footnotesize Depends on:}  & & \textbf{\footnotesize Used in:} & \\ \hline
&
\multicolumn{2}{c}{
\begin{tabular}{c}
\hyperref[]{Database Management} \\ \hline
\end{tabular} }
\\ \bottomrule
\end{longtable}

\begin{longtable}{p{3.7cm}p{3.7cm}p{3.7cm}p{3.7cm}}\toprule
\multicolumn{2}{l}{\large \textbf{ Puppet } }\label{puppet}
& \multicolumn{2}{l}{(product in: COTS SW)}
\\ \hline
\textbf{\footnotesize Manager} & \textbf{\footnotesize Owner} &
\textbf{\footnotesize WBS} & \textbf{\footnotesize Team} \\ \hline
 &
\begin{tabular}{@{}l@{}}
 \\
\end{tabular} & \begin{tabular}{@{}l@{}}
 \\
\end{tabular} & \begin{tabular}{@{}l@{}}
 \\
\end{tabular} \\ \hline
\multicolumn{4}{c}{
{\footnotesize ( Short Name: Puppet - Acronym: PUPPET ) }
}\\ \hline
\end{longtable}

  Puppet is an open-source software configuration management tool.



\begin{longtable}{p{3.7cm}p{3.7cm}p{3.7cm}p{3.7cm}}\hline
\textbf{References} &
\multicolumn{3}{l}{\href{https://en.wikipedia.org/wiki/Puppet_(software)}{https://en.wikipedia.org/wiki/Puppet\_(software)} }
\\ \cline{2-4}
& \multicolumn{3}{l}{\href{https://puppet.com/}{https://puppet.com/} }
\\ \hline \\ \hline
\textbf{\footnotesize Depends on:}  & & \textbf{\footnotesize Used in:} & \\ \hline
&
\multicolumn{2}{c}{
\begin{tabular}{c}
\hyperref[]{ICT Provisioning and Management} \\ \hline
\end{tabular} }
\\ \bottomrule
\end{longtable}

\begin{longtable}{p{3.7cm}p{3.7cm}p{3.7cm}p{3.7cm}}\toprule
\multicolumn{2}{l}{\large \textbf{ Rucio } }\label{rucio}
& \multicolumn{2}{l}{(product in: COTS SW)}
\\ \hline
\textbf{\footnotesize Manager} & \textbf{\footnotesize Owner} &
\textbf{\footnotesize WBS} & \textbf{\footnotesize Team} \\ \hline
 &
\begin{tabular}{@{}l@{}}
Michelle Butler \\
\end{tabular} & \begin{tabular}{@{}l@{}}
 \\
\end{tabular} & \begin{tabular}{@{}l@{}}
 \\
\end{tabular} \\ \hline
\multicolumn{4}{c}{
{\footnotesize ( Short Name: Rucio - Acronym: RUCIO ) }
}\\ \hline
\end{longtable}

  



\begin{longtable}{p{3.7cm}p{3.7cm}p{3.7cm}p{3.7cm}}\hline
\textbf{References} &
\multicolumn{3}{l}{\href{http://rucio.cern.ch/}{http://rucio.cern.ch/} }
\\ \cline{2-4}
& \multicolumn{3}{l}{\href{http://rucio.readthedocs.io/}{http://rucio.readthedocs.io/} }
\\ \hline \\ \hline
\textbf{\footnotesize Depends on:}  & & \textbf{\footnotesize Used in:} & \\ \hline
&
\multicolumn{2}{c}{
\begin{tabular}{c}
\hyperref[bulkdsrv]{Bulk Distribution} \\ \hline
\end{tabular} }
\\ \bottomrule
\end{longtable}

\begin{longtable}{p{3.7cm}p{3.7cm}p{3.7cm}p{3.7cm}}\toprule
\multicolumn{2}{l}{\large \textbf{ IT Security SW } }\label{security}
& \multicolumn{2}{l}{(product in: COTS SW)}
\\ \hline
\textbf{\footnotesize Manager} & \textbf{\footnotesize Owner} &
\textbf{\footnotesize WBS} & \textbf{\footnotesize Team} \\ \hline
 &
\begin{tabular}{@{}l@{}}
 \\
\end{tabular} & \begin{tabular}{@{}l@{}}
 \\
\end{tabular} & \begin{tabular}{@{}l@{}}
 \\
\end{tabular} \\ \hline
\multicolumn{4}{c}{
{\footnotesize ( Short Name: IT Security - Acronym: SECURITY ) }
}\\ \hline
\end{longtable}

  



\begin{longtable}{p{3.7cm}p{3.7cm}p{3.7cm}p{3.7cm}}\hline
\textbf{\footnotesize Depends on:}  & & \textbf{\footnotesize Used in:} & \\ \hline
&
\multicolumn{2}{c}{
\begin{tabular}{c}
\hyperref[]{IT Security} \\ \hline
\end{tabular} }
\\ \bottomrule
\end{longtable}

\begin{longtable}{p{3.7cm}p{3.7cm}p{3.7cm}p{3.7cm}}\toprule
\multicolumn{2}{l}{\large \textbf{ vSphere } }\label{vsphere}
& \multicolumn{2}{l}{(product in: COTS SW)}
\\ \hline
\textbf{\footnotesize Manager} & \textbf{\footnotesize Owner} &
\textbf{\footnotesize WBS} & \textbf{\footnotesize Team} \\ \hline
 &
\begin{tabular}{@{}l@{}}
 \\
\end{tabular} & \begin{tabular}{@{}l@{}}
 \\
\end{tabular} & \begin{tabular}{@{}l@{}}
 \\
\end{tabular} \\ \hline
\multicolumn{4}{c}{
{\footnotesize ( Short Name: vSphere - Acronym: VSPHERE ) }
}\\ \hline
\end{longtable}

  Third party software product for virtualization.



\begin{longtable}{p{3.7cm}p{3.7cm}p{3.7cm}p{3.7cm}}\hline
\textbf{References} &
\multicolumn{3}{l}{\href{https://www.vmware.com/products/vsphere.html}{https://www.vmware.com/products/vsphere.html} }
\\ \hline \\ \hline
\textbf{\footnotesize Depends on:}  & & \textbf{\footnotesize Used in:} & \\ \hline
&
\multicolumn{2}{c}{
\begin{tabular}{c}
\hyperref[]{ICT Provisioning and Management} \\ \hline
\end{tabular} }
\\ \bottomrule
\end{longtable}

\subsubsection{Compute Nodes}\label{hwcomp}
\begin{longtable}{p{3.7cm}p{3.7cm}p{3.7cm}p{3.7cm}}\hline
\textbf{Manager} & \textbf{Owner} & \textbf{WBS} & \textbf{Team} \\ \hline
 &
\begin{tabular}{@{}l@{}}
Michelle Butler \\
\end{tabular} & \begin{tabular}{@{}l@{}}
 \\
\end{tabular} & \begin{tabular}{@{}l@{}}
 \\
\end{tabular} \\ \hline
\multicolumn{4}{c}{
{\footnotesize ( Short Name: Compute Nodes - Acronym: HWCOMP ) }
}\\ \hline
\end{longtable}

  Computational elements: workstation and nodes that provide COP and
memory to process the data, run services. Secondary hardware parts, like
for example racks, re also included in this section. Details will be
added here on a per need base.



\begin{longtable}{p{3.7cm}p{3.7cm}p{3.7cm}p{3.7cm}}\toprule
\multicolumn{2}{l}{\large \textbf{ Dell blade } }\label{dell8}
& \multicolumn{2}{l}{(product in: Compute Nodes)}
\\ \hline
\textbf{\footnotesize Manager} & \textbf{\footnotesize Owner} &
\textbf{\footnotesize WBS} & \textbf{\footnotesize Team} \\ \hline
Margaret Johnson &
\begin{tabular}{@{}l@{}}
Michelle Butler \\
\end{tabular} & \begin{tabular}{@{}l@{}}
1.02C.07.09 \\
\end{tabular} & \begin{tabular}{@{}l@{}}
LDF \\
\end{tabular} \\ \hline
\multicolumn{4}{c}{
{\footnotesize ( Short Name: Blade (e.g.) - Acronym: DELL8 ) }
}\\ \hline
\end{longtable}

  This is just an example of computational element to be included in this
package.\\

Main Characteristics:

\begin{itemize}
\tightlist
\item
  8 cores
\item
  16 GM ram
\item
  100 GB disk space\\
\end{itemize}



\begin{longtable}{p{3.7cm}p{3.7cm}p{3.7cm}p{3.7cm}}\hline
\textbf{\footnotesize Depends on:}  & & \textbf{\footnotesize Used in:} & \\ \hline
&
\\ \bottomrule
\end{longtable}

\subsubsection{Network Nodes}\label{hwnet}
\begin{longtable}{p{3.7cm}p{3.7cm}p{3.7cm}p{3.7cm}}\hline
\textbf{Manager} & \textbf{Owner} & \textbf{WBS} & \textbf{Team} \\ \hline
 &
\begin{tabular}{@{}l@{}}
Multiple \\
\end{tabular} & \begin{tabular}{@{}l@{}}
 \\
\end{tabular} & \begin{tabular}{@{}l@{}}
 \\
\end{tabular} \\ \hline
\multicolumn{4}{c}{
{\footnotesize ( Short Name: Network Nodes - Acronym: HWNET ) }
}\\ \hline
\end{longtable}

  The network components to be purchased in order to implement local and
long distance networks. The intension is to keep the scope of this
component very high level without detailing the different bit and pieces
in which it can be decomposed. However, in case more details are
required, they should be related to hardware required to implement the
different networks. Some examples are: - cables (coaxial, optic fiber,
other) - switches Details will be added here on a per need base.



\begin{longtable}{p{3.7cm}p{3.7cm}p{3.7cm}p{3.7cm}}\toprule
\multicolumn{2}{l}{\large \textbf{ Network Component X } }\label{netx}
& \multicolumn{2}{l}{(product in: Network Nodes)}
\\ \hline
\textbf{\footnotesize Manager} & \textbf{\footnotesize Owner} &
\textbf{\footnotesize WBS} & \textbf{\footnotesize Team} \\ \hline
Margaret Johnson &
\begin{tabular}{@{}l@{}}
Michelle Butler \\
\end{tabular} & \begin{tabular}{@{}l@{}}
1.02C.07.09 \\
\end{tabular} & \begin{tabular}{@{}l@{}}
LDF \\
\end{tabular} \\ \hline
\multicolumn{4}{c}{
{\footnotesize ( Short Name: Router (e.g.) - Acronym: NETX ) }
}\\ \hline
\end{longtable}

  This is just an example of what type of elements can be included in this
package



\begin{longtable}{p{3.7cm}p{3.7cm}p{3.7cm}p{3.7cm}}\hline
\textbf{\footnotesize Depends on:}  & & \textbf{\footnotesize Used in:} & \\ \hline
&
\\ \bottomrule
\end{longtable}

\subsubsection{Storage Nodes}\label{hwstor}
\begin{longtable}{p{3.7cm}p{3.7cm}p{3.7cm}p{3.7cm}}\hline
\textbf{Manager} & \textbf{Owner} & \textbf{WBS} & \textbf{Team} \\ \hline
 &
\begin{tabular}{@{}l@{}}
Michelle Butler \\
\end{tabular} & \begin{tabular}{@{}l@{}}
 \\
\end{tabular} & \begin{tabular}{@{}l@{}}
 \\
\end{tabular} \\ \hline
\multicolumn{4}{c}{
{\footnotesize ( Short Name: Storage Nodes - Acronym: HWSTOR ) }
}\\ \hline
\end{longtable}

  Storage components to host data. In general are disks, but tape storage
for long term archive shall also be included in this area. Details will
be added here on a per need base.



\begin{longtable}{p{3.7cm}p{3.7cm}p{3.7cm}p{3.7cm}}\toprule
\multicolumn{2}{l}{\large \textbf{ SAN disk 2TB storage } }\label{san2}
& \multicolumn{2}{l}{(product in: Storage Nodes)}
\\ \hline
\textbf{\footnotesize Manager} & \textbf{\footnotesize Owner} &
\textbf{\footnotesize WBS} & \textbf{\footnotesize Team} \\ \hline
Margaret Johnson &
\begin{tabular}{@{}l@{}}
Michelle Butler \\
\end{tabular} & \begin{tabular}{@{}l@{}}
1.02C.07.09 \\
\end{tabular} & \begin{tabular}{@{}l@{}}
LDF \\
\end{tabular} \\ \hline
\multicolumn{4}{c}{
{\footnotesize ( Short Name: Disk (e.g.) - Acronym: SAN2 ) }
}\\ \hline
\end{longtable}

  This is just an example of what has to be included in this package



\begin{longtable}{p{3.7cm}p{3.7cm}p{3.7cm}p{3.7cm}}\hline
\textbf{\footnotesize Depends on:}  & & \textbf{\footnotesize Used in:} & \\ \hline
&
\\ \bottomrule
\end{longtable}

\subsubsection{Low Level SW}\label{llsw}
\begin{longtable}{p{3.7cm}p{3.7cm}p{3.7cm}p{3.7cm}}\hline
\textbf{Manager} & \textbf{Owner} & \textbf{WBS} & \textbf{Team} \\ \hline
 &
\begin{tabular}{@{}l@{}}
Multiple \\
\end{tabular} & \begin{tabular}{@{}l@{}}
 \\
\end{tabular} & \begin{tabular}{@{}l@{}}
 \\
\end{tabular} \\ \hline
\multicolumn{4}{c}{
{\footnotesize ( Short Name: Low Level SW - Acronym: LLSW ) }
}\\ \hline
\end{longtable}

  Low level software required by the hardware components to work. In
general these components are provided by the vendor, but in some cases
it may be required to customize specifically for the project purposes.
Some examples are: - Firmwares - Operating Systems Details will be added
here on a per need base.



\begin{longtable}{p{3.7cm}p{3.7cm}p{3.7cm}p{3.7cm}}\toprule
\multicolumn{2}{l}{\large \textbf{ RedHat CentOS } }\label{centos}
& \multicolumn{2}{l}{(product in: Low Level SW)}
\\ \hline
\textbf{\footnotesize Manager} & \textbf{\footnotesize Owner} &
\textbf{\footnotesize WBS} & \textbf{\footnotesize Team} \\ \hline
 &
\begin{tabular}{@{}l@{}}
 \\
\end{tabular} & \begin{tabular}{@{}l@{}}
 \\
\end{tabular} & \begin{tabular}{@{}l@{}}
 \\
\end{tabular} \\ \hline
\multicolumn{4}{c}{
{\footnotesize ( Short Name: OS (e.g.) - Acronym: CENTOS ) }
}\\ \hline
\end{longtable}

  



\begin{longtable}{p{3.7cm}p{3.7cm}p{3.7cm}p{3.7cm}}\hline
\textbf{\footnotesize Depends on:}  & & \textbf{\footnotesize Used in:} & \\ \hline
&
\\ \bottomrule
\end{longtable}

\subsubsection{Third Party Libraries}\label{thpl}
\begin{longtable}{p{3.7cm}p{3.7cm}p{3.7cm}p{3.7cm}}\hline
\textbf{Manager} & \textbf{Owner} & \textbf{WBS} & \textbf{Team} \\ \hline
 &
\begin{tabular}{@{}l@{}}
 \\
\end{tabular} & \begin{tabular}{@{}l@{}}
 \\
\end{tabular} & \begin{tabular}{@{}l@{}}
 \\
\end{tabular} \\ \hline
\multicolumn{4}{c}{
{\footnotesize ( Short Name: Third Party Libs - Acronym: THPL ) }
}\\ \hline
\end{longtable}

  External libraries required by the DM SW products in order to compile
and to run.



\begin{longtable}{p{3.7cm}p{3.7cm}p{3.7cm}p{3.7cm}}\toprule
\multicolumn{2}{l}{\large \textbf{ Boost } }\label{boost}
& \multicolumn{2}{l}{(product in: Third Party Libs)}
\\ \hline
\textbf{\footnotesize Manager} & \textbf{\footnotesize Owner} &
\textbf{\footnotesize WBS} & \textbf{\footnotesize Team} \\ \hline
 &
\begin{tabular}{@{}l@{}}
 \\
\end{tabular} & \begin{tabular}{@{}l@{}}
 \\
\end{tabular} & \begin{tabular}{@{}l@{}}
 \\
\end{tabular} \\ \hline
\multicolumn{4}{c}{
{\footnotesize ( Short Name: Boost - Acronym: BOOST ) }
}\\ \hline
\end{longtable}

  



\begin{longtable}{p{3.7cm}p{3.7cm}p{3.7cm}p{3.7cm}}\hline
\textbf{\footnotesize Depends on:}  & & \textbf{\footnotesize Used in:} & \\ \hline
&
\\ \bottomrule
\end{longtable}

\begin{longtable}{p{3.7cm}p{3.7cm}p{3.7cm}p{3.7cm}}\toprule
\multicolumn{2}{l}{\large \textbf{ Firefly API Access } }\label{ffaa}
& \multicolumn{2}{l}{(product in: Third Party Libs)}
\\ \hline
\textbf{\footnotesize Manager} & \textbf{\footnotesize Owner} &
\textbf{\footnotesize WBS} & \textbf{\footnotesize Team} \\ \hline
Xiuqin Wu &
\begin{tabular}{@{}l@{}}
Gregory Dubois-Felsmann \\
\end{tabular} & \begin{tabular}{@{}l@{}}
 \\
\end{tabular} & \begin{tabular}{@{}l@{}}
SUIT \\
\end{tabular} \\ \hline
\multicolumn{4}{c}{
{\footnotesize ( Short Name: Firefly API A. - Acronym: FFAA ) }
}\\ \hline
\end{longtable}

  Provides hooks to allow Firefly's Javascript code to be loaded starting
from npm-style package loading



\begin{longtable}{p{3.7cm}p{3.7cm}p{3.7cm}p{3.7cm}}\hline
\multicolumn{2}{r}{\textbf{GutHub Packages:}} &
\multicolumn{2}{l}{\href{https://github.com/Caltech-IPAC/firefly-api-access}{Caltech-IPAC/firefly-api-access} }
\\ \hline \\ \hline
\textbf{\footnotesize Depends on:}  & & \textbf{\footnotesize Used in:} & \\ \hline
&
\\ \bottomrule
\end{longtable}

\begin{longtable}{p{3.7cm}p{3.7cm}p{3.7cm}p{3.7cm}}\toprule
\multicolumn{2}{l}{\large \textbf{ Firefly } }\label{firefly}
& \multicolumn{2}{l}{(product in: Third Party Libs)}
\\ \hline
\textbf{\footnotesize Manager} & \textbf{\footnotesize Owner} &
\textbf{\footnotesize WBS} & \textbf{\footnotesize Team} \\ \hline
Xiuqin Wu &
\begin{tabular}{@{}l@{}}
Gregory Dubois-Felsmann \\
\end{tabular} & \begin{tabular}{@{}l@{}}
1.02C.05.06 \\
\end{tabular} & \begin{tabular}{@{}l@{}}
SUIT \\
\end{tabular} \\ \hline
\multicolumn{4}{c}{
{\footnotesize ( Short Name: Firefly - Acronym: FIREFLY ) }
}\\ \hline
\end{longtable}

  IPAC's Advanced Astronomy WEB UI Framework.



\begin{longtable}{p{3.7cm}p{3.7cm}p{3.7cm}p{3.7cm}}\hline
\multicolumn{2}{r}{\textbf{GutHub Packages:}} &
\multicolumn{2}{l}{\href{https://github.com/Caltech-IPAC/firefly}{Caltech-IPAC/firefly} }
\\ \hline \\ \hline
\textbf{References} &
\multicolumn{3}{l}{\href{http://web.ipac.caltech.edu/staff/roby/camera-team-ff-overview.pdf}{http://web.ipac.caltech.edu/staff/roby/camera-team-ff-overview.pdf} }
\\ \hline \\ \hline
\textbf{\footnotesize Depends on:}  & & \textbf{\footnotesize Used in:} & \\ \hline
&
\multicolumn{2}{c}{
\begin{tabular}{c}
\hyperref[suit]{SUIT} \\ \hline
\end{tabular} }
\\ \bottomrule
\end{longtable}

\begin{longtable}{p{3.7cm}p{3.7cm}p{3.7cm}p{3.7cm}}\toprule
\multicolumn{2}{l}{\large \textbf{ Python } }\label{pth}
& \multicolumn{2}{l}{(product in: Third Party Libs)}
\\ \hline
\textbf{\footnotesize Manager} & \textbf{\footnotesize Owner} &
\textbf{\footnotesize WBS} & \textbf{\footnotesize Team} \\ \hline
 &
\begin{tabular}{@{}l@{}}
 \\
\end{tabular} & \begin{tabular}{@{}l@{}}
 \\
\end{tabular} & \begin{tabular}{@{}l@{}}
 \\
\end{tabular} \\ \hline
\multicolumn{4}{c}{
{\footnotesize ( Short Name: Python - Acronym: PTH ) }
}\\ \hline
\end{longtable}

  



\begin{longtable}{p{3.7cm}p{3.7cm}p{3.7cm}p{3.7cm}}\hline
\textbf{\footnotesize Depends on:}  & & \textbf{\footnotesize Used in:} & \\ \hline
&
\\ \bottomrule
\end{longtable}

\subsection{Infrastructure}\label{infra}
\begin{longtable}{p{3.7cm}p{3.7cm}p{3.7cm}p{3.7cm}}\hline
\textbf{Manager} & \textbf{Owner} & \textbf{WBS} & \textbf{Team} \\ \hline
 &
\begin{tabular}{@{}l@{}}
Multiple \\
\end{tabular} & \begin{tabular}{@{}l@{}}
 \\
\end{tabular} & \begin{tabular}{@{}l@{}}
 \\
\end{tabular} \\ \hline
\multicolumn{4}{c}{
{\footnotesize ( Short Name: Infrastructure - Acronym: INFRA ) }
}\\ \hline
\end{longtable}

  DM infrastructural products.

~

These products are mainly physical facilities or logical groups
services.

~



\subsubsection{Enclaves}\label{enc}
\begin{longtable}{p{3.7cm}p{3.7cm}p{3.7cm}p{3.7cm}}\hline
\textbf{Manager} & \textbf{Owner} & \textbf{WBS} & \textbf{Team} \\ \hline
 &
\begin{tabular}{@{}l@{}}
Multiple \\
\end{tabular} & \begin{tabular}{@{}l@{}}
 \\
\end{tabular} & \begin{tabular}{@{}l@{}}
 \\
\end{tabular} \\ \hline
\multicolumn{4}{c}{
{\footnotesize ( Short Name: Enclaves - Acronym: ENC ) }
}\\ \hline
\end{longtable}

  Logical group of DM Services.



\begin{longtable}{p{3.7cm}p{3.7cm}p{3.7cm}p{3.7cm}}\toprule
\multicolumn{2}{l}{\large \textbf{ Archive Base Enclave } }\label{encarcb}
& \multicolumn{2}{l}{(product in: Enclaves)}
\\ \hline
\textbf{\footnotesize Manager} & \textbf{\footnotesize Owner} &
\textbf{\footnotesize WBS} & \textbf{\footnotesize Team} \\ \hline
Margaret Johnson &
\begin{tabular}{@{}l@{}}
Michelle Butler \\
\end{tabular} & \begin{tabular}{@{}l@{}}
1.02C.08.01 \\
\end{tabular} & \begin{tabular}{@{}l@{}}
LDF \\
\end{tabular} \\ \hline
\multicolumn{4}{c}{
{\footnotesize ( Short Name: Arch Base Encl - Acronym: ENCARCB ) }
}\\ \hline
\end{longtable}

  This product groups all archive services (Data BackBone) located at the
Base facility.



\begin{longtable}{p{3.7cm}p{3.7cm}p{3.7cm}p{3.7cm}}\hline
\textbf{\footnotesize Depends on:}  & & \textbf{\footnotesize Used in:} & \\ \hline
\multicolumn{2}{c}{
\begin{tabular}{c}
\hyperref[dbbmdsrv]{DBB Ingest/ Metadata Management} \\ \hline
\hyperref[dbbtrsrv]{DBB Transport/ Replication/ Backup} \\ \hline
\hyperref[dbblifesrv]{DBB Lifetime Management} \\ \hline
\hyperref[dbbstrsrv]{DBB Storage} \\ \hline
\end{tabular} }
&
\multicolumn{2}{c}{
\begin{tabular}{c}
\hyperref[facbase]{Base Facility} \\ \hline
\end{tabular} }
\\ \bottomrule
\end{longtable}

\begin{longtable}{p{3.7cm}p{3.7cm}p{3.7cm}p{3.7cm}}\toprule
\multicolumn{2}{l}{\large \textbf{ Archive NCSA Enclave } }\label{encarcn}
& \multicolumn{2}{l}{(product in: Enclaves)}
\\ \hline
\textbf{\footnotesize Manager} & \textbf{\footnotesize Owner} &
\textbf{\footnotesize WBS} & \textbf{\footnotesize Team} \\ \hline
Margaret Johnson &
\begin{tabular}{@{}l@{}}
Michelle Butler \\
\end{tabular} & \begin{tabular}{@{}l@{}}
1.02C.07.09 \\
\end{tabular} & \begin{tabular}{@{}l@{}}
LDF \\
\end{tabular} \\ \hline
\multicolumn{4}{c}{
{\footnotesize ( Short Name: Arch NCSA Encl - Acronym: ENCARCN ) }
}\\ \hline
\end{longtable}

  This product groups all archive services (Data BackBone) located at the
NCSA facility.



\begin{longtable}{p{3.7cm}p{3.7cm}p{3.7cm}p{3.7cm}}\hline
\textbf{\footnotesize Depends on:}  & & \textbf{\footnotesize Used in:} & \\ \hline
\multicolumn{2}{c}{
\begin{tabular}{c}
\hyperref[dbbmdsrv]{DBB Ingest/ Metadata Management} \\ \hline
\hyperref[dbblifesrv]{DBB Lifetime Management} \\ \hline
\hyperref[dbbstrsrv]{DBB Storage} \\ \hline
\hyperref[dbbtrsrv]{DBB Transport/ Replication/ Backup} \\ \hline
\end{tabular} }
&
\multicolumn{2}{c}{
\begin{tabular}{c}
\hyperref[facncsa]{NCSA Facility} \\ \hline
\end{tabular} }
\\ \bottomrule
\end{longtable}

\begin{longtable}{p{3.7cm}p{3.7cm}p{3.7cm}p{3.7cm}}\toprule
\multicolumn{2}{l}{\large \textbf{ Commissioning Cluster Enclave } }\label{enccomm}
& \multicolumn{2}{l}{(product in: Enclaves)}
\\ \hline
\textbf{\footnotesize Manager} & \textbf{\footnotesize Owner} &
\textbf{\footnotesize WBS} & \textbf{\footnotesize Team} \\ \hline
Margaret Johnson &
\begin{tabular}{@{}l@{}}
Simon Krughoff \\
\end{tabular} & \begin{tabular}{@{}l@{}}
1.02C.08.01 \\
\end{tabular} & \begin{tabular}{@{}l@{}}
LDF \\
\end{tabular} \\ \hline
\multicolumn{4}{c}{
{\footnotesize ( Short Name: Comm Clust Encl - Acronym: ENCCOMM ) }
}\\ \hline
\end{longtable}

  This product groups all DAC services required for Commissioning, at the
Base facility.



\begin{longtable}{p{3.7cm}p{3.7cm}p{3.7cm}p{3.7cm}}\hline
\textbf{\footnotesize Depends on:}  & & \textbf{\footnotesize Used in:} & \\ \hline
\multicolumn{2}{c}{
\begin{tabular}{c}
\hyperref[lspprtlsrv]{LSP Portal} \\ \hline
\hyperref[lspjlsrv]{LSP JupyterLab} \\ \hline
\hyperref[lspwebsrv]{LSP Web API} \\ \hline
\hyperref[wdav]{WebDAV} \\ \hline
\hyperref[siasrv]{SIA} \\ \hline
\hyperref[soda]{SODA} \\ \hline
\hyperref[tapsev]{TAP} \\ \hline
\hyperref[lspdb]{LSP Database} \\ \hline
\end{tabular} }
&
\multicolumn{2}{c}{
\begin{tabular}{c}
\hyperref[facbase]{Base Facility} \\ \hline
\end{tabular} }
\\ \bottomrule
\end{longtable}

\begin{longtable}{p{3.7cm}p{3.7cm}p{3.7cm}p{3.7cm}}\toprule
\multicolumn{2}{l}{\large \textbf{ DAC Chile Enclave } }\label{encdacc}
& \multicolumn{2}{l}{(product in: Enclaves)}
\\ \hline
\textbf{\footnotesize Manager} & \textbf{\footnotesize Owner} &
\textbf{\footnotesize WBS} & \textbf{\footnotesize Team} \\ \hline
Margaret Johnson &
\begin{tabular}{@{}l@{}}
Michelle Butler \\
\end{tabular} & \begin{tabular}{@{}l@{}}
1.02C.08.02 \\
\end{tabular} & \begin{tabular}{@{}l@{}}
LDF \\
\end{tabular} \\ \hline
\multicolumn{4}{c}{
{\footnotesize ( Short Name: DAC Chile Encl - Acronym: ENCDACC ) }
}\\ \hline
\end{longtable}

  This product groups all DAC services required for nominal operations,
located at the Base facility.



\begin{longtable}{p{3.7cm}p{3.7cm}p{3.7cm}p{3.7cm}}\hline
\textbf{\footnotesize Depends on:}  & & \textbf{\footnotesize Used in:} & \\ \hline
\multicolumn{2}{c}{
\begin{tabular}{c}
\hyperref[lspjlsrv]{LSP JupyterLab} \\ \hline
\hyperref[lspwebsrv]{LSP Web API} \\ \hline
\hyperref[lspprtlsrv]{LSP Portal} \\ \hline
\hyperref[wdav]{WebDAV} \\ \hline
\hyperref[siasrv]{SIA} \\ \hline
\hyperref[soda]{SODA} \\ \hline
\hyperref[tapsev]{TAP} \\ \hline
\hyperref[lspdb]{LSP Database} \\ \hline
\end{tabular} }
&
\multicolumn{2}{c}{
\begin{tabular}{c}
\hyperref[facbase]{Base Facility} \\ \hline
\end{tabular} }
\\ \bottomrule
\end{longtable}

\begin{longtable}{p{3.7cm}p{3.7cm}p{3.7cm}p{3.7cm}}\toprule
\multicolumn{2}{l}{\large \textbf{ DAC US Enclave } }\label{encdacu}
& \multicolumn{2}{l}{(product in: Enclaves)}
\\ \hline
\textbf{\footnotesize Manager} & \textbf{\footnotesize Owner} &
\textbf{\footnotesize WBS} & \textbf{\footnotesize Team} \\ \hline
Margaret Johnson &
\begin{tabular}{@{}l@{}}
Michelle Butler \\
\end{tabular} & \begin{tabular}{@{}l@{}}
1.02C.07.09 \\
\end{tabular} & \begin{tabular}{@{}l@{}}
LDF \\
\end{tabular} \\ \hline
\multicolumn{4}{c}{
{\footnotesize ( Short Name: DAC US Encl - Acronym: ENCDACU ) }
}\\ \hline
\end{longtable}

  This product groups all DAC services required for operations, located at
the NCSA facility.



\begin{longtable}{p{3.7cm}p{3.7cm}p{3.7cm}p{3.7cm}}\hline
\textbf{\footnotesize Depends on:}  & & \textbf{\footnotesize Used in:} & \\ \hline
\multicolumn{2}{c}{
\begin{tabular}{c}
\hyperref[lspjlsrv]{LSP JupyterLab} \\ \hline
\hyperref[lspwebsrv]{LSP Web API} \\ \hline
\hyperref[lspprtlsrv]{LSP Portal} \\ \hline
\hyperref[wdav]{WebDAV} \\ \hline
\hyperref[siasrv]{SIA} \\ \hline
\hyperref[soda]{SODA} \\ \hline
\hyperref[tapsev]{TAP} \\ \hline
\hyperref[lspdb]{LSP Database} \\ \hline
\end{tabular} }
&
\multicolumn{2}{c}{
\begin{tabular}{c}
\hyperref[facncsa]{NCSA Facility} \\ \hline
\end{tabular} }
\\ \bottomrule
\end{longtable}

\begin{longtable}{p{3.7cm}p{3.7cm}p{3.7cm}p{3.7cm}}\toprule
\multicolumn{2}{l}{\large \textbf{ Offline Production Enclave } }\label{encoffl}
& \multicolumn{2}{l}{(product in: Enclaves)}
\\ \hline
\textbf{\footnotesize Manager} & \textbf{\footnotesize Owner} &
\textbf{\footnotesize WBS} & \textbf{\footnotesize Team} \\ \hline
Margaret Johnson &
\begin{tabular}{@{}l@{}}
Michelle Butler \\
\end{tabular} & \begin{tabular}{@{}l@{}}
1.02C.07.09 \\
\end{tabular} & \begin{tabular}{@{}l@{}}
LDF \\
\end{tabular} \\ \hline
\multicolumn{4}{c}{
{\footnotesize ( Short Name: Offline Prod Encl - Acronym: ENCOFFL ) }
}\\ \hline
\end{longtable}

  This product groups all processing services to be executed offline, at
the NCSA facility.



\begin{longtable}{p{3.7cm}p{3.7cm}p{3.7cm}p{3.7cm}}\hline
\textbf{\footnotesize Depends on:}  & & \textbf{\footnotesize Used in:} & \\ \hline
\multicolumn{2}{c}{
\begin{tabular}{c}
\hyperref[prodsrv]{Batch Production} \\ \hline
\hyperref[offlqcsrv]{Offline Quality Control} \\ \hline
\hyperref[bulkdsrv]{Bulk Distribution} \\ \hline
\end{tabular} }
&
\multicolumn{2}{c}{
\begin{tabular}{c}
\hyperref[facncsa]{NCSA Facility} \\ \hline
\end{tabular} }
\\ \bottomrule
\end{longtable}

\begin{longtable}{p{3.7cm}p{3.7cm}p{3.7cm}p{3.7cm}}\toprule
\multicolumn{2}{l}{\large \textbf{ Prompt Base Enclave } }\label{encprb}
& \multicolumn{2}{l}{(product in: Enclaves)}
\\ \hline
\textbf{\footnotesize Manager} & \textbf{\footnotesize Owner} &
\textbf{\footnotesize WBS} & \textbf{\footnotesize Team} \\ \hline
Margaret Johnson &
\begin{tabular}{@{}l@{}}
Michelle Butler \\
\end{tabular} & \begin{tabular}{@{}l@{}}
1.02C.08.01 \\
\end{tabular} & \begin{tabular}{@{}l@{}}
LDF \\
\end{tabular} \\ \hline
\multicolumn{4}{c}{
{\footnotesize ( Short Name: Prmpt Base Encl - Acronym: ENCPRB ) }
}\\ \hline
\end{longtable}

  This product groups all services required to be executed in a nearly
real time manner, at Base facility.



\begin{longtable}{p{3.7cm}p{3.7cm}p{3.7cm}p{3.7cm}}\hline
\textbf{\footnotesize Depends on:}  & & \textbf{\footnotesize Used in:} & \\ \hline
\multicolumn{2}{c}{
\begin{tabular}{c}
\hyperref[prpingsrv]{Prompt Processing Ingest} \\ \hline
\hyperref[tmgsrv]{Telemetry Gateway} \\ \hline
\hyperref[popsrv]{Planned Observation Publication} \\ \hline
\hyperref[ocsbatsrv]{OCS-Driven Batch} \\ \hline
\hyperref[oodssrv]{Observatory Operations Data} \\ \hline
\hyperref[arcsrv]{Archiving} \\ \hline
\hyperref[efdb]{EFD Cache} \\ \hline
\end{tabular} }
&
\multicolumn{2}{c}{
\begin{tabular}{c}
\hyperref[facbase]{Base Facility} \\ \hline
\end{tabular} }
\\ \bottomrule
\end{longtable}

\begin{longtable}{p{3.7cm}p{3.7cm}p{3.7cm}p{3.7cm}}\toprule
\multicolumn{2}{l}{\large \textbf{ Prompt NCSA Enclave } }\label{encprn}
& \multicolumn{2}{l}{(product in: Enclaves)}
\\ \hline
\textbf{\footnotesize Manager} & \textbf{\footnotesize Owner} &
\textbf{\footnotesize WBS} & \textbf{\footnotesize Team} \\ \hline
Margaret Johnson &
\begin{tabular}{@{}l@{}}
Michelle Butler \\
\end{tabular} & \begin{tabular}{@{}l@{}}
1.02C.07.09 \\
\end{tabular} & \begin{tabular}{@{}l@{}}
LDF \\
\end{tabular} \\ \hline
\multicolumn{4}{c}{
{\footnotesize ( Short Name: Prmpt NCSA Encl - Acronym: ENCPRN ) }
}\\ \hline
\end{longtable}

  This product groups all services required to be executed in a nearly
real time manner, at NCSA facility.



\begin{longtable}{p{3.7cm}p{3.7cm}p{3.7cm}p{3.7cm}}\hline
\textbf{\footnotesize Depends on:}  & & \textbf{\footnotesize Used in:} & \\ \hline
\multicolumn{2}{c}{
\begin{tabular}{c}
\hyperref[alrtdstsrv]{Alert Distribution} \\ \hline
\hyperref[prpingsrv]{Prompt Processing Ingest} \\ \hline
\hyperref[offlqcsrv]{Offline Quality Control} \\ \hline
\hyperref[prqcsrv]{Prompt Quality Control} \\ \hline
\hyperref[prprsrv]{Prompt Processing} \\ \hline
\end{tabular} }
&
\multicolumn{2}{c}{
\begin{tabular}{c}
\hyperref[facncsa]{NCSA Facility} \\ \hline
\end{tabular} }
\\ \bottomrule
\end{longtable}

\subsubsection{Facilities}\label{fac}
\begin{longtable}{p{3.7cm}p{3.7cm}p{3.7cm}p{3.7cm}}\hline
\textbf{Manager} & \textbf{Owner} & \textbf{WBS} & \textbf{Team} \\ \hline
 &
\begin{tabular}{@{}l@{}}
Multiple \\
\end{tabular} & \begin{tabular}{@{}l@{}}
 \\
\end{tabular} & \begin{tabular}{@{}l@{}}
 \\
\end{tabular} \\ \hline
\multicolumn{4}{c}{
{\footnotesize ( Short Name: Facilities - Acronym: FAC ) }
}\\ \hline
\end{longtable}

  Physical facilities where operational DM activities take place.



\begin{longtable}{p{3.7cm}p{3.7cm}p{3.7cm}p{3.7cm}}\toprule
\multicolumn{2}{l}{\large \textbf{ Base Facility } }\label{facbase}
& \multicolumn{2}{l}{(product in: Facilities)}
\\ \hline
\textbf{\footnotesize Manager} & \textbf{\footnotesize Owner} &
\textbf{\footnotesize WBS} & \textbf{\footnotesize Team} \\ \hline
Jeff Kantor &
\begin{tabular}{@{}l@{}}
Jeff Kantor \\
\end{tabular} & \begin{tabular}{@{}l@{}}
1.02C.08.01 \\
1.02C.08.02 \\
\end{tabular} & \begin{tabular}{@{}l@{}}
Net/Base \\
\end{tabular} \\ \hline
\multicolumn{4}{c}{
{\footnotesize ( Short Name: Base Facility - Acronym: FACBASE ) }
}\\ \hline
\end{longtable}

  Base facility located at La Serena, Chile.



\begin{longtable}{p{3.7cm}p{3.7cm}p{3.7cm}p{3.7cm}}\hline
\textbf{\footnotesize Depends on:}  & & \textbf{\footnotesize Used in:} & \\ \hline
\multicolumn{2}{c}{
\begin{tabular}{c}
\hyperref[encarcb]{Archive Base Enclave} \\ \hline
\hyperref[encdacc]{DAC Chile Enclave} \\ \hline
\hyperref[enccomm]{Commissioning Cluster Enclave} \\ \hline
\hyperref[encprb]{Prompt Base Enclave} \\ \hline
\end{tabular} }
&
\\ \bottomrule
\end{longtable}

\begin{longtable}{p{3.7cm}p{3.7cm}p{3.7cm}p{3.7cm}}\toprule
\multicolumn{2}{l}{\large \textbf{ NCSA Facility } }\label{facncsa}
& \multicolumn{2}{l}{(product in: Facilities)}
\\ \hline
\textbf{\footnotesize Manager} & \textbf{\footnotesize Owner} &
\textbf{\footnotesize WBS} & \textbf{\footnotesize Team} \\ \hline
Margaret Johnson &
\begin{tabular}{@{}l@{}}
Michelle Butler \\
\end{tabular} & \begin{tabular}{@{}l@{}}
1.02C.07.09 \\
\end{tabular} & \begin{tabular}{@{}l@{}}
LDF \\
\end{tabular} \\ \hline
\multicolumn{4}{c}{
{\footnotesize ( Short Name: NCSA Facility - Acronym: FACNCSA ) }
}\\ \hline
\end{longtable}

  NCSA processing facility located at Urbana, Illinois(USA).



\begin{longtable}{p{3.7cm}p{3.7cm}p{3.7cm}p{3.7cm}}\hline
\textbf{\footnotesize Depends on:}  & & \textbf{\footnotesize Used in:} & \\ \hline
\multicolumn{2}{c}{
\begin{tabular}{c}
\hyperref[encdacu]{DAC US Enclave} \\ \hline
\hyperref[encoffl]{Offline Production Enclave} \\ \hline
\hyperref[]{Development and Integration E.} \\ \hline
\hyperref[encprn]{Prompt NCSA Enclave} \\ \hline
\hyperref[encarcn]{Archive NCSA Enclave} \\ \hline
\end{tabular} }
&
\\ \bottomrule
\end{longtable}

\subsubsection{Networks}\label{net}
\begin{longtable}{p{3.7cm}p{3.7cm}p{3.7cm}p{3.7cm}}\hline
\textbf{Manager} & \textbf{Owner} & \textbf{WBS} & \textbf{Team} \\ \hline
 &
\begin{tabular}{@{}l@{}}
Multiple \\
\end{tabular} & \begin{tabular}{@{}l@{}}
 \\
\end{tabular} & \begin{tabular}{@{}l@{}}
 \\
\end{tabular} \\ \hline
\multicolumn{4}{c}{
{\footnotesize ( Short Name: Networks - Acronym: NET ) }
}\\ \hline
\end{longtable}

  High level physical DM networks definition.



\begin{longtable}{p{3.7cm}p{3.7cm}p{3.7cm}p{3.7cm}}\toprule
\multicolumn{2}{l}{\large \textbf{ Base to Archive Network } }\label{netba}
& \multicolumn{2}{l}{(product in: Networks)}
\\ \hline
\textbf{\footnotesize Manager} & \textbf{\footnotesize Owner} &
\textbf{\footnotesize WBS} & \textbf{\footnotesize Team} \\ \hline
Jeff Kantor &
\begin{tabular}{@{}l@{}}
Jeff Kantor \\
\end{tabular} & \begin{tabular}{@{}l@{}}
1.02C.08.03 \\
\end{tabular} & \begin{tabular}{@{}l@{}}
Net/Base \\
\end{tabular} \\ \hline
\multicolumn{4}{c}{
{\footnotesize ( Short Name: Base/Arch Net - Acronym: NETBA ) }
}\\ \hline
\end{longtable}

  



\begin{longtable}{p{3.7cm}p{3.7cm}p{3.7cm}p{3.7cm}}\hline
\textbf{\footnotesize Depends on:}  & & \textbf{\footnotesize Used in:} & \\ \hline
&
\\ \bottomrule
\end{longtable}

\begin{longtable}{p{3.7cm}p{3.7cm}p{3.7cm}p{3.7cm}}\toprule
\multicolumn{2}{l}{\large \textbf{ Base LAN Network } }\label{netbase}
& \multicolumn{2}{l}{(product in: Networks)}
\\ \hline
\textbf{\footnotesize Manager} & \textbf{\footnotesize Owner} &
\textbf{\footnotesize WBS} & \textbf{\footnotesize Team} \\ \hline
Margaret Johnson &
\begin{tabular}{@{}l@{}}
Michelle Butler \\
\end{tabular} & \begin{tabular}{@{}l@{}}
1.02C.07.08 \\
\end{tabular} & \begin{tabular}{@{}l@{}}
LDF \\
\end{tabular} \\ \hline
\multicolumn{4}{c}{
{\footnotesize ( Short Name: Base LAN - Acronym: NETBASE ) }
}\\ \hline
\end{longtable}

  Base Local Area Network.



\begin{longtable}{p{3.7cm}p{3.7cm}p{3.7cm}p{3.7cm}}\hline
\textbf{\footnotesize Depends on:}  & & \textbf{\footnotesize Used in:} & \\ \hline
&
\\ \bottomrule
\end{longtable}

\begin{longtable}{p{3.7cm}p{3.7cm}p{3.7cm}p{3.7cm}}\toprule
\multicolumn{2}{l}{\large \textbf{ NCSA LAN Network } }\label{netncsa}
& \multicolumn{2}{l}{(product in: Networks)}
\\ \hline
\textbf{\footnotesize Manager} & \textbf{\footnotesize Owner} &
\textbf{\footnotesize WBS} & \textbf{\footnotesize Team} \\ \hline
Margaret Johnson &
\begin{tabular}{@{}l@{}}
Michelle Butler \\
\end{tabular} & \begin{tabular}{@{}l@{}}
1.02C.07.09 \\
\end{tabular} & \begin{tabular}{@{}l@{}}
LDF \\
\end{tabular} \\ \hline
\multicolumn{4}{c}{
{\footnotesize ( Short Name: NCSA LAN - Acronym: NETNCSA ) }
}\\ \hline
\end{longtable}

  NCSA Local Area Network.



\begin{longtable}{p{3.7cm}p{3.7cm}p{3.7cm}p{3.7cm}}\hline
\textbf{\footnotesize Depends on:}  & & \textbf{\footnotesize Used in:} & \\ \hline
&
\\ \bottomrule
\end{longtable}

\begin{longtable}{p{3.7cm}p{3.7cm}p{3.7cm}p{3.7cm}}\toprule
\multicolumn{2}{l}{\large \textbf{ Summit to Base Network } }\label{netsb}
& \multicolumn{2}{l}{(product in: Networks)}
\\ \hline
\textbf{\footnotesize Manager} & \textbf{\footnotesize Owner} &
\textbf{\footnotesize WBS} & \textbf{\footnotesize Team} \\ \hline
Jeff Kantor &
\begin{tabular}{@{}l@{}}
Jeff Kantor \\
\end{tabular} & \begin{tabular}{@{}l@{}}
1.02C.08.03 \\
\end{tabular} & \begin{tabular}{@{}l@{}}
Net/Base \\
\end{tabular} \\ \hline
\multicolumn{4}{c}{
{\footnotesize ( Short Name: Sum/Base Net - Acronym: NETSB ) }
}\\ \hline
\end{longtable}

  La Serena - AURA Gatehouse Network



\begin{longtable}{p{3.7cm}p{3.7cm}p{3.7cm}p{3.7cm}}\hline
\textbf{\footnotesize Depends on:}  & & \textbf{\footnotesize Used in:} & \\ \hline
&
\\ \bottomrule
\end{longtable}

\subsection{Reference Data}\label{refd}
\begin{longtable}{p{3.7cm}p{3.7cm}p{3.7cm}p{3.7cm}}\hline
\textbf{Manager} & \textbf{Owner} & \textbf{WBS} & \textbf{Team} \\ \hline
 &
\begin{tabular}{@{}l@{}}
 \\
\end{tabular} & \begin{tabular}{@{}l@{}}
 \\
\end{tabular} & \begin{tabular}{@{}l@{}}
 \\
\end{tabular} \\ \hline
\multicolumn{4}{c}{
{\footnotesize ( Short Name: Ref Data - Acronym: REFD ) }
}\\ \hline
\end{longtable}

  DM reference datasets to be used during operations, and in preparation,
during constructions..



\begin{longtable}{p{3.7cm}p{3.7cm}p{3.7cm}p{3.7cm}}\toprule
\multicolumn{2}{l}{\large \textbf{ Atronomy.net Data } }\label{and}
& \multicolumn{2}{l}{(product in: Ref Data)}
\\ \hline
\textbf{\footnotesize Manager} & \textbf{\footnotesize Owner} &
\textbf{\footnotesize WBS} & \textbf{\footnotesize Team} \\ \hline
 &
\begin{tabular}{@{}l@{}}
 \\
\end{tabular} & \begin{tabular}{@{}l@{}}
 \\
\end{tabular} & \begin{tabular}{@{}l@{}}
 \\
\end{tabular} \\ \hline
\multicolumn{4}{c}{
{\footnotesize ( Short Name: Astronomy.net Data - Acronym: AND ) }
}\\ \hline
\end{longtable}

  



\begin{longtable}{p{3.7cm}p{3.7cm}p{3.7cm}p{3.7cm}}\hline
\multicolumn{2}{r}{\textbf{GutHub Packages:}} &
\multicolumn{2}{l}{\href{https://github.com/lsst/astronomy_net_data}{astronomy\_net\_data} }
\\ \hline \\ \hline
\textbf{\footnotesize Depends on:}  & & \textbf{\footnotesize Used in:} & \\ \hline
&
\multicolumn{2}{c}{
\begin{tabular}{c}
\hyperref[spdist]{Science Pipelines Distribution} \\ \hline
\end{tabular} }
\\ \bottomrule
\end{longtable}

\begin{longtable}{p{3.7cm}p{3.7cm}p{3.7cm}p{3.7cm}}\toprule
\multicolumn{2}{l}{\large \textbf{ Gaia Data } }\label{gdata}
& \multicolumn{2}{l}{(product in: Ref Data)}
\\ \hline
\textbf{\footnotesize Manager} & \textbf{\footnotesize Owner} &
\textbf{\footnotesize WBS} & \textbf{\footnotesize Team} \\ \hline
 &
\begin{tabular}{@{}l@{}}
 \\
\end{tabular} & \begin{tabular}{@{}l@{}}
 \\
\end{tabular} & \begin{tabular}{@{}l@{}}
 \\
\end{tabular} \\ \hline
\multicolumn{4}{c}{
{\footnotesize ( Short Name: Gaia Data - Acronym: GDATA ) }
}\\ \hline
\end{longtable}

  



\begin{longtable}{p{3.7cm}p{3.7cm}p{3.7cm}p{3.7cm}}\hline
\textbf{\footnotesize Depends on:}  & & \textbf{\footnotesize Used in:} & \\ \hline
&
\\ \bottomrule
\end{longtable}





\newpage
\section{Supporting products}\label{sec:sups}

This section will list all DM products that are not involved in the operational data processing activities. 
However these products have an important role for construction and maintenance.

These products are organized and maintained in MagicDraw by the DM System Engineering group.
\newpage
\subsection{DM Development and Maintenance products}
% auto generated from multiple sources - DO NOT EDIT!
% using template at <template>.
% Collecting data for component: ""
% using docsteady version 
%
% This file is meant to be included in LaTeX document in order to provide:
%   -  MagicDraw Top Level Product Tree (section 2)

\begin{longtable}{p{3.7cm}p{3.7cm}p{3.7cm}p{3.7cm}}\hline
\textbf{Manager} & \textbf{Owner} & \textbf{WBS} & \textbf{Team} \\ \hline
\parbox{3.5cm}{

\vspace{2mm}%
} &
\begin{tabular}{@{}l@{}}
\parbox{3.5cm}{

\vspace{2mm}%
} \\
\end{tabular}
 &
\begin{tabular}{@{}l@{}}
 \\
\end{tabular} &
\begin{tabular}{@{}l@{}}
 \\
\end{tabular} \\ \hline
\multicolumn{4}{c}{
{\footnotesize ( DM Support Prd.
 - DMDEV ) }
}\\ \hline
\end{longtable}

  


\newpage
\subsection{DevM Services}\label{dmdsrv}
\begin{longtable}{p{3.7cm}p{3.7cm}p{3.7cm}p{3.7cm}}\hline
\textbf{Manager} & \textbf{Owner} & \textbf{WBS} & \textbf{Team} \\ \hline
\parbox{3.5cm}{

\vspace{2mm}%
} &
\begin{tabular}{@{}l@{}}
\parbox{3.5cm}{

\vspace{2mm}%
} \\
\end{tabular}
 &
\begin{tabular}{@{}l@{}}
 \\
\end{tabular} &
\begin{tabular}{@{}l@{}}
 \\
\end{tabular} \\ \hline
\multicolumn{4}{c}{
{\footnotesize ( Support Services
 - DMDSRV ) }
}\\ \hline
\end{longtable}

  


    
\newpage
\begin{longtable}{p{3.7cm}p{3.7cm}p{3.7cm}p{3.7cm}}\toprule
\multicolumn{2}{l}{\large \textbf{ Build/ CI  } }\label{bci}
& \multicolumn{2}{l}{(product in: Build and CI
)}
\\ \hline
\textbf{\footnotesize Manager} & \textbf{\footnotesize Owner} &
\textbf{\footnotesize WBS} & \textbf{\footnotesize Team} \\ \hline
\parbox{3.5cm}{

\vspace{2mm}%
} &
\begin{tabular}{@{}l@{}}
\parbox{3.5cm}{

\vspace{2mm}%
} \\
\end{tabular} &
\begin{tabular}{@{}l@{}}
 \\
\end{tabular} & \begin{tabular}{@{}l@{}}
 \\
\end{tabular} \\ \hline
\multicolumn{4}{c}{
{\footnotesize ( Build and CI
 - BCI ) }
}\\ \hline
\end{longtable}

  Build and Continuous Integration services, This service is running the
compilation and the unit test continuously, in order to ensure that
latest changes introduced in the repository do not affect the
compilation nor the functionality. Continuous Integration is not only
exercise in a single software package, but ensure that all packages
together are still building and provides nightly and weekly a working
build as tags in the git repository and as EUPS distribution packages.
Documents are build continuously using the same approach.


\begin{longtable}{p{3.7cm}p{3.7cm}p{3.7cm}p{3.7cm}}\hline
\textbf{\footnotesize Uses:}  & & \textbf{\footnotesize Used in:} & \\ \hline
\multicolumn{2}{c}{
} &
\multicolumn{2}{c}{
} \\ \bottomrule
\end{longtable}

       
\newpage
\begin{longtable}{p{3.7cm}p{3.7cm}p{3.7cm}p{3.7cm}}\toprule
\multicolumn{2}{l}{\large \textbf{ Containerized Application Management } }\label{cam}
& \multicolumn{2}{l}{(product in: Cont. Application Mng
)}
\\ \hline
\textbf{\footnotesize Manager} & \textbf{\footnotesize Owner} &
\textbf{\footnotesize WBS} & \textbf{\footnotesize Team} \\ \hline
\parbox{3.5cm}{

\vspace{2mm}%
} &
\begin{tabular}{@{}l@{}}
\parbox{3.5cm}{

\vspace{2mm}%
} \\
\end{tabular} &
\begin{tabular}{@{}l@{}}
 \\
\end{tabular} & \begin{tabular}{@{}l@{}}
 \\
\end{tabular} \\ \hline
\multicolumn{4}{c}{
{\footnotesize ( Cont. Application Mng
 - CAM ) }
}\\ \hline
\end{longtable}

  To be described.


\begin{longtable}{p{3.7cm}p{3.7cm}p{3.7cm}p{3.7cm}}\hline
\textbf{\footnotesize Uses:}  & & \textbf{\footnotesize Used in:} & \\ \hline
\multicolumn{2}{c}{
\begin{tabular}{c}
\hyperref[k8s]{Kubernetes} \\ \hline
\hyperref[docker]{Docker} \\ \hline
\end{tabular}
} &
\multicolumn{2}{c}{
} \\ \bottomrule
\end{longtable}

       
\newpage
\begin{longtable}{p{3.7cm}p{3.7cm}p{3.7cm}p{3.7cm}}\toprule
\multicolumn{2}{l}{\large \textbf{ Developer Communication Tools } }\label{dmdcom}
& \multicolumn{2}{l}{(product in: Devlop. Comm
)}
\\ \hline
\textbf{\footnotesize Manager} & \textbf{\footnotesize Owner} &
\textbf{\footnotesize WBS} & \textbf{\footnotesize Team} \\ \hline
\parbox{3.5cm}{

\vspace{2mm}%
} &
\begin{tabular}{@{}l@{}}
\parbox{3.5cm}{

\vspace{2mm}%
} \\
\end{tabular} &
\begin{tabular}{@{}l@{}}
 \\
\end{tabular} & \begin{tabular}{@{}l@{}}
 \\
\end{tabular} \\ \hline
\multicolumn{4}{c}{
{\footnotesize ( Devlop. Comm
 - DMDCOM ) }
}\\ \hline
\end{longtable}

  Developer Communication Service (SLACK?)


\begin{longtable}{p{3.7cm}p{3.7cm}p{3.7cm}p{3.7cm}}\hline
\textbf{\footnotesize Uses:}  & & \textbf{\footnotesize Used in:} & \\ \hline
\multicolumn{2}{c}{
} &
\multicolumn{2}{c}{
} \\ \bottomrule
\end{longtable}

       
\newpage
\begin{longtable}{p{3.7cm}p{3.7cm}p{3.7cm}p{3.7cm}}\toprule
\multicolumn{2}{l}{\large \textbf{ Developer Services } }\label{ddvsrv}
& \multicolumn{2}{l}{(product in: DM Dev Services
)}
\\ \hline
\textbf{\footnotesize Manager} & \textbf{\footnotesize Owner} &
\textbf{\footnotesize WBS} & \textbf{\footnotesize Team} \\ \hline
\parbox{3.5cm}{

\vspace{2mm}%
} &
\begin{tabular}{@{}l@{}}
\parbox{3.5cm}{

\vspace{2mm}%
} \\
\end{tabular} &
\begin{tabular}{@{}l@{}}
 \\
\end{tabular} & \begin{tabular}{@{}l@{}}
 \\
\end{tabular} \\ \hline
\multicolumn{4}{c}{
{\footnotesize ( DM Dev Services
 - DDVSRV ) }
}\\ \hline
\end{longtable}

  This service includes all required at NCSA in order to develop and
maintain the DMS components.


\begin{longtable}{p{3.7cm}p{3.7cm}p{3.7cm}p{3.7cm}}\hline
\textbf{\footnotesize Uses:}  & & \textbf{\footnotesize Used in:} & \\ \hline
\multicolumn{2}{c}{
} &
\multicolumn{2}{c}{
\begin{tabular}{c}
\hyperref[]{Development and Integration E.} \\ \hline
\end{tabular}
} \\ \bottomrule
\end{longtable}

       
\newpage
\begin{longtable}{p{3.7cm}p{3.7cm}p{3.7cm}p{3.7cm}}\toprule
\multicolumn{2}{l}{\large \textbf{ Documentation Publication } }\label{ddcpub}
& \multicolumn{2}{l}{(product in: DM Doc Publication
)}
\\ \hline
\textbf{\footnotesize Manager} & \textbf{\footnotesize Owner} &
\textbf{\footnotesize WBS} & \textbf{\footnotesize Team} \\ \hline
\parbox{3.5cm}{

\vspace{2mm}%
} &
\begin{tabular}{@{}l@{}}
\parbox{3.5cm}{

\vspace{2mm}%
} \\
\end{tabular} &
\begin{tabular}{@{}l@{}}
 \\
\end{tabular} & \begin{tabular}{@{}l@{}}
 \\
\end{tabular} \\ \hline
\multicolumn{4}{c}{
{\footnotesize ( DM Doc Publication
 - DDCPUB ) }
}\\ \hline
\end{longtable}

  Documentation Publication service


\begin{longtable}{p{3.7cm}p{3.7cm}p{3.7cm}p{3.7cm}}\hline
\textbf{\footnotesize Uses:}  & & \textbf{\footnotesize Used in:} & \\ \hline
\multicolumn{2}{c}{
} &
\multicolumn{2}{c}{
} \\ \bottomrule
\end{longtable}

       
\newpage
\begin{longtable}{p{3.7cm}p{3.7cm}p{3.7cm}p{3.7cm}}\toprule
\multicolumn{2}{l}{\large \textbf{ Packaging/ Distribution } }\label{pkgdst}
& \multicolumn{2}{l}{(product in: Packagin Distrib
)}
\\ \hline
\textbf{\footnotesize Manager} & \textbf{\footnotesize Owner} &
\textbf{\footnotesize WBS} & \textbf{\footnotesize Team} \\ \hline
\parbox{3.5cm}{

\vspace{2mm}%
} &
\begin{tabular}{@{}l@{}}
\parbox{3.5cm}{

\vspace{2mm}%
} \\
\end{tabular} &
\begin{tabular}{@{}l@{}}
 \\
\end{tabular} & \begin{tabular}{@{}l@{}}
 \\
\end{tabular} \\ \hline
\multicolumn{4}{c}{
{\footnotesize ( Packagin Distrib
 - PKGDST ) }
}\\ \hline
\end{longtable}

  Packaging and Distribution Service Is not this service duplicated
(partially at least) with SW Deployment Srv.?


\begin{longtable}{p{3.7cm}p{3.7cm}p{3.7cm}p{3.7cm}}\hline
\textbf{\footnotesize Uses:}  & & \textbf{\footnotesize Used in:} & \\ \hline
\multicolumn{2}{c}{
} &
\multicolumn{2}{c}{
} \\ \bottomrule
\end{longtable}

       
\newpage
\begin{longtable}{p{3.7cm}p{3.7cm}p{3.7cm}p{3.7cm}}\toprule
\multicolumn{2}{l}{\large \textbf{ SW Deployment } }\label{deploy}
& \multicolumn{2}{l}{(product in: SW Deployment
)}
\\ \hline
\textbf{\footnotesize Manager} & \textbf{\footnotesize Owner} &
\textbf{\footnotesize WBS} & \textbf{\footnotesize Team} \\ \hline
\parbox{3.5cm}{

\vspace{2mm}%
} &
\begin{tabular}{@{}l@{}}
\parbox{3.5cm}{

\vspace{2mm}%
} \\
\end{tabular} &
\begin{tabular}{@{}l@{}}
 \\
\end{tabular} & \begin{tabular}{@{}l@{}}
 \\
\end{tabular} \\ \hline
\multicolumn{4}{c}{
{\footnotesize ( SW Deployment
 - DEPLOY ) }
}\\ \hline
\end{longtable}

  This service provide the ability of DMS to deploy the different software
packages in instantiated services in order to fulfill DMS objectives and
requirements. This imply for example the capability to deploy for the
general production DRP payload the corresponding science pipeline, with
the proper version and configuration in order to process the data and
provide the periodic data release. Different deployment strategies can
be identified depending of the type of software and service. 3rd party
software will be usually deployed manually.


\begin{longtable}{p{3.7cm}p{3.7cm}p{3.7cm}p{3.7cm}}\hline
\textbf{\footnotesize Uses:}  & & \textbf{\footnotesize Used in:} & \\ \hline
\multicolumn{2}{c}{
} &
\multicolumn{2}{c}{
} \\ \bottomrule
\end{longtable}

       
\newpage
\begin{longtable}{p{3.7cm}p{3.7cm}p{3.7cm}p{3.7cm}}\toprule
\multicolumn{2}{l}{\large \textbf{ SW Version Control } }\label{swver}
& \multicolumn{2}{l}{(product in: SW Version Control
)}
\\ \hline
\textbf{\footnotesize Manager} & \textbf{\footnotesize Owner} &
\textbf{\footnotesize WBS} & \textbf{\footnotesize Team} \\ \hline
\parbox{3.5cm}{

\vspace{2mm}%
} &
\begin{tabular}{@{}l@{}}
\parbox{3.5cm}{

\vspace{2mm}%
} \\
\end{tabular} &
\begin{tabular}{@{}l@{}}
 \\
\end{tabular} & \begin{tabular}{@{}l@{}}
 \\
\end{tabular} \\ \hline
\multicolumn{4}{c}{
{\footnotesize ( SW Version Control
 - SWVER ) }
}\\ \hline
\end{longtable}

  This service provides software release management in order to obtain
consistent releases for the execution of the main DMS services that will
provide the final data products required.


\begin{longtable}{p{3.7cm}p{3.7cm}p{3.7cm}p{3.7cm}}\hline
\textbf{\footnotesize Uses:}  & & \textbf{\footnotesize Used in:} & \\ \hline
\multicolumn{2}{c}{
} &
\multicolumn{2}{c}{
} \\ \bottomrule
\end{longtable}

     \newpage
\subsection{DevM SW Products}\label{dmdsw}
\begin{longtable}{p{3.7cm}p{3.7cm}p{3.7cm}p{3.7cm}}\hline
\textbf{Manager} & \textbf{Owner} & \textbf{WBS} & \textbf{Team} \\ \hline
\parbox{3.5cm}{

\vspace{2mm}%
} &
\begin{tabular}{@{}l@{}}
\parbox{3.5cm}{

\vspace{2mm}%
} \\
\end{tabular}
 &
\begin{tabular}{@{}l@{}}
 \\
\end{tabular} &
\begin{tabular}{@{}l@{}}
 \\
\end{tabular} \\ \hline
\multicolumn{4}{c}{
{\footnotesize ( Dm Dev Software
 - DMDSW ) }
}\\ \hline
\end{longtable}

  


\newpage
\subsubsection{DevM Supporting SW}\label{dmdss}
\begin{longtable}{p{3.7cm}p{3.7cm}p{3.7cm}p{3.7cm}}\hline
\textbf{Manager} & \textbf{Owner} & \textbf{WBS} & \textbf{Team} \\ \hline
\parbox{3.5cm}{

\vspace{2mm}%
} &
\begin{tabular}{@{}l@{}}
\parbox{3.5cm}{

\vspace{2mm}%
} \\
\end{tabular}
 &
\begin{tabular}{@{}l@{}}
 \\
\end{tabular} &
\begin{tabular}{@{}l@{}}
 \\
\end{tabular} \\ \hline
\multicolumn{4}{c}{
{\footnotesize ( DMDev Supp SW
 - DMDSS ) }
}\\ \hline
\end{longtable}

  


   
\newpage
\begin{longtable}{p{3.7cm}p{3.7cm}p{3.7cm}p{3.7cm}}\toprule
\multicolumn{2}{l}{\large \textbf{ codekit } }\label{cdkt}
& \multicolumn{2}{l}{(product in: Codekit
)}
\\ \hline
\textbf{\footnotesize Manager} & \textbf{\footnotesize Owner} &
\textbf{\footnotesize WBS} & \textbf{\footnotesize Team} \\ \hline
\parbox{3.5cm}{

\vspace{2mm}%
} &
\begin{tabular}{@{}l@{}}
\parbox{3.5cm}{

\vspace{2mm}%
} \\
\end{tabular} &
\begin{tabular}{@{}l@{}}
 \\
\end{tabular} & \begin{tabular}{@{}l@{}}
 \\
\end{tabular} \\ \hline
\multicolumn{4}{c}{
{\footnotesize ( Codekit
 - CDKT ) }
}\\ \hline
\end{longtable}

  


\begin{longtable}{p{3.7cm}p{3.7cm}p{3.7cm}p{3.7cm}}\hline
\multicolumn{2}{r}{\textbf{GutHub Packages:}} &
\multicolumn{2}{l}{\href{https://github.com/lsst/sqr_codekit}{sqr\_codekit} }
\\ \hline \\ \hline
\textbf{\footnotesize Uses:}  & & \textbf{\footnotesize Used in:} & \\ \hline
\multicolumn{2}{c}{
} &
\multicolumn{2}{c}{
} \\ \bottomrule
\end{longtable}

     
\newpage
\begin{longtable}{p{3.7cm}p{3.7cm}p{3.7cm}p{3.7cm}}\toprule
\multicolumn{2}{l}{\large \textbf{ lsst\_build } }\label{lbld}
& \multicolumn{2}{l}{(product in: lsst\_build
)}
\\ \hline
\textbf{\footnotesize Manager} & \textbf{\footnotesize Owner} &
\textbf{\footnotesize WBS} & \textbf{\footnotesize Team} \\ \hline
\parbox{3.5cm}{

\vspace{2mm}%
} &
\begin{tabular}{@{}l@{}}
\parbox{3.5cm}{

\vspace{2mm}%
} \\
\end{tabular} &
\begin{tabular}{@{}l@{}}
 \\
\end{tabular} & \begin{tabular}{@{}l@{}}
 \\
\end{tabular} \\ \hline
\multicolumn{4}{c}{
{\footnotesize ( lsst\_build
 - LBLD ) }
}\\ \hline
\end{longtable}

  Builder and Continuous Integration Tools for LSST.


\begin{longtable}{p{3.7cm}p{3.7cm}p{3.7cm}p{3.7cm}}\hline
\multicolumn{2}{r}{\textbf{GutHub Packages:}} &
\multicolumn{2}{l}{\href{https://github.com/lsst/lsst_build}{lsst\_build} }
\\ \hline \\ \hline
\textbf{References} &
\multicolumn{3}{l}{\href{https://github.com/lsst/lsst_build}{https://github.com/lsst/lsst\_build} }
\\ \hline \\ \hline
\textbf{\footnotesize Uses:}  & & \textbf{\footnotesize Used in:} & \\ \hline
\multicolumn{2}{c}{
} &
\multicolumn{2}{c}{
} \\ \bottomrule
\end{longtable}

     
\newpage
\begin{longtable}{p{3.7cm}p{3.7cm}p{3.7cm}p{3.7cm}}\toprule
\multicolumn{2}{l}{\large \textbf{ lsstsw } }\label{lsstsw}
& \multicolumn{2}{l}{(product in: lsstsw
)}
\\ \hline
\textbf{\footnotesize Manager} & \textbf{\footnotesize Owner} &
\textbf{\footnotesize WBS} & \textbf{\footnotesize Team} \\ \hline
\parbox{3.5cm}{

\vspace{2mm}%
} &
\begin{tabular}{@{}l@{}}
\parbox{3.5cm}{

\vspace{2mm}%
} \\
\end{tabular} &
\begin{tabular}{@{}l@{}}
 \\
\end{tabular} & \begin{tabular}{@{}l@{}}
 \\
\end{tabular} \\ \hline
\multicolumn{4}{c}{
{\footnotesize ( lsstsw
 - LSSTSW ) }
}\\ \hline
\end{longtable}

  


\begin{longtable}{p{3.7cm}p{3.7cm}p{3.7cm}p{3.7cm}}\hline
\multicolumn{2}{r}{\textbf{GutHub Packages:}} &
\multicolumn{2}{l}{\href{https://github.com/lsst/lsstsw}{lsstsw} }
\\ \hline \\ \hline
\textbf{\footnotesize Uses:}  & & \textbf{\footnotesize Used in:} & \\ \hline
\multicolumn{2}{c}{
} &
\multicolumn{2}{c}{
} \\ \bottomrule
\end{longtable}

    
   \newpage
\subsubsection{SciencePipelines SW}\label{dsplsw}
\begin{longtable}{p{3.7cm}p{3.7cm}p{3.7cm}p{3.7cm}}\hline
\textbf{Manager} & \textbf{Owner} & \textbf{WBS} & \textbf{Team} \\ \hline
\parbox{3.5cm}{

\vspace{2mm}%
} &
\begin{tabular}{@{}l@{}}
\parbox{3.5cm}{

\vspace{2mm}%
} \\
\end{tabular}
 &
\begin{tabular}{@{}l@{}}
 \\
\end{tabular} &
\begin{tabular}{@{}l@{}}
 \\
\end{tabular} \\ \hline
\multicolumn{4}{c}{
{\footnotesize ( Dev Science Pipelines
 - DSPLSW ) }
}\\ \hline
\end{longtable}

  


   
\newpage
\begin{longtable}{p{3.7cm}p{3.7cm}p{3.7cm}p{3.7cm}}\toprule
\multicolumn{2}{l}{\large \textbf{ SPL Workflow } }\label{splwf}
& \multicolumn{2}{l}{(product in: SPL Workflow
)}
\\ \hline
\textbf{\footnotesize Manager} & \textbf{\footnotesize Owner} &
\textbf{\footnotesize WBS} & \textbf{\footnotesize Team} \\ \hline
\parbox{3.5cm}{

\vspace{2mm}%
} &
\begin{tabular}{@{}l@{}}
\parbox{3.5cm}{

\vspace{2mm}%
} \\
\end{tabular} &
\begin{tabular}{@{}l@{}}
 \\
\end{tabular} & \begin{tabular}{@{}l@{}}
 \\
\end{tabular} \\ \hline
\multicolumn{4}{c}{
{\footnotesize ( SPL Workflow
 - SPLWF ) }
}\\ \hline
\end{longtable}

  


\begin{longtable}{p{3.7cm}p{3.7cm}p{3.7cm}p{3.7cm}}\hline
\multicolumn{2}{r}{\textbf{GutHub Packages:}} &
\multicolumn{2}{l}{\href{https://github.com/lsst/spl_workflow}{spl\_workflow} }
\\ \hline \\ \hline
\textbf{\footnotesize Uses:}  & & \textbf{\footnotesize Used in:} & \\ \hline
\multicolumn{2}{c}{
} &
\multicolumn{2}{c}{
} \\ \bottomrule
\end{longtable}

    
     \newpage
\subsection{DevM Test Data}\label{dmdtd}
\begin{longtable}{p{3.7cm}p{3.7cm}p{3.7cm}p{3.7cm}}\hline
\textbf{Manager} & \textbf{Owner} & \textbf{WBS} & \textbf{Team} \\ \hline
\parbox{3.5cm}{

\vspace{2mm}%
} &
\begin{tabular}{@{}l@{}}
\parbox{3.5cm}{

\vspace{2mm}%
} \\
\end{tabular}
 &
\begin{tabular}{@{}l@{}}
 \\
\end{tabular} &
\begin{tabular}{@{}l@{}}
 \\
\end{tabular} \\ \hline
\multicolumn{4}{c}{
{\footnotesize ( DMDev Test Data
 - DMDTD ) }
}\\ \hline
\end{longtable}

  


    
\newpage
\begin{longtable}{p{3.7cm}p{3.7cm}p{3.7cm}p{3.7cm}}\toprule
\multicolumn{2}{l}{\large \textbf{ HSC-RC1 } }\label{hscrc1}
& \multicolumn{2}{l}{(product in: HSC-RC1
)}
\\ \hline
\textbf{\footnotesize Manager} & \textbf{\footnotesize Owner} &
\textbf{\footnotesize WBS} & \textbf{\footnotesize Team} \\ \hline
\parbox{3.5cm}{

\vspace{2mm}%
} &
\begin{tabular}{@{}l@{}}
\parbox{3.5cm}{

\vspace{2mm}%
} \\
\end{tabular} &
\begin{tabular}{@{}l@{}}
 \\
\end{tabular} & \begin{tabular}{@{}l@{}}
 \\
\end{tabular} \\ \hline
\multicolumn{4}{c}{
{\footnotesize ( HSC-RC1
 - HSCRC1 ) }
}\\ \hline
\end{longtable}

  Hyper Suprime-Cam ``RC1'' This is an example of Dataset that can be
included in this package. To be better characterized.


\begin{longtable}{p{3.7cm}p{3.7cm}p{3.7cm}p{3.7cm}}\hline
\textbf{\footnotesize Uses:}  & & \textbf{\footnotesize Used in:} & \\ \hline
\multicolumn{2}{c}{
} &
\multicolumn{2}{c}{
} \\ \bottomrule
\end{longtable}

       
\newpage
\begin{longtable}{p{3.7cm}p{3.7cm}p{3.7cm}p{3.7cm}}\toprule
\multicolumn{2}{l}{\large \textbf{ Science Pipelines Test Data } }\label{spltd}
& \multicolumn{2}{l}{(product in: SPL Test Data
)}
\\ \hline
\textbf{\footnotesize Manager} & \textbf{\footnotesize Owner} &
\textbf{\footnotesize WBS} & \textbf{\footnotesize Team} \\ \hline
\parbox{3.5cm}{

\vspace{2mm}%
} &
\begin{tabular}{@{}l@{}}
\parbox{3.5cm}{

\vspace{2mm}%
} \\
\end{tabular} &
\begin{tabular}{@{}l@{}}
 \\
\end{tabular} & \begin{tabular}{@{}l@{}}
 \\
\end{tabular} \\ \hline
\multicolumn{4}{c}{
{\footnotesize ( SPL Test Data
 - SPLTD ) }
}\\ \hline
\end{longtable}

  


\begin{longtable}{p{3.7cm}p{3.7cm}p{3.7cm}p{3.7cm}}\hline
\multicolumn{2}{r}{\textbf{GutHub Packages:}} &
\multicolumn{2}{l}{\href{https://github.com/lsst/spl_testdata}{spl\_testdata} }
\\ \hline \\ \hline
\textbf{\footnotesize Uses:}  & & \textbf{\footnotesize Used in:} & \\ \hline
\multicolumn{2}{c}{
} &
\multicolumn{2}{c}{
} \\ \bottomrule
\end{longtable}

     \newpage
\subsection{DevM COTS 3rdPLs ENVs}\label{dmdc3e}
\begin{longtable}{p{3.7cm}p{3.7cm}p{3.7cm}p{3.7cm}}\hline
\textbf{Manager} & \textbf{Owner} & \textbf{WBS} & \textbf{Team} \\ \hline
\parbox{3.5cm}{

\vspace{2mm}%
} &
\begin{tabular}{@{}l@{}}
\parbox{3.5cm}{

\vspace{2mm}%
} \\
\end{tabular}
 &
\begin{tabular}{@{}l@{}}
 \\
\end{tabular} &
\begin{tabular}{@{}l@{}}
 \\
\end{tabular} \\ \hline
\multicolumn{4}{c}{
{\footnotesize ( COTS 3rdPLs ENVs
 - DMDC3E ) }
}\\ \hline
\end{longtable}

  


\newpage
\subsubsection{DevM COTS}\label{cotsdm}
\begin{longtable}{p{3.7cm}p{3.7cm}p{3.7cm}p{3.7cm}}\hline
\textbf{Manager} & \textbf{Owner} & \textbf{WBS} & \textbf{Team} \\ \hline
\parbox{3.5cm}{

\vspace{2mm}%
} &
\begin{tabular}{@{}l@{}}
\parbox{3.5cm}{

\vspace{2mm}%
} \\
\end{tabular}
 &
\begin{tabular}{@{}l@{}}
 \\
\end{tabular} &
\begin{tabular}{@{}l@{}}
 \\
\end{tabular} \\ \hline
\multicolumn{4}{c}{
{\footnotesize ( DM DevM COTS
 - COTSDM ) }
}\\ \hline
\end{longtable}

  


   
\newpage
\begin{longtable}{p{3.7cm}p{3.7cm}p{3.7cm}p{3.7cm}}\toprule
\multicolumn{2}{l}{\large \textbf{ EUPS } }\label{eups}
& \multicolumn{2}{l}{(product in: EUPS
)}
\\ \hline
\textbf{\footnotesize Manager} & \textbf{\footnotesize Owner} &
\textbf{\footnotesize WBS} & \textbf{\footnotesize Team} \\ \hline
\parbox{3.5cm}{

\vspace{2mm}%
} &
\begin{tabular}{@{}l@{}}
\parbox{3.5cm}{

\vspace{2mm}%
} \\
\end{tabular} &
\begin{tabular}{@{}l@{}}
 \\
\end{tabular} & \begin{tabular}{@{}l@{}}
 \\
\end{tabular} \\ \hline
\multicolumn{4}{c}{
{\footnotesize ( EUPS
 - EUPS ) }
}\\ \hline
\end{longtable}

  


\begin{longtable}{p{3.7cm}p{3.7cm}p{3.7cm}p{3.7cm}}\hline
\textbf{\footnotesize Uses:}  & & \textbf{\footnotesize Used in:} & \\ \hline
\multicolumn{2}{c}{
} &
\multicolumn{2}{c}{
} \\ \bottomrule
\end{longtable}

     
\newpage
\begin{longtable}{p{3.7cm}p{3.7cm}p{3.7cm}p{3.7cm}}\toprule
\multicolumn{2}{l}{\large \textbf{ GCC } }\label{gcc}
& \multicolumn{2}{l}{(product in: GCC
)}
\\ \hline
\textbf{\footnotesize Manager} & \textbf{\footnotesize Owner} &
\textbf{\footnotesize WBS} & \textbf{\footnotesize Team} \\ \hline
\parbox{3.5cm}{

\vspace{2mm}%
} &
\begin{tabular}{@{}l@{}}
\parbox{3.5cm}{

\vspace{2mm}%
} \\
\end{tabular} &
\begin{tabular}{@{}l@{}}
 \\
\end{tabular} & \begin{tabular}{@{}l@{}}
 \\
\end{tabular} \\ \hline
\multicolumn{4}{c}{
{\footnotesize ( GCC
 - GCC ) }
}\\ \hline
\end{longtable}

  


\begin{longtable}{p{3.7cm}p{3.7cm}p{3.7cm}p{3.7cm}}\hline
\textbf{\footnotesize Uses:}  & & \textbf{\footnotesize Used in:} & \\ \hline
\multicolumn{2}{c}{
} &
\multicolumn{2}{c}{
} \\ \bottomrule
\end{longtable}

     
\newpage
\begin{longtable}{p{3.7cm}p{3.7cm}p{3.7cm}p{3.7cm}}\toprule
\multicolumn{2}{l}{\large \textbf{ Github } }\label{github}
& \multicolumn{2}{l}{(product in: Github
)}
\\ \hline
\textbf{\footnotesize Manager} & \textbf{\footnotesize Owner} &
\textbf{\footnotesize WBS} & \textbf{\footnotesize Team} \\ \hline
\parbox{3.5cm}{

\vspace{2mm}%
} &
\begin{tabular}{@{}l@{}}
\parbox{3.5cm}{

\vspace{2mm}%
} \\
\end{tabular} &
\begin{tabular}{@{}l@{}}
 \\
\end{tabular} & \begin{tabular}{@{}l@{}}
 \\
\end{tabular} \\ \hline
\multicolumn{4}{c}{
{\footnotesize ( Github
 - GITHUB ) }
}\\ \hline
\end{longtable}

  


\begin{longtable}{p{3.7cm}p{3.7cm}p{3.7cm}p{3.7cm}}\hline
\textbf{\footnotesize Uses:}  & & \textbf{\footnotesize Used in:} & \\ \hline
\multicolumn{2}{c}{
} &
\multicolumn{2}{c}{
} \\ \bottomrule
\end{longtable}

     
\newpage
\begin{longtable}{p{3.7cm}p{3.7cm}p{3.7cm}p{3.7cm}}\toprule
\multicolumn{2}{l}{\large \textbf{ Jenkins } }\label{jnkns}
& \multicolumn{2}{l}{(product in: Jenkins
)}
\\ \hline
\textbf{\footnotesize Manager} & \textbf{\footnotesize Owner} &
\textbf{\footnotesize WBS} & \textbf{\footnotesize Team} \\ \hline
\parbox{3.5cm}{

\vspace{2mm}%
} &
\begin{tabular}{@{}l@{}}
\parbox{3.5cm}{

\vspace{2mm}%
} \\
\end{tabular} &
\begin{tabular}{@{}l@{}}
 \\
\end{tabular} & \begin{tabular}{@{}l@{}}
 \\
\end{tabular} \\ \hline
\multicolumn{4}{c}{
{\footnotesize ( Jenkins
 - JNKNS ) }
}\\ \hline
\end{longtable}

  


\begin{longtable}{p{3.7cm}p{3.7cm}p{3.7cm}p{3.7cm}}\hline
\textbf{\footnotesize Uses:}  & & \textbf{\footnotesize Used in:} & \\ \hline
\multicolumn{2}{c}{
} &
\multicolumn{2}{c}{
} \\ \bottomrule
\end{longtable}

     
\newpage
\begin{longtable}{p{3.7cm}p{3.7cm}p{3.7cm}p{3.7cm}}\toprule
\multicolumn{2}{l}{\large \textbf{ JIRA } }\label{jra}
& \multicolumn{2}{l}{(product in: Jira
)}
\\ \hline
\textbf{\footnotesize Manager} & \textbf{\footnotesize Owner} &
\textbf{\footnotesize WBS} & \textbf{\footnotesize Team} \\ \hline
\parbox{3.5cm}{

\vspace{2mm}%
} &
\begin{tabular}{@{}l@{}}
\parbox{3.5cm}{

\vspace{2mm}%
} \\
\end{tabular} &
\begin{tabular}{@{}l@{}}
 \\
\end{tabular} & \begin{tabular}{@{}l@{}}
 \\
\end{tabular} \\ \hline
\multicolumn{4}{c}{
{\footnotesize ( Jira
 - JRA ) }
}\\ \hline
\end{longtable}

  


\begin{longtable}{p{3.7cm}p{3.7cm}p{3.7cm}p{3.7cm}}\hline
\textbf{\footnotesize Uses:}  & & \textbf{\footnotesize Used in:} & \\ \hline
\multicolumn{2}{c}{
} &
\multicolumn{2}{c}{
\begin{tabular}{c}
\hyperref[]{Issue Tracking} \\ \hline
\end{tabular}
} \\ \bottomrule
\end{longtable}

     
\newpage
\begin{longtable}{p{3.7cm}p{3.7cm}p{3.7cm}p{3.7cm}}\toprule
\multicolumn{2}{l}{\large \textbf{ Python } }\label{pthn}
& \multicolumn{2}{l}{(product in: Python
)}
\\ \hline
\textbf{\footnotesize Manager} & \textbf{\footnotesize Owner} &
\textbf{\footnotesize WBS} & \textbf{\footnotesize Team} \\ \hline
\parbox{3.5cm}{

\vspace{2mm}%
} &
\begin{tabular}{@{}l@{}}
\parbox{3.5cm}{

\vspace{2mm}%
} \\
\end{tabular} &
\begin{tabular}{@{}l@{}}
 \\
\end{tabular} & \begin{tabular}{@{}l@{}}
 \\
\end{tabular} \\ \hline
\multicolumn{4}{c}{
{\footnotesize ( Python
 - PTHN ) }
}\\ \hline
\end{longtable}

  


\begin{longtable}{p{3.7cm}p{3.7cm}p{3.7cm}p{3.7cm}}\hline
\textbf{\footnotesize Uses:}  & & \textbf{\footnotesize Used in:} & \\ \hline
\multicolumn{2}{c}{
} &
\multicolumn{2}{c}{
} \\ \bottomrule
\end{longtable}

     
\newpage
\begin{longtable}{p{3.7cm}p{3.7cm}p{3.7cm}p{3.7cm}}\toprule
\multicolumn{2}{l}{\large \textbf{ Travis } }\label{trvs}
& \multicolumn{2}{l}{(product in: Travis
)}
\\ \hline
\textbf{\footnotesize Manager} & \textbf{\footnotesize Owner} &
\textbf{\footnotesize WBS} & \textbf{\footnotesize Team} \\ \hline
\parbox{3.5cm}{

\vspace{2mm}%
} &
\begin{tabular}{@{}l@{}}
\parbox{3.5cm}{

\vspace{2mm}%
} \\
\end{tabular} &
\begin{tabular}{@{}l@{}}
 \\
\end{tabular} & \begin{tabular}{@{}l@{}}
 \\
\end{tabular} \\ \hline
\multicolumn{4}{c}{
{\footnotesize ( Travis
 - TRVS ) }
}\\ \hline
\end{longtable}

  


\begin{longtable}{p{3.7cm}p{3.7cm}p{3.7cm}p{3.7cm}}\hline
\textbf{\footnotesize Uses:}  & & \textbf{\footnotesize Used in:} & \\ \hline
\multicolumn{2}{c}{
} &
\multicolumn{2}{c}{
} \\ \bottomrule
\end{longtable}

    
   \newpage
\subsubsection{DevM Environments}\label{denv}
\begin{longtable}{p{3.7cm}p{3.7cm}p{3.7cm}p{3.7cm}}\hline
\textbf{Manager} & \textbf{Owner} & \textbf{WBS} & \textbf{Team} \\ \hline
\parbox{3.5cm}{

\vspace{2mm}%
} &
\begin{tabular}{@{}l@{}}
\parbox{3.5cm}{

\vspace{2mm}%
} \\
\end{tabular}
 &
\begin{tabular}{@{}l@{}}
 \\
\end{tabular} &
\begin{tabular}{@{}l@{}}
 \\
\end{tabular} \\ \hline
\multicolumn{4}{c}{
{\footnotesize ( DM DevM Envs
 - DENV ) }
}\\ \hline
\end{longtable}

  


   
\newpage
\begin{longtable}{p{3.7cm}p{3.7cm}p{3.7cm}p{3.7cm}}\toprule
\multicolumn{2}{l}{\large \textbf{ SciencePipelines Conda Env. } }\label{splce}
& \multicolumn{2}{l}{(product in: SPL Conda Env
)}
\\ \hline
\textbf{\footnotesize Manager} & \textbf{\footnotesize Owner} &
\textbf{\footnotesize WBS} & \textbf{\footnotesize Team} \\ \hline
\parbox{3.5cm}{

\vspace{2mm}%
} &
\begin{tabular}{@{}l@{}}
\parbox{3.5cm}{

\vspace{2mm}%
} \\
\end{tabular} &
\begin{tabular}{@{}l@{}}
 \\
\end{tabular} & \begin{tabular}{@{}l@{}}
 \\
\end{tabular} \\ \hline
\multicolumn{4}{c}{
{\footnotesize ( SPL Conda Env
 - SPLCE ) }
}\\ \hline
\end{longtable}

  


\begin{longtable}{p{3.7cm}p{3.7cm}p{3.7cm}p{3.7cm}}\hline
\multicolumn{2}{r}{\textbf{GutHub Packages:}} &
\multicolumn{2}{l}{\href{https://github.com/lsst/scipipe_conda_env}{scipipe\_conda\_env} }
\\ \hline \\ \hline
\textbf{\footnotesize Uses:}  & & \textbf{\footnotesize Used in:} & \\ \hline
\multicolumn{2}{c}{
} &
\multicolumn{2}{c}{
} \\ \bottomrule
\end{longtable}

    
         



\newpage
\section{GitHub Packages}\label{sec:low}

This section lists the GitHub packages related to the DM products listed in previous sections \ref{sec:top} and \ref{sec:sups}.
The detailed information is extracted from GitHub.

The information provided includes, when available, the list of dependencies tracted from the ups table file.
Note that the release procedure as described in \citeds{DMTN-106} 
can be applied to a software product only if all dependencies 
are not used in other software products. If this is not the case, only one of these software products can be released.

As it can be evinced by a quick inspection in the following subsections, 
all Science Pipelines software products share a large number of dependencies. 
Therefore, the only releasebale software product, at the time of writing (July 2020) 
is the Science Pipeline distribution.


% do not edit, generated automatically from Github

\newpage
\subsection{nublado}\label{lsst-sqre/nublado}

JupyterLab + JupyterHub + k8s deployment used by LSST for its Science
Platform


{\footnotesize
\begin{longtable}{rl}
\hline
Open it in GitHUb: & \href{https://github.com/lsst-sqre/nublado}{https://github.com/lsst-sqre/nublado} \\ \cdashline{1-2}
Top Level Component: & \hyperlink{nblsrv}{LSP Nublado} \\
\hline
\end{longtable} }


README.md (First 20 lines only)
{\scriptsize
\begin{lstlisting}[breaklines]
# LSST Science Platform Notebook Aspect

## Do Not Use This

If what you want to do is simply deploy a Jupyter setup under Kubernetes
you're much better off
using
[Zero to JupyterHub](https://zero-to-jupyterhub.readthedocs.io/en/latest/),
which is an excellent general tutorial for setting up
JupyterHub in a Kubernetes environment.

This cluster is much more specifically tailored to the needs
of [LSST](https://lsst.org).  If you want an example of how to set up
persistent storage for your users, how to ship logs to a remote ELK
stack, a worked example of how to subclass a spawner, or how to use an
image-spawner options menu, you may find it useful.

## Overview

The LSST Science Platform Notebook Aspect is a JupyterHub + JupyterLab
\end{lstlisting}
}



\newpage
\subsection{lsst-tap-service}\label{lsst-sqre/lsst-tap-service}

IVOA TAP service for LSST


{\footnotesize
\begin{longtable}{rl}
\hline
Open it in GitHUb: & \href{https://github.com/lsst-sqre/lsst-tap-service}{https://github.com/lsst-sqre/lsst-tap-service} \\ \cdashline{1-2}
Top Level Component: & \hyperlink{tapsrv}{TAP API} \\
\hline
\end{longtable} }


README.md (First 20 lines only)
{\scriptsize
\begin{lstlisting}[breaklines]
# LSST TAP Service

This repository contains the LSST TAP service.  It is based on the CADC TAP service
code and uses this as a dependency, and then adds special logic to work with QServ.

## Build

Run ./build.sh

## Deployment

### Docker
After the [Build](#build) step above, a set of containers with the `dev` tag will exist
on your local machine.  Then when you run:

`docker-compose up -d && ./waitForContainersReady.sh && ./checkAvailability.sh`

This should start a local group of containers, wait for them to be ready, and then
check that the availability endpoint returns a 200 and a simple sync query works.
This validates that your local TAP implementation is working.  You can now either
\end{lstlisting}
}



\newpage
\subsection{davt}\label{davt}

WebDAV with substitute user impersonation per-request


{\footnotesize
\begin{longtable}{rl}
\hline
Open it in GitHUb: & \href{https://github.com/lsst/davt}{https://github.com/lsst/davt} \\ \cdashline{1-2}
Top Level Component: & \hyperlink{wdavsrv}{WebDAV API} \\
\cdashline{1-2}
{GitHub Teams:} &
 Overlords \\
 & Data Management \\
 & Database \\
\hline
\end{longtable} }


README.md (First 20 lines only)
{\scriptsize
\begin{lstlisting}[breaklines]
# davt

`davt` is a lua module for nginx to aid with impersonation. Its target use case is for use with 
WebDAV, so that all operations are executed _as the user in the request_. 

For every incoming request, davt enables nginx to switch the OS user (with `setfsuid`) and/or 
group IDs/supplementary group IDs (via `setfsgid`, `setgroups`, `initgroups`) to match the 
authenticated user or specific groups before performing any file opertions.

As davt enables impersonation, a few properties follow:

* The files do NOT need to be owned by an nginx service account user, nor does an ACL need to be 
modified to allow for access to an service group (for filesystems supporting ACLs). This allows 
you to transperently operate the service over existing directories.

* Ownership when creating files is preserved for the files in question. This ensures that files 
created for the user via WebDAV are also readable when the user is in a shell, for example.

## Requirements
davt requires ljsyscall. It also used the ffi library from LuaJIT.
\end{lstlisting}
}



\newpage
\subsection{HeaderService}\label{lsst-dm/headerservice}

LSST Meta-data aggregator for FITS header service


{\footnotesize
\begin{longtable}{rl}
\hline
Open it in GitHUb: & \href{https://github.com/lsst-dm/HeaderService}{https://github.com/lsst-dm/HeaderService} \\ \cdashline{1-2}
Top Level Component: & \hyperlink{header}{Header Service SW} \\
\hline
\end{longtable} }

\begin{longtable}{>{\raggedright\arraybackslash}p{3cm}p{12cm}}
\multicolumn{2}{c}{EUPS dependencies} \\ \hline
\textbf{name} & \textbf{description} \\ \hline
{\footnotesize fitsio } &
 \\ \hline
\end{longtable}

README.md (First 20 lines only)
{\scriptsize
\begin{lstlisting}[breaklines]
# HeaderService

Development for LSST Meta-data FITS header service

Description
-----------

This is the development for the LSST Meta-data FITS header client. It
uses a set of FITS header library templates and DDS/SAL Python-based
communication layer to populate meta-data and command the header
client to write header files.

Requirements
------------
+ numpy
+ astropy
+ fitsio (https://github.com/esheldon/fitsio)
+ salobj
+ OpenSplice compiled binaries for centOS7
+ A CentOS7 VM or docker container
\end{lstlisting}
}

\begin{figure}
\begin{center}
\includegraphics[max width=\linewidth]{dot/headerservice.dot.ps}
\caption{Git packages dependency tree for \textbf{ lsst-dm/HeaderService }.}
\end{center}
\end{figure}


\newpage
\subsection{ctrl\_oods}\label{lsst-dm/ctrl_oods}

Observatory Operations Data Service


{\footnotesize
\begin{longtable}{rl}
\hline
Open it in GitHUb: & \href{https://github.com/lsst-dm/ctrl_oods}{https://github.com/lsst-dm/ctrl\_oods} \\ \cdashline{1-2}
Top Level Component: & \hyperlink{oods}{Observatory Operations Data Service SW} \\
\hline
\end{longtable} }

\begin{longtable}{>{\raggedleft\arraybackslash}p{3cm}p{12cm}}
\multicolumn{2}{c}{EUPS dependencies} \\ \hline
\textbf{name} & \textbf{description} \\ \hline
{\footnotesize base } &
{\footnotesize C++/Python import utilities and doxygen configuration for LSST Data
Management.
 }
 \\ \hline
{\footnotesize utils } &
{\footnotesize Common code, floating-point utilities, and angle stringification for
LSST Data Management.
 }
 \\ \hline
{\footnotesize sconsUtils } &
{\footnotesize Build system for LSST Data Management packages with standard layout.
 }
 \\ \hline
{\footnotesize obs\_lsst } &
{\footnotesize The obs model for the LSST cameras
 }
 \\ \hline
\end{longtable}

README.rst (First 20 lines only)
{\scriptsize
\begin{lstlisting}[breaklines]
#########
ctrl_oods
#########

``ctrl_oods`` is an `LDF Prompt Enclave Software`_ package.

.. Add a brief (few sentence) description of what this package provides.

The Observatory Operations Data Service watches for files in one or more directories, and then ingests them into an LSST Butler repository.   
Files are expired from the repository at specified intervals.
\end{lstlisting}
}

\begin{figure}
\begin{center}
\includegraphics[max width=\linewidth]{dot/ctrl_oods.dot.ps}
\caption{Git packages dependency tree for \textbf{ lsst-dm/ctrl\_oods }.}
\end{center}
\end{figure}


\newpage
\subsection{ctrl\_iip}\label{ctrl_iip}

Image ingest and processing


{\footnotesize
\begin{longtable}{rl}
\hline
Open it in GitHUb: & \href{https://github.com/lsst/ctrl_iip}{https://github.com/lsst/ctrl\_iip} \\ \cdashline{1-2}
Top Level Component: & \hyperlink{iip}{Image Ingest and Processing} \\
\cdashline{1-2}
{GitHub Teams:} &
 Data Management \\
\hline
\end{longtable} }


README.md (First 20 lines only)
{\scriptsize
\begin{lstlisting}[breaklines]
# ctrl_iip
Image ingest and processing

environment variables:

Set CTRL_IIP_DIR to the root of this repository. (this will be set automatically
when this is integrated with the DM system)

Set PYTHONPATH to include $CTRL_IPP_DIR/python



Note about configuration files:

Configuration files are loaded from $CTRL_IIP_DIR/etc/config by default

If the environment variable IIP_CONFIG_DIR is set, it will look in 
this directory for configuration files.
\end{lstlisting}
}



\newpage
\subsection{alert\_stream}\label{lsst-dm/alert_stream}

Mock alert stream distribution system using Kafka producers and
consumers.


{\footnotesize
\begin{longtable}{rl}
\hline
Open it in GitHUb: & \href{https://github.com/lsst-dm/alert_stream}{https://github.com/lsst-dm/alert\_stream} \\ \cdashline{1-2}
Top Level Component: & \hyperlink{alrtdstr}{Alert Distribution SW} \\
\hline
\end{longtable} }


README.rst (First 20 lines only)
{\scriptsize
\begin{lstlisting}[breaklines]
############
alert_stream
############

This package implements the LSST Alert Distribution Service.
The Alert Distribution Service provides a mechanism for rapidly disseminating and filtering notifications of transient and variable sources observed by LSST.
The service is described in detail in `DMTN-093`_.

.. _DMTN-093: https://dmtn-093.lsst.io/

Prerequisites
=============

- Cloning this repository requires `Git LFS`_ (Large File Storage) support.
  Refer to the `DM Developer Guide`_ for more information.

.. _Git LFS: https://git-lfs.github.com
.. _DM Developer Guide: https://developer.lsst.io/git/git-lfs.html
\end{lstlisting}
}



\newpage
\subsection{squash}\label{lsst-sqre/squash}

SQuaSH web interface


{\footnotesize
\begin{longtable}{rl}
\hline
Open it in GitHUb: & \href{https://github.com/lsst-sqre/squash}{https://github.com/lsst-sqre/squash} \\ \cdashline{1-2}
Top Level Component: & \hyperlink{qcsw}{Quality Control SW} \\
\hline
\end{longtable} }


README.md (First 20 lines only)
{\scriptsize
\begin{lstlisting}[breaklines]
# squash

`squash` is the web frontend to embed the bokeh apps and navigate through them. You can learn more about SQuaSH at [SQR-009](https://sqr-009.lsst.io).

[![Build Status](https://travis-ci.org/lsst-sqre/squash.svg?branch=master)](https://travis-ci.org/lsst-sqre/squash)

## Requirements

The `squash` web frontend requires the [squash-restful-api](https://github.com/lsst-sqre/squash-restful-api) and [squash-bokeh](https://github.com/lsst-sqre/squash-bokeh) microservices, and the TLS certificats that are installed by the
[`squash-deployment`](https://github.com/lsst-sqre/squash-deployment).

## Kubernetes deployment

You can provision a Kubernetes cluster in GKE, clone this repo and deploy the `squash` microservice using:

```
cd squash
TAG=latest make service deployment
```

\end{lstlisting}
}



\newpage
\subsection{dbb\_gwclient}\label{lsst-dm/dbb_gwclient}

Prototype code to save raw files to Data Backbone Gateway


{\footnotesize
\begin{longtable}{rl}
\hline
Open it in GitHUb: & \href{https://github.com/lsst-dm/dbb_gwclient}{https://github.com/lsst-dm/dbb\_gwclient} \\ \cdashline{1-2}
Top Level Component: & \hyperlink{dbbmd}{DBB Ingest/ Metadata Management SW} \\
\hline
\end{longtable} }


README.md (First 20 lines only)
{\scriptsize
\begin{lstlisting}[breaklines]
# dbb_gwclient
Prototype code to save raw files to Data Backbone Gateway.

This is a Python 3 only package (we assume Python 3.6 or higher).
\end{lstlisting}
}



\newpage
\subsection{dbb\_gateway}\label{lsst-dm/dbb_gateway}

Prototype code that ingests into the Data Backbone raw files delivered
by the dbb\_gwclient to the DBB gateway


{\footnotesize
\begin{longtable}{rl}
\hline
Open it in GitHUb: & \href{https://github.com/lsst-dm/dbb_gateway}{https://github.com/lsst-dm/dbb\_gateway} \\ \cdashline{1-2}
Top Level Component: & \hyperlink{dbbmd}{DBB Ingest/ Metadata Management SW} \\
\hline
\end{longtable} }

\begin{longtable}{>{\raggedright\arraybackslash}p{3cm}p{12cm}}
\multicolumn{2}{c}{EUPS dependencies} \\ \hline
\textbf{name} & \textbf{description} \\ \hline
{\footnotesize pyfits } &
{\footnotesize  }
 \\ \hline
\end{longtable}

README.md (First 20 lines only)
{\scriptsize
\begin{lstlisting}[breaklines]
# dbb_gateway
Prototype code that ingests into the Data Backbone raw files delivered by the dbb_gwclient to the DBB gateway
\end{lstlisting}
}

\begin{figure}
\begin{center}
\includegraphics[max width=\linewidth]{dot/dbb_gateway.dot.ps}
\caption{Git packages dependency tree for \textbf{ lsst-dm/dbb\_gateway }.}
\end{center}
\end{figure}


\newpage
\subsection{suit}\label{suit}



{\footnotesize
\begin{longtable}{rl}
\hline
Open it in GitHUb: & \href{https://github.com/lsst/suit}{https://github.com/lsst/suit} \\ \cdashline{1-2}
Top Level Component: & \hyperlink{prtlsw}{LSP Portal Software} \\
\cdashline{1-2}
{GitHub Teams:} &
 Data Management \\
\hline
\end{longtable} }


README.md (First 20 lines only)
{\scriptsize
\begin{lstlisting}[breaklines]
# SUIT 


## Description
The SUIT (Science User Interface and Tools) repository contains applications built on the Firefly Toolkit.
It is meant to be used with [Firefly](https://github.com/Caltech-IPAC/firefly).

The principal current application is "suit", otherwise known as the Portal Aspect application, which
contains both the Portal search screens and visualization capabilities, and the "slate" endpoint that
is used for Python-based image and table visualizations.


## Build Instuctions
 
 - Install [JDK 8](http://www.oracle.com/technetwork/java/javase/downloads/jdk8-downloads-2133151.html)   
   
 - Install [Gradle 4.x](https://gradle.org/install/)
 
 - Install [Node.js 8.x](https://nodejs.org/en/download/)
 
\end{lstlisting}
}



\newpage
\subsection{jupyterhubutils}\label{lsst-sqre/jupyterhubutils}

Utilities for LSST LSP notebook environment (Hub/spawner side)


{\footnotesize
\begin{longtable}{rl}
\hline
Open it in GitHUb: & \href{https://github.com/lsst-sqre/jupyterhubutils}{https://github.com/lsst-sqre/jupyterhubutils} \\ \cdashline{1-2}
Top Level Component: & \hyperlink{nbsw}{LSP Notebook Software} \\
\hline
\end{longtable} }


README.md (First 20 lines only)
{\scriptsize
\begin{lstlisting}[breaklines]
# Utilities for LSST LSP notebook environment (Hub/spawner side)
\end{lstlisting}
}



\newpage
\subsection{jupyterlabutils}\label{lsst-sqre/jupyterlabutils}

Utilities for JupyterLab containers in LSST Science Platform environment


{\footnotesize
\begin{longtable}{rl}
\hline
Open it in GitHUb: & \href{https://github.com/lsst-sqre/jupyterlabutils}{https://github.com/lsst-sqre/jupyterlabutils} \\ \cdashline{1-2}
Top Level Component: & \hyperlink{nbsw}{LSP Notebook Software} \\
\hline
\end{longtable} }


README.md (First 20 lines only)
{\scriptsize
\begin{lstlisting}[breaklines]
# Utilities for LSST LSP Science Platform notebook aspect (user pod side)
\end{lstlisting}
}



\newpage
\subsection{suit-onlinehelp}\label{suit-onlinehelp}



{\footnotesize
\begin{longtable}{rl}
\hline
Open it in GitHUb: & \href{https://github.com/lsst/suit-onlinehelp}{https://github.com/lsst/suit-onlinehelp} \\ \cdashline{1-2}
Top Level Component: & \hyperlink{prtloh}{LSP Portal Online Help} \\
\hline
\end{longtable} }


README.md (First 20 lines only)
{\scriptsize
\begin{lstlisting}[breaklines]
# suit-onlinehelp

Prerequisites
-------------
    - gradle v2.2+
    - clone onlinehelp repository and its dependent repository
        - git clone https://github.com/lsst/suit-onlinehelp
        - git clone https://github.com/Caltech-IPAC/firefly


Build and Install Individually
------------------------------
- cd suit-onlinehelp
- gradle :<project_name>:build      // build only
    - creates an archive of html and supporting files to be install to a webserver
    - the file is placed in ./build/libs/

- gradle :<project_name>:install    // build and install.
    - crates and install online help files
    - HTML_DOC_ROOT environment variable is required to locate the path to the webserver's document root.
\end{lstlisting}
}



\newpage
\subsection{dax\_imgserv}\label{dax_imgserv}

Web Interface for LSST Image Services


{\footnotesize
\begin{longtable}{rl}
\hline
Open it in GitHUb: & \href{https://github.com/lsst/dax_imgserv}{https://github.com/lsst/dax\_imgserv} \\ \cdashline{1-2}
Top Level Component: & \hyperlink{daximg}{Image/ Cutout Server} \\
\cdashline{1-2}
{GitHub Teams:} &
 Overlords \\
 & Data Management \\
 & Database \\
\hline
\end{longtable} }


README.txt (First 20 lines only)
{\scriptsize
\begin{lstlisting}[breaklines]
# Useful link:
http://blog.miguelgrinberg.com/post/designing-a-restful-api-with-python-and-flask

# To install flask:
sudo aptitude install python-flask

# To run some quick tests:

  # run the server
  python bin/imageServer.py

  # and fetch the urls:
  http://localhost:5000/api/image/soda/availability
  http://localhost:5000/api/image/soda/capabilities
  http://localhost:5000/api/image/soda/examples
  http://localhost:5000/api/image/soda/sync?ID=DC_W13_Stripe82.calexp.r&POS=CIRCLE+37.644598+0.104625+100
  http://localhost:5000/api/image/soda/sync?ID=DC_W13_Stripe82.calexp.r&POS=RANGE+37.616820222+37.67235778+0.07684722222+0.132402777
  http://localhost:5000/api/image/soda/sync?ID=DC_W13_Stripe82.calexp.r&POS=POLYGON+37.6580803+0.0897081+37.6580803+0.1217858+37.6186104+0.1006648
  http://localhost:5000/api/image/soda/sync?ID=DC_W13_Stripe82.calexp.r&POS=BRECT+37.644598+0.104625+100+100+pixel
\end{lstlisting}
}



\newpage
\subsection{ap\_pipe}\label{ap_pipe}

LSST Data Management Alert Production Pipeline


{\footnotesize
\begin{longtable}{rl}
\hline
Open it in GitHUb: & \href{https://github.com/lsst/ap_pipe}{https://github.com/lsst/ap\_pipe} \\ \cdashline{1-2}
Top Level Component: & \hyperlink{apprmpt}{Alert Production} \\
\cdashline{1-2}
{GitHub Teams:} &
 Overlords \\
 & Data Management \\
\hline
\end{longtable} }

\begin{longtable}{>{\raggedright\arraybackslash}p{3cm}p{12cm}}
\multicolumn{2}{c}{EUPS dependencies} \\ \hline
\textbf{name} & \textbf{description} \\ \hline
{\footnotesize utils } &
{\footnotesize Common code, floating-point utilities, and angle stringification for
LSST Data Management.
 }
 \\ \hline
{\footnotesize pex\_config } &
{\footnotesize Configuration interface and history-tracking for LSST Data Management.
 }
 \\ \hline
{\footnotesize pipe\_base } &
{\footnotesize LSST Data Management: base classes for data processing tasks
 }
 \\ \hline
{\footnotesize pipe\_tasks } &
{\footnotesize LSST Data Management: astronomical data processing tasks
 }
 \\ \hline
{\footnotesize ap\_association } &
{\footnotesize Repository for holding code related to Alert Production difference
source association
 }
 \\ \hline
\end{longtable}

README.md (First 20 lines only)
{\scriptsize
\begin{lstlisting}[breaklines]
# ap_pipe

This package contains the LSST Data Management Alert Production Pipeline.

For up-to-date documentation, including a tutorial, see the `doc` directory.

ap_pipe processes raw images that have been ingested into a Butler repository
with corresponding calibration products and templates. It produces calexps,
difference images and source catalogs, and an association database.

The user must specify the main repository with ingested images (and the
location of the calibration products and templates if they reside elsewhere),
the name of the association database (may be either created from scratch or
connected to for continued associating), and a Butler data ID.
\end{lstlisting}
}

\begin{figure}
\begin{center}
\includegraphics[max width=\linewidth]{dot/ap_pipe.dot.ps}
\caption{Git packages dependency tree for \textbf{ ap\_pipe }.}
\end{center}
\end{figure}


\newpage
\subsection{cp\_pipe}\label{cp_pipe}

Calibration-products production pipeline


{\footnotesize
\begin{longtable}{rl}
\hline
Open it in GitHUb: & \href{https://github.com/lsst/cp_pipe}{https://github.com/lsst/cp\_pipe} \\ \cdashline{1-2}
Top Level Component: & \hyperlink{dmcal}{Calibration Software} \\
\cdashline{1-2}
{GitHub Teams:} &
 Overlords \\
 & Data Management \\
 & DMLT \\
\hline
\end{longtable} }

\begin{longtable}{>{\raggedright\arraybackslash}p{3cm}p{12cm}}
\multicolumn{2}{c}{EUPS dependencies} \\ \hline
\textbf{name} & \textbf{description} \\ \hline
{\footnotesize pex\_config } &
{\footnotesize Configuration interface and history-tracking for LSST Data Management.
 }
 \\ \hline
{\footnotesize pipe\_base } &
{\footnotesize LSST Data Management: base classes for data processing tasks
 }
 \\ \hline
{\footnotesize log } &
{\footnotesize LSST DM Logging for C++ and Python
 }
 \\ \hline
{\footnotesize ip\_isr } &
{\footnotesize LSST data management: instrument signature removal (detrending) for
astronomical images
 }
 \\ \hline
{\footnotesize afw } &
{\footnotesize LSST data management: pipeline library code and primitives including
images and tables
 }
 \\ \hline
{\footnotesize meas\_algorithms } &
{\footnotesize LSST Data Management: astronomical measurement algorithms
 }
 \\ \hline
{\footnotesize pipe\_drivers } &
{\footnotesize LSST Data Management: high level task coordination scripts
 }
 \\ \hline
{\footnotesize pipe\_tasks } &
{\footnotesize LSST Data Management: astronomical data processing tasks
 }
 \\ \hline
\end{longtable}

README.rst (First 20 lines only)
{\scriptsize
\begin{lstlisting}[breaklines]
#######################################
Calibration Products Production Package
#######################################

Code to produce calibration products, required to perform ISR and other calibration tasks.
\end{lstlisting}
}

\begin{figure}
\begin{center}
\includegraphics[max width=\linewidth]{dot/cp_pipe.dot.ps}
\caption{Git packages dependency tree for \textbf{ cp\_pipe }.}
\end{center}
\end{figure}


\newpage
\subsection{mops\_daymops}\label{mops_daymops}



{\footnotesize
\begin{longtable}{rl}
\hline
Open it in GitHUb: & \href{https://github.com/lsst/mops_daymops}{https://github.com/lsst/mops\_daymops} \\ \cdashline{1-2}
Top Level Component: & \hyperlink{ssp}{Solar System processing and Forced Photometry} \\
\cdashline{1-2}
{GitHub Teams:} &
 Overlords \\
 & Data Management \\
\hline
\end{longtable} }

\begin{longtable}{>{\raggedright\arraybackslash}p{3cm}p{12cm}}
\multicolumn{2}{c}{EUPS dependencies} \\ \hline
\textbf{name} & \textbf{description} \\ \hline
{\footnotesize scons } &
{\footnotesize  }
 \\ \hline
{\footnotesize swig } &
{\footnotesize  }
 \\ \hline
{\footnotesize gsl } &
{\footnotesize  }
 \\ \hline
{\footnotesize daf\_base } &
{\footnotesize Low-level data structures, including memory-management helpers
(Citizen), mappings (PropertySet, PropertyList), and DateTime.
 }
 \\ \hline
{\footnotesize pex\_exceptions } &
{\footnotesize Exception base classes and common exceptions, including C++-Python
exception translation for LSST Data Management.
 }
 \\ \hline
{\footnotesize slalib } &
 \\ \hline
{\footnotesize eigen } &
{\footnotesize  }
 \\ \hline
{\footnotesize mysqlpython } &
{\footnotesize  }
 \\ \hline
{\footnotesize numpy } &
{\footnotesize  }
 \\ \hline
\end{longtable}

README.quickstart.txt (First 20 lines only)
{\scriptsize
\begin{lstlisting}[breaklines]
Here's a greatly simplified guide to running MOPS at the moment, which
will run findTracklets, collapseTracklets linkTracklets for you.

I was using Bash when I came up with these, you may need to change a
few things if you're using *csh.

# set up your environment
setlsst
setup mysqlpython
setup mops_daymops
export MOPS_HACKS=$MOPS_DAYMOPS_DIR/tests/experimentScripts/

# get data
mkdir myMopsRun
cd myMopsRun
wget --user=USER --password=PASSWORD dias_pt1_nodeep.short.astromErr


# populate the DB for later. I assume you have the OpSim DB already.
echo "CREATE DATABASE myMops; USE myMops; `cat fullerDiaSource.sql`;" | mysql
\end{lstlisting}
}
README.txt (First 20 lines only)
{\scriptsize
\begin{lstlisting}[breaklines]
Jmyers Oct 22

Updated thoroughly to describe how I'm currently doing things.

The following is a set of instructions for running
find/collapse/linkTracklets on some diaSources. 

In the future these scripts (or more likely, better versions of
all of this) will be modified so that pipelines can run each
stage of find/collapse/linkTracklets on particular sets of data.

All the scripts should be in the same directory as this readme file.



INSTALLING/BUILDING C++ FIND/LINKTRACKLETS (etc.)
---------------------------------------

Build the C++ tools using instructions online at http://dev.lsstcorp.org/trac/wiki/MOPS/Installing_MOPS

\end{lstlisting}
}

\begin{figure}
\begin{center}
\includegraphics[max width=\linewidth]{dot/mops_daymops.dot.ps}
\caption{Git packages dependency tree for \textbf{ mops\_daymops }.}
\end{center}
\end{figure}


\newpage
\subsection{lsst\_distrib}\label{lsst_distrib}



{\footnotesize
\begin{longtable}{rl}
\hline
Open it in GitHUb: & \href{https://github.com/lsst/lsst_distrib}{https://github.com/lsst/lsst\_distrib} \\ \cdashline{1-2}
Top Level Component: & \hyperlink{spdist}{Science Pipelines Distribution} \\
\cdashline{1-2}
{GitHub Teams:} &
 Overlords \\
 & Data Management \\
\hline
\end{longtable} }

\begin{longtable}{>{\raggedright\arraybackslash}p{3cm}p{12cm}}
\multicolumn{2}{c}{EUPS dependencies} \\ \hline
\textbf{name} & \textbf{description} \\ \hline
{\footnotesize lsst\_apps } &
{\footnotesize  }
 \\ \hline
{\footnotesize ctrl\_execute } &
{\footnotesize LSST Data Management orchestration execution wrapper
 }
 \\ \hline
{\footnotesize ctrl\_mpexec } &
{\footnotesize Execution framework for PipelineTask
 }
 \\ \hline
{\footnotesize ctrl\_platform\_lsstvc } &
{\footnotesize Verification cluster configuration and execution files
 }
 \\ \hline
{\footnotesize jointcal } &
{\footnotesize Simultaneous astrometry and photometry
 }
 \\ \hline
{\footnotesize verify } &
{\footnotesize lsst.verify - LSST Science Pipelines Verification Framework
 }
 \\ \hline
{\footnotesize ap\_verify } &
{\footnotesize Verification test framework for DM Alert Production
 }
 \\ \hline
{\footnotesize display\_firefly } &
{\footnotesize Interface between afw and firefly
 }
 \\ \hline
{\footnotesize display\_matplotlib } &
{\footnotesize afwDisplay using matplotlib as a backend
 }
 \\ \hline
{\footnotesize cp\_pipe } &
{\footnotesize Calibration-products production pipeline
 }
 \\ \hline
{\footnotesize sdm\_schemas } &
{\footnotesize  }
 \\ \hline
{\footnotesize validate\_drp } &
{\footnotesize Validate an output data repository against LSST Science Requirements
Document Key Performance Metrics.
 }
 \\ \hline
{\footnotesize fgcmcal } &
{\footnotesize Global Photometric Calibration in LSST with FGCM
 }
 \\ \hline
\end{longtable}


\begin{figure}
\begin{center}
\includegraphics[max width=\linewidth]{dot/lsst_distrib.dot.ps}
\caption{Git packages dependency tree for \textbf{ lsst\_distrib }.}
\end{center}
\end{figure}


\newpage
\subsection{lsst\_apps}\label{lsst_apps}



{\footnotesize
\begin{longtable}{rl}
\hline
Open it in GitHUb: & \href{https://github.com/lsst/lsst_apps}{https://github.com/lsst/lsst\_apps} \\ \cdashline{1-2}
Top Level Component: & \hyperlink{scipipe}{Science Pipelines Libraries} \\
\cdashline{1-2}
{GitHub Teams:} &
 Overlords \\
 & Data Management \\
\hline
\end{longtable} }

\begin{longtable}{>{\raggedright\arraybackslash}p{3cm}p{12cm}}
\multicolumn{2}{c}{EUPS dependencies} \\ \hline
\textbf{name} & \textbf{description} \\ \hline
{\footnotesize meas\_deblender } &
{\footnotesize LSST Data Management: astronomical source deblender
 }
 \\ \hline
{\footnotesize meas\_modelfit } &
{\footnotesize LSST Data Management: model fitting algorithms
 }
 \\ \hline
{\footnotesize pipe\_tasks } &
{\footnotesize LSST Data Management: astronomical data processing tasks
 }
 \\ \hline
{\footnotesize ap\_pipe } &
{\footnotesize LSST Data Management Alert Production Pipeline
 }
 \\ \hline
{\footnotesize obs\_sdss } &
{\footnotesize SDSS-specific configuration and tasks for the LSST Data Management Stack
 }
 \\ \hline
{\footnotesize obs\_test } &
{\footnotesize Configuration and tasks for a test camera on the LSST Data Management
Stack
 }
 \\ \hline
{\footnotesize meas\_extensions\_simpleShape } &
{\footnotesize LSST Data Management: Measure second moments with a pre-defined circular
Gaussian weighting
 }
 \\ \hline
\end{longtable}

README.md (First 20 lines only)
{\scriptsize
\begin{lstlisting}[breaklines]
# lsst_apps

This is a metapackage providing a minimalist set of [LSST Science Pipelines](https://pipelines.lsst.io) packages. This is a subset of the full [lsst_distrib](https://github.com/lsst/lsst_distrib) package. This dependency list is tracked via [eups](https://github.com/RobertLuptonTheGood/eups) in the `ups/lsst_apps.table` file.
\end{lstlisting}
}

\begin{figure}
\begin{center}
\includegraphics[max width=\linewidth]{dot/lsst_apps.dot.ps}
\caption{Git packages dependency tree for \textbf{ lsst\_apps }.}
\end{center}
\end{figure}


\newpage
\subsection{daf\_butler}\label{daf_butler}

Prototype for data access framework described in \citeds{DMTN-056}

{\footnotesize
\begin{longtable}{rl}
\hline
Open it in GitHUb: & \href{https://github.com/lsst/daf_butler}{https://github.com/lsst/daf\_butler} \\ \cdashline{1-2}
Top Level Component: & \hyperlink{butler}{Data Butler} \\
\cdashline{1-2}
{GitHub Teams:} &
 Overlords \\
 & Data Management \\
 & Database \\
\hline
\end{longtable} }

\begin{longtable}{>{\raggedright\arraybackslash}p{3cm}p{12cm}}
\multicolumn{2}{c}{EUPS dependencies} \\ \hline
\textbf{name} & \textbf{description} \\ \hline
{\footnotesize sphgeom } &
{\footnotesize C++ spherical geometry primitives for LSST Data Management
 }
 \\ \hline
{\footnotesize sconsUtils } &
{\footnotesize Build system for LSST Data Management packages with standard layout.
 }
 \\ \hline
{\footnotesize utils } &
{\footnotesize Common code, floating-point utilities, and angle stringification for
LSST Data Management.
 }
 \\ \hline
\end{longtable}

README.md (First 20 lines only)
{\scriptsize
\begin{lstlisting}[breaklines]
# daf_butler

LSST Data Access framework described in [DMTN-056](https://dmtn-056.lsst.io).

This is a **Python 3 only** package (we assume Python 3.6 or higher).
\end{lstlisting}
}

\begin{figure}
\begin{center}
\includegraphics[max width=\linewidth]{dot/daf_butler.dot.ps}
\caption{Git packages dependency tree for \textbf{ daf\_butler }.}
\end{center}
\end{figure}


\newpage
\subsection{pipe\_supertask}\label{pipe_supertask}

Super Task implementation


{\footnotesize
\begin{longtable}{rl}
\hline
Open it in GitHUb: & \href{https://github.com/lsst/pipe_supertask}{https://github.com/lsst/pipe\_supertask} \\ \cdashline{1-2}
Top Level Component: & \hyperlink{txf}{Task Framework} \\
\hline
\end{longtable} }

\begin{longtable}{>{\raggedright\arraybackslash}p{3cm}p{12cm}}
\multicolumn{2}{c}{EUPS dependencies} \\ \hline
\textbf{name} & \textbf{description} \\ \hline
{\footnotesize daf\_butler } &
{\footnotesize Prototype for data access framework described in \citeds{DMTN-056} }
 \\ \hline
{\footnotesize log } &
{\footnotesize LSST DM Logging for C++ and Python
 }
 \\ \hline
{\footnotesize pex\_config } &
{\footnotesize Configuration interface and history-tracking for LSST Data Management.
 }
 \\ \hline
{\footnotesize pipe\_base } &
{\footnotesize LSST Data Management: base classes for data processing tasks
 }
 \\ \hline
\end{longtable}

README.md (First 20 lines only)
{\scriptsize
\begin{lstlisting}[breaklines]
# pipe_supertask
Super Task implementation
\end{lstlisting}
}

\begin{figure}
\begin{center}
\includegraphics[max width=\linewidth]{dot/pipe_supertask.dot.ps}
\caption{Git packages dependency tree for \textbf{ pipe\_supertask }.}
\end{center}
\end{figure}


\newpage
\subsection{qserv}\label{qserv}

LSST Query Services


{\footnotesize
\begin{longtable}{rl}
\hline
Open it in GitHUb: & \href{https://github.com/lsst/qserv}{https://github.com/lsst/qserv} \\ \cdashline{1-2}
Top Level Component: & \hyperlink{qserv}{Distributed Database} \\
\cdashline{1-2}
{GitHub Teams:} &
 Overlords \\
 & Data Management \\
\hline
\end{longtable} }

\begin{longtable}{>{\raggedright\arraybackslash}p{3cm}p{12cm}}
\multicolumn{2}{c}{EUPS dependencies} \\ \hline
\textbf{name} & \textbf{description} \\ \hline
{\footnotesize antlr4 } &
{\footnotesize antlr4 third party package for lsst
 }
 \\ \hline
{\footnotesize db } &
{\footnotesize Lightweight database interface wrappers and abstraction layer for LSST
Data Management code, currently limited to MySQL.
 }
 \\ \hline
{\footnotesize log } &
{\footnotesize LSST DM Logging for C++ and Python
 }
 \\ \hline
{\footnotesize lua } &
{\footnotesize  }
 \\ \hline
{\footnotesize mariadb } &
{\footnotesize  }
 \\ \hline
{\footnotesize mysqlproxy } &
{\footnotesize  }
 \\ \hline
{\footnotesize partition } &
{\footnotesize Spatial Data Partitioning for LSST Qserv
 }
 \\ \hline
{\footnotesize redis\_plus\_plus } &
{\footnotesize redis-plus-plus third party package for Vera Rubin Observatory
 }
 \\ \hline
{\footnotesize scisql } &
{\footnotesize  }
 \\ \hline
{\footnotesize sphgeom } &
{\footnotesize C++ spherical geometry primitives for LSST Data Management
 }
 \\ \hline
{\footnotesize xrootd } &
{\footnotesize The XRootD central repository
 }
 \\ \hline
\end{longtable}

README.md (First 20 lines only)
{\scriptsize
\begin{lstlisting}[breaklines]
# Qserv: petascale distributed database

## Master branch status

Continuous integration server launches Qserv build and also multi-node integration tests:

[![Build Status](https://travis-ci.org/lsst/qserv.svg?branch=master)](https://travis-ci.org/lsst/qserv)

[![Code Climate](https://codeclimate.com/github/lsst/qserv/badges/gpa.svg)](https://codeclimate.com/github/lsst/qserv)

[![Issue Count](https://codeclimate.com/github/lsst/qserv/badges/issue_count.svg)](https://codeclimate.com/github/lsst/qserv)

## Documentation

[Documentation for master branch](https://qserv.lsst.io/)



\end{lstlisting}
}

\begin{figure}
\begin{center}
\includegraphics[max width=\linewidth]{dot/qserv.dot.ps}
\caption{Git packages dependency tree for \textbf{ qserv }.}
\end{center}
\end{figure}


\newpage
\subsection{albuquery}\label{albuquery}

DAX Query Services in Kotlin


{\footnotesize
\begin{longtable}{rl}
\hline
Open it in GitHUb: & \href{https://github.com/lsst/albuquery}{https://github.com/lsst/albuquery} \\ \cdashline{1-2}
Top Level Component: & \hyperlink{adql}{ADQL Translator} \\
\hline
\end{longtable} }


README.rst (First 20 lines only)
{\scriptsize
\begin{lstlisting}[breaklines]
#########
albuquery
#########

``albuquery`` will implement a TAP database query service for the Web APIs Aspect of the LSST Science Platform (a.k.a. Data Access Services/DAX).
\end{lstlisting}
}



\newpage
\subsection{scipipe\_conda\_env}\label{scipipe_conda_env}

Conda environment for LSST Science Pipelines


{\footnotesize
\begin{longtable}{rl}
\hline
Open it in GitHUb: & \href{https://github.com/lsst/scipipe_conda_env}{https://github.com/lsst/scipipe\_conda\_env} \\ \cdashline{1-2}
Top Level Component: & \hyperlink{splce}{SciencePipelines Conda Env.} \\
\cdashline{1-2}
{GitHub Teams:} &
 Data Management \\
 & DM Auxilliaries \\
\hline
\end{longtable} }


README.md (First 20 lines only)
{\scriptsize
\begin{lstlisting}[breaklines]
# Conda Environment for Science Pipelines

This repository contains the definition of the Conda environment used by the LSST Science Pipelines.


## Contents

There are two core types of files in the `etc` directory.

- `bleed` files, in the pattern of `conda3_bleed-<platform>-64.yml`, indicating the names of packages on which the 
  Science Pipelines directly depend, or
- `lock` files, in the pattern of `conda-<platform>-64.lock`, indicating a specific versioned set of those packages, 
  and packages upon which they depend, which can be directly instantiated as a conda environment;

Where `<platform>` is one of:

- `linux`, indicating that this file has been tested on CentOS (our reference platform), and, by extension, is appropriate for use on a Linux systems;
- `osx`, indicating that this file has been tested on macOS.

### Lock files
\end{lstlisting}
}



\newpage
\subsection{astrometry\_net\_data}\label{astrometry_net_data}



{\footnotesize
\begin{longtable}{rl}
\hline
Open it in GitHUb: & \href{https://github.com/lsst/astrometry_net_data}{https://github.com/lsst/astrometry\_net\_data} \\ \cdashline{1-2}
Top Level Component: & \hyperlink{and}{Astrometry.net Data} \\
\cdashline{1-2}
{GitHub Teams:} &
 Overlords \\
 & Data Management \\
 & DM Externals \\
\hline
\end{longtable} }

\begin{longtable}{>{\raggedright\arraybackslash}p{3cm}p{12cm}}
\multicolumn{2}{c}{EUPS dependencies} \\ \hline
\textbf{name} & \textbf{description} \\ \hline
{\footnotesize sconsUtils } &
{\footnotesize Build system for LSST Data Management packages with standard layout.
 }
 \\ \hline
\end{longtable}


\begin{figure}
\begin{center}
\includegraphics[max width=\linewidth]{dot/astrometry_net_data.dot.ps}
\caption{Git packages dependency tree for \textbf{ astrometry\_net\_data }.}
\end{center}
\end{figure}


\newpage
\subsection{lsst\_build}\label{lsst_build}



{\footnotesize
\begin{longtable}{rl}
\hline
Open it in GitHUb: & \href{https://github.com/lsst/lsst_build}{https://github.com/lsst/lsst\_build} \\ \cdashline{1-2}
Top Level Component: & \hyperlink{lbld}{lsst\_build} \\
\cdashline{1-2}
{GitHub Teams:} &
 Overlords \\
 & Data Management \\
 & DM Auxilliaries \\
\hline
\end{longtable} }


README.md (First 20 lines only)
{\scriptsize
\begin{lstlisting}[breaklines]
lsst-build, a builder and continuous integration tool for LSST
==============================================================

[![Build Status](https://travis-ci.org/lsst/lsst_build.svg?branch=master)](https://travis-ci.org/lsst/lsst_build)

Provides the following capabilities:

* Given one or more top-level packages, intelligently clone their git
  repositories and check out the requested branches into a build directory:

  ```bash
  lsst-build prepare
     [--repository-pattern=format_pattern_for_repo_URLs]
     [--exclusion-map=exclusions.txt]
     [--version-git-repo=versiondbdir]
     [--ref=branch1 [--ref=branch2 [...]]]
     <builddir> <product1> [product2 [product3 [...]]]
  ```

  Run `lsst-build prepare -h` to see the full list of options.
\end{lstlisting}
}



\newpage
\subsection{jenkins-dm-jobs}\label{lsst-dm/jenkins-dm-jobs}

Jenkins jobs and pipeline scripts for LSST DM


{\footnotesize
\begin{longtable}{rl}
\hline
Open it in GitHUb: & \href{https://github.com/lsst-dm/jenkins-dm-jobs}{https://github.com/lsst-dm/jenkins-dm-jobs} \\ \cdashline{1-2}
Top Level Component: & \hyperlink{jscr}{jenkins scripting} \\
\hline
\end{longtable} }


README.md (First 20 lines only)
{\scriptsize
\begin{lstlisting}[breaklines]
jenkins-dm-jobs
===

[![Build Status](https://travis-ci.org/lsst-dm/jenkins-dm-jobs.png)](https://travis-ci.org/lsst-dm/jenkins-dm-jobs)
\end{lstlisting}
}



\newpage
\subsection{sqre-codekit}\label{lsst-sqre/sqre-codekit}

LSST DM SQuaRE misc. code management tools


{\footnotesize
\begin{longtable}{rl}
\hline
Open it in GitHUb: & \href{https://github.com/lsst-sqre/sqre-codekit}{https://github.com/lsst-sqre/sqre-codekit} \\ \cdashline{1-2}
Top Level Component: & \hyperlink{cdkt}{codekit} \\
\hline
\end{longtable} }


README.md (First 20 lines only)
{\scriptsize
\begin{lstlisting}[breaklines]
[![Build Status](https://travis-ci.org/lsst-sqre/sqre-codekit.svg?branch=master)](https://travis-ci.org/lsst-sqre/sqre-codekit)

# sqre-codekit

LSST DM SQuaRE misc. code management tools

## Installation

sqre-codekit runs on Python 3.6 or newer. You can install it with

```bash
pip install sqre-codekit
```

## Available commands

- `github-auth`: Generate a GitHub authentication token.
- `github-decimate-org`: Delete repos and/or teams from a GitHub organization.
- `github-fork-org`: Fork repositories from one GitHub organization to another.
- `github-get-ratelimit`: Display the current github ReST API request ratelimit.
\end{lstlisting}
}



\newpage
\subsection{lsstsw}\label{lsstsw}

loadLSST


{\footnotesize
\begin{longtable}{rl}
\hline
Open it in GitHUb: & \href{https://github.com/lsst/lsstsw}{https://github.com/lsst/lsstsw} \\ \cdashline{1-2}
Top Level Component: & \hyperlink{lsstsw}{lsstsw} \\
\cdashline{1-2}
{GitHub Teams:} &
 Overlords \\
 & Data Management \\
\hline
\end{longtable} }


README.md (First 20 lines only)
{\scriptsize
\begin{lstlisting}[breaklines]
LSST Distribution Server Account
================================

[![Build Status](https://travis-ci.org/lsst/lsstsw.png)](https://travis-ci.org/lsst/lsstsw)

**`repos.yaml` has been migrated to [`lsst/repos`](https://github.com/lsst/repos).**

For a guide to using `lsstsw`, see:

http://developer.lsst.io/en/latest/build-ci/lsstsw.html

*Note: this directory is git managed.*

Structure
---------

| path       | description                                                    |
| :----------|:---------------------------------------------------------------|
| miniconda  | Anaconda Python distribution                                   |
| bin        | software distribution binaries (rebuild, publish)              |
\end{lstlisting}
}







\newpage
\section{Non DM Products}\label{sec:nondm}

This section will list non DM products that are relevant in order to fulfill DMS requirements.


\newpage
\section{DM Jira Components}\label{sec:jiracomponents}

This section will list the components used in the DM Jira project.
Some of them are crealy related to a product included in the product tree or a low level git package.
Other Jira components are not mapped in the above sections, and will be described here.

The information should be extracted from Jira automatically and checks are executed in order to mach it with the information existing in MagicDraw and GitHub.
