\subsection{Batch Prod SW Software Products}
\subsubsection{Campaign Mgmt git packages}
No git packages defined.\subsubsection{Workload/ flow git packages}
No git packages defined.\subsection{DBB SW Software Products}
\subsubsection{DBB Lifetime SW git packages}
No git packages defined.\subsubsection{DBB Meta SW git packages}
\paragraph{dbb\_gwclient}
\textit{Prototype code to save raw files to Data Backbone Gateway}

Github package organization: \textit{lsst-dm}

Readme header:

\textit{No README provided.}

\subsubsection{DBB Transport SW git packages}
No git packages defined.\subsection{LSP SW Software Products}
\subsubsection{LSP JL SW git packages}
\paragraph{jupyterlab*}
\textit{}

Github package organization: \textit{}

Readme header:

\textit{No README provided.}

\subsubsection{LSP Portal SW git packages}
\paragraph{firefly\_client}
\textit{Python API for Firefly}

Github package organization: \textit{lsst}

Readme header:

\begin{verbatim}
# firefly_client

Python API for Firefly, IPAC's Advanced Astronomy Web UI Framework

\end{verbatim}

\subsubsection{LSP Web SW git packages}
\paragraph{dax\_webserv}
\textit{Web Interface for LSST Data Access Services http:/ / dm.lsst.org/ }

Github package organization: \textit{lsst}

Readme header:

\begin{verbatim}
# Useful link:
http://blog.miguelgrinberg.com/post/designing-a-restful-api-with-python-and-flask
http://pycoder.net/bospy/presentation.html

# To install flask:
sudo aptitude install python-flask


# An example Tap query to dbserv (if running locally)
  curl -d 'query=SELECT+ra,decl,filterName+FROM+DC_W13_Stripe82.Science_Ccd_Exposure+WHERE+scienceCcdExposureId=125230127' http://localhost:5000/db/v0/sync\end{verbatim}

\subsection{Prompt SW Software Products}
\subsubsection{Alert Distrib SW git packages}
\paragraph{alert\_stream}
\textit{Mock alert stream distribution system using Kafka producers and
consumers.}

Github package organization: \textit{lsst-dm}

Readme header:

\textit{No README provided.}

\subsubsection{EFD Transform git packages}
No git packages defined.\subsubsection{Header Srv SW git packages}
\paragraph{HeaderService}
\textit{Development for LSST Meta}

Github package organization: \textit{lsst-dm}

Readme header:

\textit{No README provided.}

\subsubsection{Image Ingest SW git packages}
\paragraph{ctrl\_iip}
\textit{Image ingest and processing}

Github package organization: \textit{lsst}

Readme header:

\begin{verbatim}
# ctrl_iip
Image ingest and processing\end{verbatim}

\subsubsection{Plan Obs Pub SW git packages}
No git packages defined.\subsubsection{OCS Batch SW git packages}
No git packages defined.\subsubsection{Obs Ops Data SW git packages}
No git packages defined.\subsection{Sci Pipelines SW Software Products}
\subsubsection{Alert Prod SW git packages}
\paragraph{spl\_ap}
\textit{Science Pipeline Alert Production top level SW package}

Github package organization: \textit{testpackaging}

Readme header:

\textit{No README provided.}

\subsubsection{Calibration SW git packages}
\paragraph{spl\_calibration}
\textit{Science Pipelines Calibration top level SW package}

Github package organization: \textit{testpackaging}

Readme header:

\textit{No README provided.}

\subsubsection{DR Prod SW git packages}
\paragraph{spl\_drp}
\textit{Science Pipelines DRP top level SW package}

Github package organization: \textit{testpackaging}

Readme header:

\textit{No README provided.}

\subsubsection{MOPS SW git packages}
\paragraph{mops\_daymops}
\textit{\_No description, website, or topics provided.\_}

Github package organization: \textit{lsst}

Readme header:

\begin{verbatim}
Jmyers Oct 22

Updated thoroughly to describe how I'm currently doing things.

The following is a set of instructions for running
find/collapse/linkTracklets on some diaSources. 

In the future these scripts (or more likely, better versions of
all of this) will be modified so that pipelines can run each
stage of find/collapse/linkTracklets on particular sets of data.

All the scripts should be in the same directory as this readme file.


\end{verbatim}

\subsubsection{Spec Prog SW git packages}
No git packages defined.\subsubsection{SciencePL Distrib git packages}
\paragraph{lsst\_distrib}
\textit{\_No description, website, or topics provided.\_}

Github package organization: \textit{lsst}

Readme header:

\textit{No README provided.}

\subsubsection{SciencePL Plugins git packages}
\paragraph{spl\_plugins}
\textit{Science Pipelines Plugins meta package}

Github package organization: \textit{testpackaging}

Readme header:

\textit{No README provided.}

\subsubsection{Tmpl Gen SW git packages}
No git packages defined.\subsection{QC Products Software Products}
\subsubsection{Quality Ctrl SW git packages}
\paragraph{validate\_drp}
\textit{Validate an output data repository against LSST Science Requirements
Document Key Performance Metrics.}

Github package organization: \textit{lsst}

Readme header:

\begin{verbatim}
# lsst.validate.drp

**Validate an LSST DM processCcd.py output repository against a set of [LSST Science Requirements Document](https://ls.st/srd) Key Performance Metrics.**

Also assess expected analytic models for photometric and astrometric performance following the [LSST Overview paper](http://arxiv.org/abs/0805.2366v4).

`validate_drp` is part of the LSST Science Pipelines.
You can learn how to install the Pipelines at https://pipelines.lsst.io/install/index.html.

## Quick run with CFHT observations

```
setup validate_drp
validateDrp.py CFHT/output
```\end{verbatim}

\subsection{Supporting SW Software Products}
\subsubsection{ADQL Translator git packages}
\paragraph{albuquery}
\textit{DAX Query Services in Kotlin}

Github package organization: \textit{lsst}

Readme header:

\textit{No README provided.}

\subsubsection{Data Butler git packages}
\paragraph{daf\_butler}
\textit{Prototype for data access framework described in DMTN}

Github package organization: \textit{lsst}

Readme header:

\begin{verbatim}
# daf_butler

LSST Data Access framework described in [DMTN-056](https://dmtn-056.lsst.io).

This is a **Python 3 only** package (we assume Python 3.6 or higher).\end{verbatim}

\subsubsection{Image Server git packages}
\paragraph{dax\_imgserv}
\textit{Web Interface for LSST Image Services http:/ / dm.lsst.org/ }

Github package organization: \textit{lsst}

Readme header:

\begin{verbatim}
# Useful link:
http://blog.miguelgrinberg.com/post/designing-a-restful-api-with-python-and-flask

# To install flask:
sudo aptitude install python-flask

# To run some quick tests:

  # run the server
  python bin/imageServer.py

  # and fetch the urls:
  http://localhost:5000/api/image
  http://localhost:5000//api/image/v1/DC_W13_Strip82?ds=raw&ra=359.195&dec=-0.1055&filter=r
  http://localhost:5000/api/image/v1/DC_W13_Strip82?ds=raw&ra=359.195&dec=-0.1055&filter=r&width=30.0&height=45.0&unit=arcsec\end{verbatim}

\subsubsection{Firefly git packages}
\paragraph{firefly}
\textit{}

Github package organization: \textit{}

Readme header:

\textit{No README provided.}

\subsubsection{Distrib Database git packages}
\paragraph{daf\_ingest}
\textit{\_No description, website, or topics provided.\_}

Github package organization: \textit{lsst}

Readme header:

\textit{No README provided.}

\subsubsection{Sci Pipelines Libs git packages}
\paragraph{spl\_libs}
\textit{Science Pipelines Library top level SW package. It includes all sw
packages that are common between the Science Pipelines SW products}

Github package organization: \textit{testpackaging}

Readme header:

\textit{No README provided.}

\subsubsection{Task Framework git packages}
\paragraph{pipe\_supertask}
\textit{Super Task implementation}

Github package organization: \textit{lsst}

Readme header:

\begin{verbatim}
# pipe_supertask
Super Task implementation\end{verbatim}


